\documentclass{beamer}
\usepackage{graphics}
\usepackage{multirow}
\usepackage{tabto}

\title{Halo Dunia!}
\author{Tim Olimpiade Komputer Indonesia}

\begin{document}

\begin{frame}
\titlepage
\end{frame}

\begin{frame}
\frametitle{Pendahuluan}
Melalui dokumen ini, kalian akan:
\begin{itemize}
	\item Mengenal program, pemrograman, dan bahasa pemrograman
	\item Memahami bagaimana program dieksekusi
	\item Mengenal kompilator
	\item Mengenal bahasa Pascal
	\item Melakukan instalasi perangkat yang dibutuhkan untuk pemrograman Pascal
\end{itemize}
\end{frame}

\begin{frame}
\frametitle{Program}
definisi program + IO
\end{frame}

\begin{frame}
\frametitle{Pemrograman dan Bahasa Pemrograman}
definisi pemrograman
definsii bahasa pemrograman + pakai Pascal
\end{frame}

\begin{frame}
\frametitle{Mekanisme Eksekusi Program}
\begin{itemize}
	\item Pada masa lalu, komputer diprogram dengan menuliskan sederetan instruksi dalam bahasa Assembly.
	\item Bahasa Assembly mudah dimengerti oleh mesin, sehingga mudah untuk dieksekusi. Oleh karena itu, Bahasa Assembly termasuk dalam bahasa pemrograman tingkat rendah (dekat dengan mesin).
	\item Meskipun begitu, membaca dan mengerti alur program Assembly cukup sulit bagi manusia.
\end{itemize}
\end{frame}

\begin{frame}
\frametitle{Mekanisme Eksekusi Program (lanj.)}
\begin{itemize}
	\item Pada tahun 1960-an, mulai diciptakan bahasa pemrograman tingkat tinggi.
	\item Bahasa-bahasa ini lebih mudah dimengerti manusia karena menggunakan frase bahasa sehari-hari, seperti "jika ... maka ...", "lakukan ... hingga tercapai ...", dan sebagainya.
	\item Sayangnya, bahasa pemrograman tingkat tinggi tidak bisa dimengerti secara langsung oleh mesin. Perlu ada penerjemahan bahasa pemrograman tingkat tinggi ke tingkat rendah, sehingga mesin dapat mengerti instruksi-instruksi yang diberikan.
	\item Penerjemahan ini biasa dilakukan oleh program yang berperan sebagai kompilator, intepreter, atau keduanya. Dalam hal ini kita hanya akan membahas tentang kompilator.
\end{itemize}
\end{frame}

\begin{frame}
\frametitle{Kompilator}
\begin{itemize}
	\item Merupakan program komputer yang dapat menerjemahkan bahasa pemrograman tingkat tinggi ke bahasa mesin.
	\item Hasil terjemahan ini yang nantinya dapat dimengerti oleh mesin, sehingga dapat dieksekusi oleh komputer.
	\item Aktivitas menerjemahkan ini disebut dengan kompilasi.
	\item Siklus kerja jika kita menggunakan kompilator adalah: tulis program $\rightarrow$ kompilasi $\rightarrow$ eksekusi.
\end{itemize}

\end{frame}

\begin{frame}
\frametitle{Intepreter}
perlu? kalo iya tambahin juga di tujuan pembelajaran dan mekanisme eksekusi program
\end{frame}

\begin{frame}
\frametitle{IDE}
Apa itu IDE dan kita mau pake apa
\end{frame}

\begin{frame}
\frametitle{Mengapa Pascal?}
\begin{itemize}
	\item Mudah dibaca dan dikelola dibandingkan dengan bahasa C/C++.
	\item Mudah untuk melakukan kompilasi.
	\item Kompilasi berjalan dengan cepat.
\end{itemize}
\end{frame}

\begin{frame}
\frametitle{FreePascal}
\begin{itemize}
	\item Merupakan salah satu kompilator Pascal yang populer.
	\item Program kompilator FreePascal beserta dokumentasinya tersedia gratis.
	\item FreePascal merupakan kompilator resmi yang dipakai pada IOI (\textit{International Olympiad in Informatics}/Olimpiade Informatika Internasional).
	\item FreePascal memenuhi standar dalam bahasa Pascal.
\end{itemize}
\end{frame}

\begin{frame}
\frametitle{Instalasi FreePascal (Windows)}
\end{frame}

\begin{frame}
\frametitle{Instalasi FreePascal (Unix)}
\end{frame}

\begin{frame}
\frametitle{Instalasi IDE Geany (Windows)}
\end{frame}

\begin{frame}
\frametitle{Instalasi IDE Geany (Unix)}
\end{frame}

\begin{frame}
\frametitle{Menulis Program Pascal Sederhana}
helloworld
\end{frame}

\begin{frame}
\frametitle{Referensi}
\end{frame}

\end{document}