\documentclass{beamer}
\usetheme{tokitex}

\usepackage{pgf}
\usepackage{graphics}
\usepackage{multirow}
\usepackage{multicol}
\usepackage{tabto}
\usepackage[english,bahasa]{babel}
\newtranslation[to=bahasa]{Section}{Bagian}
\newtranslation[to=bahasa]{Subsection}{Subbagian}

\usepackage{listings}
\usepackage{color}

\definecolor{dkgreen}{rgb}{0,0.6,0}
\definecolor{gray}{rgb}{0.5,0.5,0.5}
\definecolor{mauve}{rgb}{0.58,0,0.82}

\lstset{frame=tb,
  language=pascal,
  showstringspaces=false,
  columns=flexible,
  basicstyle={\small\ttfamily},
  numbers=none,
  numberstyle=\tiny\color{gray},
  keywordstyle=\color{blue},
  commentstyle=\color{dkgreen},
  stringstyle=\color{mauve},
  breaklines=true,
  breakatwhitespace=true,
  tabsize=2
}

\title{Subprogram: \newline Fungsi dan Prosedur}
\author{Tim Olimpiade Komputer Indonesia}

\begin{document}

\begin{frame}
\titlepage
\end{frame}

\begin{frame}
\frametitle{Pendahuluan}
Melalui dokumen ini, kalian akan:
\begin{itemize}
	\item Memahami mengapa perlu ada subprogram.
	\item Membedakan fungsi dan prosedur.
	\item Mempelajari cara menulis fungsi dan prosedur.
\end{itemize}
\end{frame}

\begin{frame}
\frametitle{Motivasi-1}
\begin{itemize}
	\item Ketika menulis program, kadang-kadang kita memerlukan suatu rutinitas yang sama di beberapa tempat.
	\item Sebagai gambaran, perhatikan contoh soal berikut:
	\begin{itemize}
		\item Pak Dengklek menancapkan tiga buah tiang pancang di halaman rumahnya untuk membangun sebuah kandang bebek.
		\item Setiap tiang pancang bisa dianggap terletak di suatu sistem koordinat Kartesius, yaitu di $(A_x, A_y)$, $(B_x, B_y)$, dan $(C_x, C_y)$.
		\item Pagar akan dibentangkan menurut garis lurus antar setiap tiang pancang.
		\item Sekarang Pak Dengklek ingin tahu, berapa luas kandang bebeknya!
	\end{itemize}
	\item Persoalan yang kita hadapi adalah menghitung luas dari segitiga, jika hanya diberikan titik-titik sudutnya.
\end{itemize}
\end{frame}

\begin{frame}
\frametitle{Motivasi-1 (lanj.)}
Salah satu penyelesaian untuk soal tersebut adalah menggunakan rumus Heron:
\newline \newline
Jika sebuah segitiga memiliki panjang sisi sebesar $a$, $b$, dan $c$, maka luas segitiga tersebut adalah:
\newline
$L = \sqrt{S(S-a)(S-b)(S-c)}$, dengan $S = \frac{a+b+c}{2}$
\vfill
Bagaimana implementasinya pada program?
\end{frame}

\begin{frame}[fragile]
\frametitle{Motivasi-1 (lanj.)}
\begin{itemize}
	\item Kita perlu menghitung jarak antar titik terlebih dahulu, baru bisa menghitung luasnya.
	\item Pascal tidak menyediakan fungsi dasar untuk menghitung jarak antar dua titik, sehingga kita harus menuliskannya:
	\begin{lstlisting}
	(* tA, tB, tC merupakan ketiga titik yang diberikan *)
	a := sqrt(sqr(tA.x - tB.x) + sqr(tA.y - tB.y));
	b := sqrt(sqr(tA.x - tC.x) + sqr(tA.y - tC.y));
	c := sqrt(sqr(tB.x - tC.x) + sqr(tB.y - tC.y));
	\end{lstlisting}
	\item Perhatikan bahwa hal yang sama, yaitu menghitung jarak titik, dituliskan secara berulang-ulang.
	\item Bagaimana jika pada salah satunya terdapat kesalahan pengetikan? Atau suatu ketika rumusnya perlu diubah? Sungguh merepotkan!
\end{itemize}
\end{frame}

\begin{frame}[fragile]
\frametitle{Motivasi-2}
\begin{itemize}
	\item Seringkali ketika kita menulis program yang panjang, program menjadi lebih sulit dipahami, meskipun telah ditulis komentar sekalipun.
	\item Alangkah baiknya jika kita bisa membuat subprogram untuk suatu rutinitas tertentu dan menyatukannya di akhir, seperti:
	\begin{lstlisting}
bacaMasukan(N);
bangkitkanPrimaSampai(N);

writeln('faktorisasi:');
temp := N;
while (not cekPrima(temp)) do begin
   d := cariPembagiTerkecil(temp);
   temp := temp div d;
   writeln(d);
end;
if (temp > 1) then
   writeln(temp);
	\end{lstlisting}
\end{itemize}
\end{frame}

\section{Konsep Subprogram}
\frame{\sectionpage}

\begin{frame}
\frametitle{Konsep Subprogram}
\begin{itemize}
	\item Sesuai dengan namanya, subprogram adalah bagian dari program.
	\item Jika program merupakan serangkaian instruksi untuk mencapai suatu tujuan tertentu, maka subprogram bisa dianggap sebagai serangkaian instruksi untuk mencapai suatu tujuan tertentu, \alert{sebagai bagian dari tujuan program}.
	\item Subprogram bisa dipanggil di bagian manapun pada program.
\end{itemize}
\end{frame}

\begin{frame}[fragile]
\frametitle{Konsep Prosedur}
\begin{itemize}
	\item Kita bisa memindahkan serangkaian kode menjadi sebuah subprogram, lalu memanggilnya pada program utama.
	\item Perhatikan contoh pesan.pas berikut!
	\begin{lstlisting}
	var
	   pesan: string;
	
	(* prosedur *)
	procedure bacaPesan();
	begin
	   write('masukkan pesan: ');
	   readln(pesan);
	end;
	
	(* program utama *)
	begin
	   bacaPesan();
	   writeln('pesan = ', pesan);
	end.
	\end{lstlisting}
\end{itemize}
\end{frame}

\begin{frame}[fragile]
\frametitle{Penjelasan}
\begin{itemize}
	\item Pada pesan.pas, terdapat sebuah prosedur bernama \textbf{bacaPesan} yang melakukan perintah untuk membaca masukan.
	\item Ketika \textbf{bacaPesan} dipanggil pada program utama, bisa dianggap seluruh instruksi yang ada di dalam prosedur tersebut dipindahkan ke program utama yang memanggilnya.
	\item Sehingga, program utama seakan-akan menjadi:
	\begin{lstlisting}
	(* program utama *)
	begin
	   write('masukkan pesan: ');
	   readln(pesan);
	   writeln('pesan = ', pesan);
	end.
	\end{lstlisting}
\end{itemize}
\end{frame}

\begin{frame}[fragile]
\frametitle{Penjelasan (lanj.)}
\begin{itemize}
	\item Tentu saja, sebuah prosedur bisa dipanggil berkali-kali, dan hal yang dilakukan tetap sama.
	\item Coba modifikasi blok progam utama pesan.pas sehingga menjadi:
	\begin{lstlisting}
	(* program utama *)
	begin
	   bacaPesan();
	   writeln('pesan = ', pesan);
	   
	   bacaPesan();
	   writeln('sekarang pesan berisi = ', pesan);
	end.
	\end{lstlisting}
\end{itemize}
\end{frame}

\section{Implementasi pada Pascal}
\frame{\sectionpage}

\begin{frame}[fragile]
\frametitle{Prosedur}
Pada Pascal, prosedur bisa ditulis dengan format berikut:
\begin{lstlisting}
procedure <nama>(<parameter...>);
var
   <deklarasi variabel...>
begin
   <instruksi...>
end;
\end{lstlisting}
\begin{itemize}
	\item Blok \textbf{var} dan deklarasi variabel boleh dihilangkan jika tidak ada variabel perlu dideklarasikan.
	\item Perhatikan bahwa \textbf{end} diakhiri dengan titik koma (;), bukan titik (.) seperti pada blok program utama.
\end{itemize}
\end{frame}

\begin{frame}[fragile]
\frametitle{Konsep Parameter}
Parameter merupakan tempat untuk "memberi masukan" bagi prosedur, sehingga prosedur bisa berperilaku berdasarkan masukan yang dia terima.
\vfill
Perhatikan contoh berikut:
\begin{lstlisting}
procedure gambar(x: longint);
var
   i: longint;
begin
   for i := 1 to x do begin
      write('*');
   end;
   writeln;
end;
\end{lstlisting}
\end{frame}

\begin{frame}[fragile]
\frametitle{Konsep Parameter (lanj.)}
\begin{itemize}
	\item Prosedur \textbf{gambar} berfungsi untuk menuliskan karakter '*' pada sebuah baris sebanyak $x$ kali.
	\item Lalu apa $x$ pada prosedur tersebut?
	\item Untuk menjawabnya, perhatikan blok program utama berikut:
	\begin{lstlisting}
	(* program utama *)
	begin
	   gambar(3);
	   gambar(5);
	end.
	\end{lstlisting}
	\item Yang akan tercetak adalah:
	\begin{lstlisting}
	***
	*****
	\end{lstlisting}
\end{itemize}
\end{frame}

\begin{frame}[fragile]
\frametitle{Konsep Parameter (lanj.)}
\begin{itemize}
	\item Pada pemanggilan pertama, angka 3 pada \textbf{gambar(3)} mengakibatkan nilai $x$ untuk prosedur \textbf{gambar} bernilai 3. Sehingga tercetak 3 karakter '*'.
	\item Pada pemanggilan kedua, nilai $x$ yang diterima adalah 5. Sehingga tercetak 5 karakter '*'.
	\item Variabel $x$ pada prosedur \textbf{gambar} disebut sebagai \alert{\textbf{parameter}}.
	\item Melalui contoh ini, kalian dapat memahami bahwa nilai parameter dapat digunakan untuk mengatur perilaku prosedur.
\end{itemize}
\end{frame}

\begin{frame}[fragile]
\frametitle{Konsep Parameter (lanj.)}
\begin{itemize}
	\item Tentu saja, pemanggilan juga bisa dilakukan dengan variabel seperti contoh berikut:
	\begin{lstlisting}
	var
	   N: longint;
	   
	... (* prosedur gambar *)   
	   
	begin
	   readln(N);
	   gambar(N);
	end.
	\end{lstlisting}
\end{itemize}
\end{frame}

\begin{frame}[fragile]
\frametitle{Parameter}
\begin{itemize}
	\item Parameter dituliskan seperti pada deklarasi variabel.
	\item Contoh:
	\begin{lstlisting}
	(* tanpa parameter *)
	procedure baca();
	
	(* satu parameter *)
	procedure tes(x: longint);
	
	(* dua parameter *)
	procedure sama(x, y: longint); 
	
	(* dua parameter, berbeda tipe data *)
	procedure berbeda(x: string; y: longint); 
	\end{lstlisting}
\end{itemize}
\end{frame}

\begin{frame}[fragile]
\frametitle{Scope Variabel}
\begin{block}{Apa itu \textbf{scope variabel}?}
Bisa dianggap sebagai "masa hidup" suatu variabel; kapan dia terdefinisi dan kapan tidak.
\end{block}
\end{frame}

\begin{frame}[fragile]
\frametitle{Scope Variabel (lanj.)}
\begin{itemize}
	\item Perhatikan kembali prosedur \textbf{gambar} berikut:
	\begin{lstlisting}
	procedure gambar(x: longint);
	var
	   i: longint;
	begin
	   for i := 1 to x do begin
	      write('*');
	   end;
	   writeln;
	end;
	\end{lstlisting}
	\item Variabel \textbf{i} dan \textbf{x} pada prosedur tersebut hanya terdefinisi di antara blok \textbf{begin} dan \textbf{end} prosedur \textbf{gambar} saja.
	\item Artinya jika pada program utama terdapat pula variabel bernama \textbf{x} atau \textbf{i}, maka variabel tersebut \alert{bukan} mengacu pada \textbf{x} dan \textbf{i} pada prosedur \textbf{gambar}.
\end{itemize}
\end{frame}

\begin{frame}[fragile]
\frametitle{Scope Variabel (lanj.)}
\begin{itemize}
	\item Variabel yang dideklarasikan pada blok \textbf{var subprogram} biasa disebut sebagai \textbf{variabel lokal}.
	\item Sementara variabel yang dideklarasikan pada blok \textbf{var program utama} disebut sebagai \textbf{variabel global}.
	\item Variabel global dapat diakses di mana saja, bahkan di dalam subprogram sekalipun.
	\item Namun, variabel lokal hanya bisa diakses pada subprogram yang mendeklarasikannya.
\end{itemize}
\end{frame}

\begin{frame}[fragile]
\frametitle{Fungsi}
\begin{itemize}
	\item Setelah kalian memahami tentang prosedur, mari kita membahas tentang fungsi.
	\item Secara konsep, fungsi mirip dengan prosedur.
	\item Bedanya adalah fungsi \alert{mengembalikan sebuah nilai}.
	\item Perhatikan fungsi berikut:
	\begin{lstlisting}
	function kubik(x: longint): longint;
	begin
	   kubik := x*x*x;
	end;
	\end{lstlisting}
\end{itemize}
\end{frame}

\begin{frame}[fragile]
\frametitle{Penjelasan}
\begin{itemize}
	\item Pada program utama, kita bisa melakukan:
	\begin{lstlisting}
	(* program utama *)
	begin.
	   volume := kubik(3);
	   selisih := volume - kubik(2);
	   writeln('4 kubik adalah ', kubik(4));
	end.
	\end{lstlisting}
	\item Teringat dengan sesuatu?
	\item Fungsi \textbf{kubik} kini terlihat seperti fungsi yang biasa kalian gunakan, seperti \textbf{sqrt}, \textbf{trunc}, atau \textbf{abs}! 
\end{itemize}
\end{frame}

\begin{frame}[fragile]
\frametitle{Nilai Kembalian}
\begin{itemize}
	\item Ciri paling utama yang membedakan fungsi dengan prosedur adalah adanya \textbf{nilai kembalian}.
	\item Ketika fungsi dipanggil, fungsi akan mengembalikan nilai, dan nilai ini bisa dioperasikan ke dalam \textbf{ekspresi} atau \textbf{assignment}.
	\item Sebagai ilustrasi, perhatikan ekspresi berikut:
	\begin{lstlisting}
	x := 2;
	y := 3*sqr(x) - 1;
	\end{lstlisting}
	\item Pada saat dijalankan, fungsi sqr(x) akan dieksekusi dan \textbf{mengembalikan nilai} 4.
	\begin{lstlisting}
	y := 3*4 - 1;
	\end{lstlisting}
	\item Setelah itu ekspresi dievaluasi, menghasilkan nilai 11 pada \textbf{y}.
\end{itemize}
\end{frame}

\begin{frame}[fragile]
\frametitle{Nilai Kembalian (lanj.)}
\begin{itemize}
	\item Hal semacam ini tidak berlaku pada prosedur, karena prosedur tidak mengembalikan nilai.
	\item Hal ini juga menjelaskan mengapa fungsi dan prosedur biasa dipanggil dengan cara:
	\begin{lstlisting}
	(* prosedur *)
	prosedurKerja();
	
	(* fungsi *)
	x := fungsiKerja();
	\end{lstlisting}
\end{itemize}
\end{frame}

\begin{frame}[fragile]
\frametitle{Fungsi}
\begin{itemize}
	\item Pada Pascal, fungsi dituliskan dengan format:
	\begin{lstlisting}
	function <nama>(<parameter...): <tipe kembalian>
	var
	   <deklarasi variabel...>
	begin
	   <instruksi...>
	   ...
	   <nama> := <nilai kembalian>;
	end;
	\end{lstlisting}
	\item Dengan $<$tipe kembalian$>$ adalah tipe data untuk nilai yang dikembalikan.
	\item Untuk memberi tahu Pascal nilai yang perlu dikembalikan, lakukan \textbf{assingment} pada nama fungsi dengan nilai yang akan dikembalikan, seperti yang dicontohkan baris akhir isi subprogram pada format di atas.
\end{itemize}
\end{frame}

\begin{frame}
\frametitle{Fungsi (lanj.)}
\begin{itemize}
	\item Melakukan \textbf{assignment} pada nama fungsi tidak harus dilakukan pada bagian akhir.
	\item Yang pasti, Pascal akan mengembalikan nilai yang terakhir di-\textbf{assign} ke dalam nama fungsi tersebut.
	\item Seperti pada prosedur, bagian deklarasi variabel juga bisa dihilangkan jika tidak ada variabel lokal yang diperlukan.
	\item Cara penulisan parameter juga sama dengan prosedur.
\end{itemize}
\end{frame}

\section{Passing Parameter}
\frame{\sectionpage}

\begin{frame}
\frametitle{Passing Parameter}
\begin{itemize}
	\item \textit{Passing parameter} merupakan aktivitas menyalurkan nilai pada parameter saat memanggil subprogram.
	\item Umumnya, dikenal dua macam \textit{passing parameter}:
	\begin{itemize}
		\item \textit{By value}, yaitu mengirimkan \alert{nilai} dari setiap parameter yang diberikan.
		\item \textit{By reference}, yaitu mengirimkan \alert{alamat} dari setiap parameter yang diberikan.
	\end{itemize}
\end{itemize}
\end{frame}

\begin{frame}[fragile]
\frametitle{Passing Parameter By Value}
\begin{itemize}
	\item Sebagai penjelasan, perhatikan program berikut:
	\begin{lstlisting}
	var
	   x, y: longint;
	
	procedure tukar(a, b: longint);
	var
	   temp: longint;
	begin
	   temp := a;
	   a := b;
	   b := a;
	end;
	
	begin
	   x := 1; y := 2;
	   tukar(x, y);
	   writeln('x=', x, ' y=', y);
	end.
	\end{lstlisting}
\end{itemize}
\end{frame}

\begin{frame}[fragile]
\frametitle{Passing Parameter By Value (lanj.)}
\begin{itemize}
	\item Jika dijalankan, apa keluaran dari program tersebut? Apakah "x=2 y=1"?
	\item Jawabannya \alert{tidak}, yang tercetak adalah "x=1 y=2".
	\item Hal ini disebabkan karena ketika prosedur \textbf{tukar} dipanggil, \alert{nilai} dari $x$ dan $y$ dikirim ke parameter $a$ dan $b$ pada prosedur \textbf{tukar}.
	\item Jadi $a$ dan $b$ hanya melakukan \textbf{assignment} nilai dari $x$ dan $y$. Apapun yang terjadi pada $a$ dan $b$ selanjutnya tidak mempengaruhi $x$ dan $y$ karena mereka \alert{tidak berhubungan}.
\end{itemize}
\end{frame}

\begin{frame}[fragile]
\frametitle{Passing Parameter By Reference}
\begin{itemize}
	\item Lain halnya ketika kita menambahkan kata kunci \textbf{var} pada penulisan parameter:
	\begin{lstlisting}
	procedure tukar(var a, b: longint);
	var
	   temp: longint;
	begin
	   temp := a;
	   a := b;
	   b := a;
	end;
	\end{lstlisting}
\end{itemize}
\end{frame}

\begin{frame}[fragile]
\frametitle{Passing Parameter By Reference (lanj.)}
\begin{itemize}
	\item Dengan cara ini, ketika \textbf{tukar}($x$, $y$) dipanggil, \alert{alamat memori variabel} $x$ dan $y$ dikirimkan ke parameter $a$ dan $b$.
	\item Kini, $x$ dan $a$ mengacu pada alamat memori yang sama. 
	\item Apabila dilakukan perintah "$a := 3$", maka nilai $x$ juga ikut menjadi 3. Hal ini disebabkan karena $x$ dan $a$ mengacu pada \alert{alamat memori yang sama}.
	\item Demikian pula untuk $y$ dan $b$.
	\item Sehingga keluaran program menjadi "x=2 y=1"!
\end{itemize}
\end{frame}

\begin{frame}[fragile]
\frametitle{Passing Parameter By Reference (lanj.)}
\begin{itemize}
	\item Dengan \textit{passing parameter by reference}, kita tidak bisa melakukan:
	\begin{lstlisting}
	tukar(2, 3); 
	\end{lstlisting}
	
	\item Mengirimkan alamat memori dari variabel memang bisa dilakukan, tetapi angka 2 atau 3 jelas bukan variabel dan jelas \alert{tidak punya alamat memori}.
	\item Jika kita mengembalikan kedua parameter prosedur \textbf{tukar} untuk menggunakan \textit{passing parameter by value}, barulah hal ini bisa dilakukan. Karena yang dikirimkan adalah \alert{nilainya}.
\end{itemize}
\end{frame}

\begin{frame}[fragile]
\frametitle{Penulisan Pada Pascal}
\begin{itemize}
	\item Untuk melakukan \textit{passing parameter by reference}, cukup tambahkan kata kunci \textbf{var} di depan parameter yang akan di-\textit{pass by reference}.
	\item Sebuah subprogram bisa juga menerima parameter dengan cara \textit{passing parameter} yang campuran:
	\begin{lstlisting}
	procedure bagi(a,b:longint; var hasil,sisa:longint);
	begin
	   hasil := a div b;
	   sisa := a mod b;
	end;	
	\end{lstlisting}
\end{itemize}
\end{frame}

\section{Studi Kasus Subprogram}
\frame{\sectionpage}

\begin{frame}[fragile]
\frametitle{Fungsi Pangkat}
Berikut ini adalah fungsi untuk menghitung $a^b$:
\begin{lstlisting}
function pangkat(a, b: longint);
var
   i: longint;
   hasil: longint;
begin
   hasil := 1;
   for i := 1 to b do begin
      hasil := hasil * a;
   end;
   
   pangkat := hasil;
end;	
\end{lstlisting}
\end{frame}

\begin{frame}[fragile]
\frametitle{Prosedur Pangkat}
Berikut ini adalah prosedur untuk menghitung $a^b$, hasil ditampung pada variabel $hasil$:
\begin{lstlisting}
	procedure pangkat(a, b: longint; var hasil: longint);
	var
	   i: longint;
	begin
	   hasil := 1;
	   for i := 1 to b do begin
	      hasil := hasil * a;
	   end;
	end;		
\end{lstlisting}
\end{frame}

\begin{frame}[fragile]
\frametitle{Fungsi dan Prosedur}
\begin{itemize}
	\item Baik dengan fungsi atau prosedur, kita bisa mencapai hal yang sama.
	\item Pertanyaannya adalah: \alert{mana yang lebih tepat}?
\end{itemize}
\end{frame}

\begin{frame}[fragile]
\frametitle{Fungsi dan Prosedur (lanj.)}
\begin{itemize}
	\item Dengan fungsi, menghitung $y = 3x^5$ bisa dilakukan dengan:
	\begin{lstlisting}
	y := 3 * pangkat(x, 5);
	\end{lstlisting}
	
	\item Dengan prosedur, sedikit lebih rumit:
	\begin{lstlisting}
	pangkat(x, 5, y);
	y := 3 * y;
	\end{lstlisting}
	
	\item Untuk kasus ini, \alert{penggunaan fungsi lebih tepat}.
\end{itemize}
\end{frame}

\begin{frame}
\frametitle{Kilas Balik: Prosedur tukar}
\begin{itemize}
	\item Sekarang coba ingat kembali prosedur \textbf{tukar} yang kita bahas sebelumnya.
	\item Kurang masuk akal apabila kita menggunakan fungsi untuk melakukan penukaran.
	\item Sehingga untuk kasus penukaran isi variabel, \alert{lebih tepat digunakan prosedur}.
\end{itemize}
\end{frame}

\begin{frame}
\frametitle{Penggunaan Fungsi dan Prosedur}
\begin{itemize}
	\item Dari sini kita mempelajari bahwa ada subprogram yang lebih cocok diimplementasikan menjadi fungsi, dan ada juga yang lebih cocok menjadi prosedur.
	\item Biasanya, fungsi bersifat:
	\begin{itemize}
		\item Menghasilkan suatu nilai berdasarkan parameter.
		\item Tidak mengakibatkan efek samping, misalnya adanya perubahan nilai pada parameter yang diberikan seperti pada prosedur \textbf{tukar}.
	\end{itemize}
	\item Sementara prosedur bersifat:
	\begin{itemize}
		\item Tidak menghasilkan suatu nilai.
		\item Boleh jadi mengakibatkan perubahan pada variabel global atau parameter yang dikirimkan.
	\end{itemize}
\end{itemize}
\end{frame}

\begin{frame}
\frametitle{Manfaat Subprogram}
\begin{itemize}
	\item Meningkatkan daya daur ulang kode (\textit{reusability}).
	\newline Satu kali saja kita mendefinisikan prosedur untuk menukar isi variabel, berapa kali pun penukaran isi variabel bisa dilakukan tanpa perlu menuliskan algoritma penukaran (cukup memanggil prosedur).
	\item Memecah program menjadi beberapa subprogram yang lebih kecil. 
	\newline Keuntungannya adalah didapatkan kumpulan subprogram yang:
	\begin{itemize}
		\item Fokus pada suatu tujuan tertentu.
		\item Tersusun atas kode yang cenderung pendek.
		\item Karena kedua hal di atas, lebih mudah dibaca dan ditelusuri.
	\end{itemize}
\end{itemize}
\end{frame}

\begin{frame}
\frametitle{Selanjutnya...}
\begin{itemize}
	\item Mengulas lebih dalam tentang tipe data \textbf{string} pada Pascal.
	\item Mengeksplorasi fungsi dan prosedur dasar pada pengolahan \textbf{string}.
\end{itemize}
\end{frame}

\end{document}