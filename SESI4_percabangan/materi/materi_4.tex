\documentclass{beamer}
\usepackage{graphics}
\usepackage{multirow}
\usepackage{tabto}

\usepackage{listings}
\usepackage{color}

\definecolor{dkgreen}{rgb}{0,0.6,0}
\definecolor{gray}{rgb}{0.5,0.5,0.5}
\definecolor{mauve}{rgb}{0.58,0,0.82}

\lstset{frame=tb,
  language=pascal,
  aboveskip=3mm,
  belowskip=3mm,
  showstringspaces=false,
  columns=flexible,
  basicstyle={\small\ttfamily},
  numbers=none,
  numberstyle=\tiny\color{gray},
  keywordstyle=\color{blue},
  commentstyle=\color{dkgreen},
  stringstyle=\color{mauve},
  breaklines=true,
  breakatwhitespace=true,
  tabsize=3
}

\title{Percabangan}
\author{Tim Olimpiade Komputer Indonesia}

\begin{document}

\begin{frame}
\titlepage
\end{frame}

\begin{frame}
\frametitle{Pendahuluan}
Melalui dokumen ini, kalian akan:
\begin{itemize}
	\item Mengenal percabangan.
	\item Memahami analisa kasus dan mengimplementasikannya pada Pascal.
\end{itemize}
\end{frame}

\begin{frame}
\frametitle{Motivasi}
\begin{itemize}
	\item Diberikan sebuah bilangan. Tentukan apakah bilangan itu positif atau bukan positif!
	\item Jika positif, cetak "positif". Jika tidak, jangan cetak apa-apa.
\end{itemize}
\end{frame}

\begin{frame}
\frametitle{Motivasi (lanj.)}
\begin{itemize}
	\item Sebuah bilangan dinyatakan positif apabila bilangan tersebut lebih dari nol. 
	\item Dengan begitu, kita memerlukan suatu struktur yang memungkinkan "\alert{jika} bilangan itu lebih dari 0, \alert{maka} cetak positif".
	\item Pada Pascal, hal ini bisa diwujudkan dengan struktur kondisional \alert{if}.
\end{itemize}
\end{frame}

\begin{frame}[fragile]
\frametitle{Struktur "if ... then ..."}
\begin{itemize}
	\item Struktur dari penulisan "if ... then ..." adalah:
	\begin{lstlisting}
	if (<kondisi>) then begin
	    <perintah 1>;
	    <perintah 2>;
	    ...
	end;
	\end{lstlisting}
	
	\item Dengan $<$kondisi$>$ adalah suatu boolean.
	\item Jika nilai $<$kondisi$>$ adalah TRUE, seluruh perintah yang ada di antara blok "begin" dan "end" akan dilaksanakan.
	\item Jika FALSE, seluruh perintah yang ada di antara blok "begin" dan "end" akan dilewati.
\end{itemize}
\end{frame}

\begin{frame}[fragile]
\frametitle{Contoh Program: kondisi.pas}
\begin{itemize}
	\item Ketikkan dan jalankan program berikut:
	\begin{lstlisting}
		var
		    x:longint;
		begin
		    readln(x);
		
		    if (x > 0) then begin
		        writeln('positif');
		    end;
		end.
	\end{lstlisting}
	\item Perhatikan bahwa ekspresi "x $>$ 0" akan merupakan operasi relasional yang menghasilkan nilai boolean. Sehingga tepat untuk digunakan pada if.
	\item Bagaimana jika ingin dibuat jika bilangan itu bukan positif, cetak "non-positif"?
\end{itemize}
\end{frame}

\begin{frame}[fragile]
\frametitle{Struktur "if ... then ... else ..."}
\begin{itemize}
	\item Kita juga bisa membuat percabangan jika nilai pada $<$kondisi$>$ adalah FALSE, yaitu dengan kata kunci "else".
	\item Struktur dari penulisan "if ... then ... else ..." adalah:
	\begin{lstlisting}
	if (<kondisi>) then begin
	    <perintah 1>;
	    <perintah 2>;
	    ...
	end else begin
	    <perintah a>;
	    <perintah b>;
	    ...	
	end;
	\end{lstlisting}
	
	\item Jika nilai $<$kondisi$>$ adalah TRUE, $<$perintah 1$>$, $<$perintah 2$>$, ..., akan dilaksanakan.
	\item Jika FALSE, $<$perintah a$>$, $<$perintah b$>$, ..., akan dilaksanakan.
\end{itemize}
\end{frame}

\begin{frame}[fragile]
\frametitle{Contoh Program: kondisi2.pas}
\begin{itemize}
	\item Dengan "if ... then ... else ...", kita bisa memodifikasi "kondisi.pas" menjadi "kondisi2.pas":
	\begin{lstlisting}
		var
		    x:longint;
		begin
		    readln(x);
		
		    if (x > 0) then begin
		        writeln('positif');
		    end else begin
		        writeln('non-positif');
		    end;
		end.
	\end{lstlisting}
\end{itemize}
\end{frame}

\begin{frame}
\frametitle{Persoalan Sebenarnya}
Diberikan sebuah bilangan, sebut saja x.
\begin{itemize}
	\item Jika x positif, cetak "positif".
	\item Jika x sama dengan nol, cetak "nol".
	\item Jika x negatif, cetak "negatif".
\end{itemize}

Pada kasus ini, diperlukan struktur if yang lebih dari dua cabang!
\end{frame}

\begin{frame}[fragile]
\frametitle{Struktur "if ... then ... else if ..."}
\begin{itemize}
	\item Pascal menyediakan struktur yang memungkinkan kita memilah-milah untuk cabang yang lebih dari dua, yaitu dengan struktur "if ... then ... else if ...".
	\item Struktur dari penulisan "if ... then ... else if ..." adalah:
	\begin{lstlisting}
	if (<kondisi 1>) then begin
	    <perintah 1>;
	    <perintah 2>;
	    ...
	end else if (<kondisi 2>) then begin
	    <perintah a>;
	    <perintah b>;
	    ...	
	end else if (<kondisi 3>) then begin
	    ...
	end
	\end{lstlisting}

\end{itemize}
\end{frame}

\begin{frame}
\frametitle{Struktur "if ... then ... else if ..." (lanj.)}
\begin{itemize}
	\item Jika nilai $<$kondisi 1$>$ TRUE, $<$perintah 1$>$, $<$perintah 2$>$, ..., akan dilaksanakan.
	\item Jika nilai $<$kondisi 1$>$ FALSE, diperiksa apakah $<$kondisi 2$>$ bernilai TRUE. Jika ya, $<$perintah a$>$, $<$perintah b$>$, ..., akan dilaksanakan.
	\item Jika nilai $<$kondisi 2$>$ FALSE, diperiksa apakah $<$kondisi 3$>$ bernilai TRUE. Hal ini akan terus diulang sampai seluruh percabangan habis.
	\item Kalian juga bisa mengakhiri struktur ini dengan "else ...", yaitu ketika seluruh kondisi yang diberikan tidak terpenuhi, maka perintah-perintah di bawah else ini yang akan dilaksanakan.
\end{itemize}
\end{frame}

\begin{frame}[fragile]
\frametitle{Contoh Program: kondisi3.pas}
\begin{itemize}
	\item Dengan "if ... then ... else if ...", kita bisa memodifikasi "kondisi2.pas" menjadi "kondisi3.pas":
	\begin{lstlisting}
		var
		    x:longint;
		begin
		    readln(x);
		
		    if (x > 0) then begin
		        writeln('positif');
		    end else if (x = 0) then begin
		        writeln('nol');
		    end else if (x < 0) then begin
		        writeln('negatif');
		    end;
		end.
	\end{lstlisting}
\end{itemize}
\end{frame}

\begin{frame}[fragile]
\frametitle{Contoh Program: kondisi4.pas}
\begin{itemize}
	\item Pada "kondisi3.pas", sebenarnya "else if ..." yang terakhir tidak perlu. Ketika suatu bilangan bukan positif dan bukan nol, sudah pasti bilangan itu negatif. Sehingga bisa didapatkan "kondisi4.pas":
	\begin{lstlisting}
		var
		    x:longint;
		begin
		    readln(x);
		
		    if (x > 0) then begin
		        writeln('positif');
		    end else if (x = 0) then begin
		        writeln('nol');
		    end else begin
		        // sudah pasti negatif
		        writeln('negatif');
		    end;
		end.
	\end{lstlisting}
\end{itemize}
\end{frame}

\begin{frame}[fragile]
\frametitle{Kombinasi dengan Ekspresi Boolean}
\begin{itemize}
	\item Kalian juga bisa menggabungkan struktur if dengan ekspresi boolean:
	\begin{lstlisting}
	  if ((x > 0) and (x mod 2 = 1)) then begin
	      writeln('positif dan ganjil');
	      
	  end else if ((x > 0) and (x mod 2 = 0)) then begin
	      writeln('positif dan genap');
	      
	  end else if ((x < 0) and (x mod 2 = 1)) then begin
	      writeln('negatif dan ganjil');
	      
	  end else if ((x < 0) and (x mod 2 = 0)) then begin
 	      ...
	\end{lstlisting}
\end{itemize}
\end{frame}

\begin{frame}[fragile]
\frametitle{If Bersarang}
\begin{itemize}
	\item Solusi yang lebih rapi dicapai dengan menggunakan if secara bersarang:
	\begin{lstlisting}
	  if (x > 0) then begin
	      if (x mod 2 = 1) then begin
	          writeln('positif dan ganjil');
	      end else begin
	          writeln('positif dan genap');
	      end;
	  end else if (x < 0) then begin
	      ...
	\end{lstlisting}
\end{itemize}
\end{frame}

\begin{frame}
\frametitle{Contoh Lainnya}
\begin{itemize}
	\item Misalkan diberikan sebuah karakter yang merupakan salah satu dari 'A', 'B', 'C', 'D', atau 'E'.
	\item Cetak frase berikut sesuai karakter yang diberikan:
	\begin{itemize}
		\item 'A': Sempurna
		\item 'B': Bagus
		\item 'C': Cukup
		\item 'D': Kurang
		\item 'E': Sangat kurang
	\end{itemize}
\end{itemize}
\end{frame}

\begin{frame}[fragile]
\frametitle{Contoh Lainnya (lanj.)}
\begin{itemize}
	\item Tentu saja kita bisa menggunakan "if ... then ... else if ..." untuk kasus ini.
	\item Namun, menuliskan "if ... then ... else if ..." untuk kasus sederhana ini bisa cukup panjang.
	\item Bayangkan, untuk setiap kasus, kalian perlu menulis:
	\begin{lstlisting}
		if (nilai = 'A') then begin
		    ...
		end else begin if (nilai = 'B') then begin
		    ...
	\end{lstlisting}	
	\item Untuk kasus sejenis ini, tersedia struktur kondisional lain yang ditawarkan Pascal dan lebih cocok, yaitu \alert{case of}.
\end{itemize}
\end{frame}

\begin{frame}[fragile]
\frametitle{Struktur "case ... of ..."}
\begin{itemize}
	\item Struktur dari penulisan "case ... of ..." adalah:
	\begin{lstlisting}
	case (<ekspresi>) of
	    <nilai 1>: begin <perintah 1>; end;
	    <nilai 2>: begin <perintah 2>; end;
	    <nilai 3>: begin <perintah 3>; end;
	    ...
	end;
	\end{lstlisting}
	
	\item Dengan $<$ekspresi$>$ adalah suatu ekspresi yang menghasilkan \alert{nilai ordinal}.
	\item Ketika nilai yang dihasilkan $<$ekspresi$>$ sama dengan $<$nilai 1$>$, maka seluruh $<$perintah 1$>$ akan dilaksanakan.
	\item Ketika nilai yang dihasilkan $<$ekspresi$>$ sama dengan $<$nilai 2$>$, maka seluruh $<$perintah 2$>$ akan dilaksanakan.
	\item ... dan seterusnya.
\end{itemize}
\end{frame}

\begin{frame}[fragile]
\frametitle{Contoh Program: "case.pas"}
\begin{itemize}
	\item Berikut ini adalah solusi dengan menggunakan "case ... of ...":
	\begin{lstlisting}
		var
		    nilai:char;
		begin
		    readln(nilai);
		
		    case (nilai) of
		        'A': begin writeln('Sempurna'); end;
		        'B': begin writeln('Bagus'); end;
		        'C': begin writeln('Cukup'); end;
		        'D': begin writeln('Kurang'); end;
		        'E': begin writeln('Sangat kurang'); end;
		    end;
		end.
	\end{lstlisting}
\end{itemize}
\end{frame}

\begin{frame}
\frametitle{Selanjutnya...}
\begin{itemize}
	\item Ke bagian yang lebih menarik lagi, yaitu perulangan!
	\item Namun, pastikan kalian menguasai materi perulangan terlebih dahulu $:)$
\end{itemize}
\end{frame}

\end{document}