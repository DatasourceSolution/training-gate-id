\documentclass{beamer}
\usetheme{tokitex}

\usepackage{graphics}
\usepackage{multirow}
\usepackage{tabto}

\usepackage[english,bahasa]{babel}
\newtranslation[to=bahasa]{Section}{Bagian}
\newtranslation[to=bahasa]{Subsection}{Subbagian}

\usepackage{listings}
\usepackage{color}

\definecolor{dkgreen}{rgb}{0,0.6,0}
\definecolor{gray}{rgb}{0.5,0.5,0.5}
\definecolor{mauve}{rgb}{0.58,0,0.82}

\lstset{frame=tb,
  language=pascal,
  aboveskip=3mm,
  belowskip=3mm,
  showstringspaces=false,
  columns=flexible,
  basicstyle={\small\ttfamily},
  numbers=none,
  numberstyle=\tiny\color{gray},
  keywordstyle=\color{blue},
  commentstyle=\color{dkgreen},
  stringstyle=\color{mauve},
  breaklines=true,
  breakatwhitespace=true,
  tabsize=3
}

\title{Variabel dan Tipe Data}
\author{Tim Olimpiade Komputer Indonesia}

\begin{document}

\begin{frame}
\titlepage
\end{frame}

\begin{frame}
\frametitle{Pendahuluan}
Melalui dokumen ini, kalian akan:
\begin{itemize}
	\item Mengenal konsep variabel.
	\item Mempelajari berbagai tipe data.
	\item Mempelajari cara deklarasi variabel.
	\item Mengenal operasi assignment.
\end{itemize}
\end{frame}

\begin{frame}[fragile]
\frametitle{Kilas Balik}
\begin{itemize}
	\item Mari kita lihat kembali program halo.pas.
	\begin{lstlisting}
	begin
	    writeln('Halo Dunia!');
	end.
	\end{lstlisting}
	\item Pada program tersebut, terdapat kata kunci \textbf{begin} dan \textbf{end} yang diakhiri dengan titik. Kedua kata kunci tersebut membentuk sesuatu yang disebut dengan blok program utama.
	\item Ketika halo.pas dieksekusi, seluruh perintah yang dituliskan dalam blok program utama akan dieksekusi secara sekuensial (berurutan).
\end{itemize}
\end{frame}

\begin{frame}
\frametitle{Baris Perintah Program}
\begin{itemize}
	\item Pada halo.pas, satu-satunya perintah yang ada adalah \textbf{writeln('Halo Dunia!')};
	\item Pada Pascal, \textbf{writeln(x)} merupakan fungsi untuk mencetak \textbf{x} ke layar. Untuk \textbf{halo.pas}, \textbf{x} = 'Halo Dunia!'.
\end{itemize}
\end{frame}

\section{Konsep Variabel}
\frame{\sectionpage}

\begin{frame}[b]
\frametitle{Pengenalan Variabel}
\begin{block}{Variabel}
	 Merupakan istilah yang diadopsi dari dunia matematika, yang memetakan sebuah nama ke suatu nilai.
\end{block}
\begin{itemize}
	\item Setiap kali suatu variabel digunakan dalam ekspresi matematika, yang diacu sebenarnya adalah nilai yang dipetakan oleh nama variabel tersebut.
	\item Contoh: jika kita menyatakan $x=5$, maka hasil dari $3x^2 + x$ adalah $80$.
	\item Dalam pemrograman, kita bisa membuat variabel, mengisikan nilai pada variabel, dan mengacu nilai yang dipetakan variabel tersebut.
\end{itemize}
\end{frame}

\begin{frame}
\frametitle{Aturan Penamaan Variabel}
\begin{itemize}
	\item Variabel bebas diberi nama apapun, tetapi terbatas pada beberapa aturan berikut:
	\begin{itemize}
		\item Terdiri dari kombinasi karakter huruf, angka, dan underscore (\_).
		\item Tidak boleh dimulai dengan angka. Artinya "2kar" bukan merupakan nama variabel yang sah.
		\item Huruf kapital dan huruf kecil dianggap sama. Artinya "a1" dan "A1" dianggap merupakan dua variabel yang sama.
		\item Tidak boleh merupakan \alert{\textit{reserved word}}. Contoh \textit{reserved word} pada Pascal: \textbf{begin}, \textbf{end}, \textbf{var}, \textbf{if}, atau \textbf{then}.
	\end{itemize}
	\item Lebih jauh lagi, aturan ini berlaku pada seluruh penamaan \alert{\textit{identifier}}, yang temasuk di dalamnya nama variabel, fungsi, dan prosedur (akan dipelajari selanjutnya).
\end{itemize}
\end{frame}

\begin{frame}
\frametitle{Assignment}
\begin{block}{Assignment}
	Pengisian nilai yang diacu oleh variabel dengan suatu nilai disebut \alert{\textit{assignment}.}
\end{block}
\begin{itemize}
	\item Operator untuk \textit{assignment} adalah :=
	\item Isikan ruas kiri dengan nama suatu variabel, dan ruas kanan dengan nilai yang ingin diisikan ke variabel tersebut.
	\item Tipe data dari variabel dan nilai yang diacu \alert{harus sesuai}.
\end{itemize}
\end{frame}


\begin{frame}[fragile]
\frametitle{Contoh Program: assign.pas}
\begin{itemize}
	\item Perhatikan contoh program assign.pas berikut. Tuliskan, lalu jalankan program ini.
	\begin{lstlisting}
	var
	    x : integer;
	begin
	    x := 12;
	    writeln('Nilai = ', x);
	end.
	\end{lstlisting}
\end{itemize}
\end{frame}

\begin{frame}[fragile]
\frametitle{Penjelasan Program: assign.pas}
\begin{itemize}
	\item Keluaran yang dihasilkan dari program itu adalah sebuah baris berisikan:
	\begin{lstlisting}
		Nilai = 12
	\end{lstlisting}
	\item Pada program tersebut, \textbf{x} merupakan suatu variabel.
	\item Variabel \textbf{x} didaftarkan terlebih dahulu dengan menuliskan \textbf{x : integer} pada bagian antara \textbf{var} dengan \textbf{begin}. Semua variabel yang akan digunakan pada program utama wajib didaftarkan pada bagian ini.
	\item Pada blok program utama, \textbf{x} diisi dengan nilai 12, lalu perintah \textbf{writeln} dieksekusi.
\end{itemize}
\end{frame}

\begin{frame}[fragile]
\frametitle{Contoh Program: assign2.pas}
\begin{itemize}
	\item Berikut adalah contoh program yang melibatkan beberapa variabel.
	\begin{lstlisting}
	var
	    x : integer;
	    y : longint;
	begin
	    x := 12;
	    y := 123456;
	    writeln('Nilai x = ', x);
	    writeln('Nilai y = ', y);
	    
	    x := 15;
	    writeln('Sekarang nilai x = ', x);
	end.
	\end{lstlisting}
\end{itemize}
\end{frame}

\begin{frame}[fragile]
\frametitle{Penjelasan Program: assign2.pas}
\begin{itemize}
	\item Keluaran yang dihasilkan dari program itu adalah:
	\begin{lstlisting}
	12
	123456
	15
	\end{lstlisting}
	\item Apa maksud dari kata kunci \textbf{integer} dan \textbf{longint}? Dijelaskan pada bagian selanjutnya.
\end{itemize}
\end{frame}

\section{Tipe Data Variabel}
\frame{\sectionpage}

\begin{frame}
\frametitle{Tipe Data Variabel}
\begin{itemize}
	\item Setiap variabel pada Pascal wajib memiliki sebuah \alert{tipe data}.
	\item Jenis tipe data dasar dari suatu variabel pada:
	\begin{itemize}
		\item Bilangan bulat: hanya bisa berisikan bilangan bulat.
		\item Bilangan riil: bisa berisikan bilangan riil (termasuk pecahan).
		\item Karakter: merepresentasikan karakter, seperti 'a', 'b', '3', atau '?'.
		\item String: merepresentasikan untaian karakter, seperti 'aku' atau 'halo dunia'.
		\item Nilai kebenaran: merepresentaskan benar (\textbf{TRUE}) atau salah (\textbf{FALSE}).
	\end{itemize}
\end{itemize}
\end{frame}

\begin{frame}
\frametitle{Tipe Data: Bilangan Bulat}
\begin{table}[ht]
	\begin{tabular}{|c|c|c|}
		\hline Nama  & Jangkauan  & Ukuran \\ 
		\hline shortint & $-2^7 .. 2^7-1$ & 1 byte \\ 
		\hline byte & $0 .. 2^8$ & 1 byte\\ 
		\hline integer & $-2^{15} .. 2^{15}-1$ & 2 byte\\ 
		\hline word & $0 .. 2^{16}$ & 2 byte\\ 
		\hline longint & $-2^{31} .. 2^{31}-1$ & 4 byte\\ 
		\hline longword & $0 .. 2^{32}$ & 4 byte\\ 
		\hline int64 & $-2^{63} .. 2^{63}-1$ & 8 byte\\ 
		\hline qword & $0 .. 2^{64}$ & 8 byte\\ 
		\hline 
	\end{tabular}
\end{table} 
\begin{itemize}
	\item Biasa disebut dengan \textit{integer} saja.
	\item Pascal menawarkan beberapa tipe data bilangan bulat yang variasinya terletak pada jangkauan nilai yang bisa direpresentasikan dan ukurannya pada memori.
	\item Dalam memprogram, yang umum digunakan adalah \alert{\textbf{longint}} dan \alert{\textbf{int64}}. 
\end{itemize}
\end{frame}

\begin{frame}
\frametitle{Tipe Data: Bilangan Riil}
\begin{table}[ht]
	\begin{tabular}{|c|c|c|c|}
		\hline Nama  & Jangkauan (magnitudo) & Akurasi & Ukuran \\ 
		\hline single & $1.5\times10^{-45} .. 3.4\times10^{38}$ & 7-8 digit & 4 byte\\ 
		\hline double & $5.0\times10^{-324} .. 1.7\times10^{308}$ & 15-16 digit & 8 byte \\ 
		\hline 
	\end{tabular}
\end{table} 
\begin{itemize}
	\item Biasa disebut dengan \textit{floating point}.
	\item Tipe data \textit{floating point} bisa merepresentasikan negatif atau positif dari magnitudonya.
	\item Pada pemrograman, umumnya tipe data \textit{floating point} dihindari karena kurang akurat. Representasi 3 pada \textit{floating point} bisa jadi 2.99999999999999 atau 3.000000000000001 karena keterbatasan pada struktur penyimpanan bilangan \textit{floating point} komputer.
	\item Tipe yang umum digunakan adalah \alert{\textbf{double}}.
\end{itemize}
\end{frame}

\begin{frame}
\frametitle{Tipe Data: Karakter}
\begin{itemize}
	\item Merupakan tipe data untuk merepresentasikan karakter menurut ASCII (\textit{American Standart Code for Information Interchange}).
	\item Dalam ASCII, terdapat 128 karakter yang direpresentasikan dengan angka dari 0 sampai 127. 
	\item Misalnya, kode ASCII untuk karakter spasi (' ') adalah 32, huruf 'A' adalah 65, 'B' adalah 66, huruf 'a' adalah 97, dan huruf 'b' adalah 98.
\end{itemize}
\end{frame}

\begin{frame}
\frametitle{Tipe Data: Karakter (lanj.)}
\begin{itemize}
	\item Pada Pascal, tipe data ini dinyatakan sebagai \alert{\textbf{char}}, dengan ukuran 1 byte.
	\item Dalam memprogram, kita bisa menggunakan fungsi \textbf{ord(x)} untuk mendapatkan nilai ASCII dari \textbf{char x}, dan \textbf{chr(x)} untuk mendapatkan karakter dari kode ASCII \textbf{x}.
\end{itemize}
\end{frame}

\begin{frame}[fragile]
\frametitle{Contoh Program: karakter.pas}
\begin{itemize}
	\item Tulis dan coba eksekusi program berikut.
	\begin{lstlisting}
	begin
	    writeln(chr(64));
	    writeln(chr(65));
	    writeln(chr(49));
	    writeln(ord('A'));
	    writeln(ord('a'));
	    writeln(ord('1'));
	    writeln(chr(ord('a')));
	end.
	\end{lstlisting}
\end{itemize}
\end{frame}

\begin{frame}[fragile]
\frametitle{Penjelasan Program: karakter.pas}
\begin{itemize}
	\item Berikut adalah keluaran dari contoh program tersebut.
	\begin{lstlisting}
	A
	B
	1
	64
	97
	49
	a
	\end{lstlisting}
	\item Dari contoh tersebut, kalian dapat memahami penggunaan \textbf{chr} dan \textbf{ord}.
	\item Perhatikan pula penulisan suatu karakter perlu menggunakan tanda petik tunggal (').
	\item Bedakan antara 1 dengan '1'. Menuliskan \textbf{ord(1)} bisa mengakibatkan \textit{error} dan program tidak bisa dikompilasi.
\end{itemize}
\end{frame}

\begin{frame}
\frametitle{Tipe Data: String}
\begin{itemize}
	\item Tipe data yang merepresentasikan untaian dari \textbf{char}, contohnya 'kucing', 'dan', '' (string kosong). Tipe data ini dinyatakan sebagai \alert{\textbf{string}}.
	\item Pada Pascal, \textbf{string} bisa menampung antara 0 sampai 255 karakter. Untuk menampung lebih dari 255 karakter, gunakan tipe data \alert{\textbf{ansistring}}.
\end{itemize}
\end{frame}

\begin{frame}
\frametitle{Tipe Data: Boolean}
\begin{itemize}
	\item Merupakan tipe data yang menyimpan nilai kebenaran, yaitu hanya \textbf{TRUE} atau \textbf{FALSE}.
	\item Tipe data ini akan lebih terasa kebermanfaatannya ketika kita sudah mempelajari struktur percabangan dan \textbf{array}.
	\item Pada Pascal, kalian dapat menggunakan tipe data \alert{\textbf{boolean}}.
\end{itemize}
\end{frame}

\begin{frame}
\frametitle{Deklarasi Variabel}
\begin{itemize}
	\item Mendaftarkan nama-nama dan tipe variabel yang akan digunakan disebut dengan deklarasi variabel.
	\item Pada saat dideklarasi, setiap variabel perlu disertakan tipe datanya. 
	\item Pada Pascal, variabel dideklarasikan di antara \textbf{var} dengan \textbf{begin}.
	\item Tipe data dituliskan sesudah tanda titik dua (:), setelah nama variabel dituliskan.
	
	Contoh: \textbf{nilai : longint} atau \textbf{rerata : double}.
	\item Beberapa variabel juga bisa dideklarasikan secara bersamaan jika memiliki tipe data yang sama. Contoh: \textbf{x, y : double}.
\end{itemize}
\end{frame}

\begin{frame}
\frametitle{Tipe Data Komposit: Record}
\begin{itemize}
	\item Kadang-kadang, kita membutuhkan suatu tipe data yang sifatnya komposit; terdiri dari beberapa data lainnya.
	\item Contoh kasusnya adalah ketika kita butuh suatu representasi dari titik. Setiap titik pada bidang memiliki dua komponen, yaitu \textbf{x} dan \textbf{y}.
	\item Memang bisa saja kita mendeklarasi dua variabel, yaitu \textbf{x} dan \textbf{y}. Namun bagaimana jika kita hendak membuat beberapa titik? Apakah kita harus membuat \textbf{x1}, \textbf{y1}, \textbf{x2}, \textbf{y2}, ...? Sungguh melelahkan!
	\item Karena itulah Pascal menyajikan suatu tipe data komposit, yaitu \alert{\textbf{record}}. 
\end{itemize}
\end{frame}

\begin{frame}[fragile]
\frametitle{Tipe Data Komposit: Record}
\begin{itemize}
	\item \textbf{Record} dapat dideklarasikan pada blok \textbf{type}, yang letaknya sebelum \textbf{var}.
	\begin{lstlisting}
		type
		   <nama record> = 
		      record
		         <variabel1> : <tipe1>;
		         <variabel2> : <tipe2>;
		         ...
		      end;
		var
		   ...
	\end{lstlisting}
	\item Setelah dideklarasikan, sebuah tipe data \textbf{$<$nama record$>$} sudah bisa digunakan.
	\item Untuk mengakses nilai dari \textbf{$<$variabel 1$>$} dari suatu variabel bertipe \textbf{record}, gunakan tanda titik (.).
\end{itemize}
\end{frame}

\begin{frame}[fragile]
\frametitle{Tipe Data Komposit: Record (lanj.)}
\begin{itemize}
	\item Sebagai contoh, perhatikan contoh program titik.pas berikut:
	\begin{lstlisting}
		type
		   titik = 
		      record
		         x, y : longint;
		      end;
		var
		   t1, t2 : titik;
		begin
		   t1.x := 5;
		   t1.y := 3;
		
		   t2.x := 1;
		   t2.y := 2;
		
		   writeln(t1.x, ',', t1.y);
		   writeln(t2.x, ',', t2.y);
		end.
	\end{lstlisting}
\end{itemize}
\end{frame}

\begin{frame}
\frametitle{Tipe Data Komposit: Record}
\begin{itemize}
	\item memori yang dibutuhkan bagi sebuah tipe data \textbf{record} sama dengan jumlah memori tipe data yang menyusunnya.
	\item Artinya, \textbf{record} bernama \textbf{titik} pada contoh titik.pas mengkonsumsi memori yang sama dengan dua buah longint, yaitu 8 byte.
\end{itemize}
\end{frame}

\begin{frame}[fragile]
\frametitle{Contoh Program: tipedasar.pas}
\begin{itemize}
	\item Pahami program berikut ini dan coba jalankan!
	\begin{lstlisting}
	var
	   p1, p2 : longint;
	   x, y : double;
	   teks : string;
	begin
	   p1 := 100;
	   p2 := p1;
	   writeln('p1: ', p1, ' p2: ', p2);
	   
	   x := 3.1418;
	   y := 234.432;
	   writeln('x: ', x);
	   writeln('y: ', y);
	   
	   teks := 'ini adalah string';
	   writeln('teks: ', teks);
	end.
	\end{lstlisting}
\end{itemize}
\end{frame}

\begin{frame}[fragile]
\frametitle{Penjelasan Program: tipedasar.pas}
\begin{itemize}
	\item Berikut adalah keluaran dari program tipedasar.pas:
	\begin{lstlisting}
	p1: 100 p2: 100
	x:  3.14180000000000E+000
	y:  2.34432000000000E+002
	teks: ini adalah string
	\end{lstlisting}
	\item Perhatikan bahwa perintah \textbf{p2 := p1} sama artinya dengan \\ \textbf{p2 := 100}, karena \textbf{p1} sendiri mengacu pada nilai 100.
	\item Untuk tipe data \textit{floating point}, bilangan tercetak dalam notasi \textit{scientific}, yaitu $3.1418 \times 10^{0}$ dan $2.34432 \times 10^{2}$.
\end{itemize}
\end{frame}

\begin{frame}
\frametitle{Ordinalitas}
\begin{itemize}
	\item Menurut keberurutannya, tipe data dapat dibedakan menjadi tipe data \alert{ordinal} atau \alert{non-ordinal}.
	\item Suatu tipe data memiliki sifat ordinal jika untuk suatu elemennya, kita bisa mengetahui secara pasti apa elemen sebelum atau selanjutnya. Contoh:
	\begin{itemize}
		\item Diberikan bilangan bulat 6, kita tahu pasti sebelumnya adalah angka 5 dan sesudahnya adalah angka 7.
		\item Diberikan karakter 'y', kita tahu pasti sebelumnya adalah karakter 'x' dan sesudahnya adalah karakter 'z'.
	\end{itemize}
	\item Dengan demikian, seluruh tipe data bilangan bulat dan karakter adalah tipe data ordinal.
\end{itemize}
\end{frame}

\begin{frame}
\frametitle{Ordinalitas (lanj.)}
\begin{itemize}
	\item Kebalikannya, suatu tipe data dinyatakan memiliki sifat non-ordinal jika kita tidak bisa menentukan elemen sebelum dan sesudahnya. Contohnya:
	\begin{itemize}
		\item Diberikan bilangan riil 6, apakah elemen sesudahnya 7, atau 6.1, atau 6.01, atau 6.001, atau 6.00000000001?
		\item Diberikan string 'telur', apakah elemen sesudahnya? Apakah 'ayam'? 'bebek'?
	\end{itemize}
	\item \textbf{String} dan bilangan \textit{floating point} termasuk dalam tipe data non-ordinal.
\end{itemize}
\end{frame}

\begin{frame}
\frametitle{Yang Sudah Kita Pelajari...}
\begin{itemize}
	\item Mengenal konsep variabel.
	\item Mempelajari berbagai tipe data.
	\item Mempelajari cara deklarasi variabel.
	\item Mengenal operasi assignment.
\end{itemize}
\end{frame}

\begin{frame}
\frametitle{Selanjutnya...}
\begin{itemize}
	\item Mengenal tentang ekspresi.
	\item Mengenal input dan output.
\end{itemize}
\end{frame}

\end{document}