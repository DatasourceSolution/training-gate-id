\documentclass{beamer}
\usetheme{tokitex}

\usepackage{pgf}
\usepackage{graphics}
\usepackage{multirow}
\usepackage{multicol}
\usepackage{tabto}
\usepackage[english,bahasa]{babel}
\newtranslation[to=bahasa]{Section}{Bagian}
\newtranslation[to=bahasa]{Subsection}{Subbagian}

\usepackage{listings, lstautogobble}
\usepackage{color}

\definecolor{dkgreen}{rgb}{0,0.6,0}
\definecolor{gray}{rgb}{0.5,0.5,0.5}
\definecolor{mauve}{rgb}{0.58,0,0.82}

\lstset{frame=tb,
  language=pascal,
  aboveskip=3mm,
  belowskip=3mm,
  showstringspaces=false,
  columns=fullflexible,
  keepspaces=true,
  basicstyle={\small\ttfamily},
  numbers=none,
  numberstyle=\tiny\color{gray},
  keywordstyle=\color{blue},
  commentstyle=\color{dkgreen},
  stringstyle=\color{mauve},
  breaklines=true,
  breakatwhitespace=true,
  autogobble=true
}

\title{Pengurutan Dasar}
\author{Tim Olimpiade Komputer Indonesia}
\date{}

\begin{document}

\begin{frame}
\titlepage
\end{frame}

\begin{frame}
\frametitle{Pendahuluan}
Melalui dokumen ini, kalian akan:
\begin{itemize}
  \item Mempelajari konsep algoritma sederhana.
  \item Memahami berbagai algoritma pengurutan sederhana.
  \item Memahami keuntungan dan kerugian dari masing-masing algoritma.
\end{itemize}
\end{frame}

\begin{frame}
\frametitle{Pendahuluan (lanj.)}
\begin{itemize}
  \item Pengurutan sering digunakan dalam pemrograman untuk membantu membuat data lebih mudah diolah.
  \item Terdapat berbagai macam cara untuk melakukan pengurutan, masing-masing dengan keuntungan dan kekurangan masing-masing.
\end{itemize}
\end{frame}

\begin{frame}
\frametitle{Soal: Bebek Berbaris}
Deskripsi:
\begin{itemize}
  \item Sebelum masuk ke dalam kandang, para bebek akan berbaris terlebih dahulu.
  \item Seiring dengan berjalannya waktu, bebek-bebek tumbuh tinggi. Pertumbuhan ini berbeda-beda; ada bebek yang lebih tinggi dari bebek lainnya.
  \item Terdapat $N$ ekor bebek, bebek ke-$i$ memiliki tinggi sebesar $h_i$.
  \item Perbedaan tinggi ini menyebabkan barisan terlihat kurang rapi, sehingga Pak Dengklek ingin agar bebek-bebek berbaris dari yang paling pendek ke paling tinggi.
  \item Bantulah para bebek untuk mengurutkan barisan mereka!
\end{itemize}
\end{frame}

\begin{frame}
\frametitle{Soal: Bebek Berbaris (lanj.)}
Format masukan:
\begin{itemize}
  \item Baris pertama berisi sebuah bilangan bulat, yaitu $N$.
  \item Baris kedua berisi $N$ bilangan bulat. Bilangan ke-$i$ menyatakan $h_i$.
\end{itemize}
Format keluaran:
\begin{itemize}
  \item Keluarkan ketinggian bebek dalam keadaan terurut menaik, seekor bebek pada setiap barisnya.
\end{itemize}
Batasan:
\begin{itemize}
  \item $1 \le N \le 1.000$
  \item $1 \le h_i \le 100.000$, untuk $1 \le i \le N$
\end{itemize}
\end{frame}

\section{Bubble Sort}
\frame{\sectionpage}

\begin{frame}
\frametitle{$<$TODO: Konten Wajib$>$}
\begin{itemize}
  \item ide dasar
  \item contoh kode
  \item analisis kompleksitas (best case, worst case, average case, semuanya sama)
  \item bisa gunakan berkas "TEKS : Sorting Standard.html"  (lihat  https://www.dropbox.com/sh/zxcd1uct6rf3mio/AAD5v7W-EaypSSh7uP8D5FRVa?dl=0 sebagai acuan)
\end{itemize}
\end{frame}

\section{Insertion Sort}
\frame{\sectionpage}

\begin{frame}
\frametitle{$<$TODO: Konten Wajib$>$}
\begin{itemize}
  \item ide dasar
  \item contoh kode
  \item penjelasan insertion sort bisa cepat kalau datanya sudah hampir sorted
  \item analisis kompleksitas (best case, worst case, average case)
  \item bisa gunakan berkas "TEKS : Sorting Standard.html"  (lihat  https://www.dropbox.com/sh/zxcd1uct6rf3mio/AAD5v7W-EaypSSh7uP8D5FRVa?dl=0 sebagai acuan)
\end{itemize}
\end{frame}

\section{Selection Sort}
\frame{\sectionpage}

\begin{frame}
\frametitle{$<$TODO: Konten Wajib$>$}
\begin{itemize}
  \item ide dasar
  \item contoh kode
  \item analisis kompleksitas (best case, worst case, average case, semuanya sama)
  \item bisa gunakan berkas "TEKS : Sorting Standard.html"  (lihat  https://www.dropbox.com/sh/zxcd1uct6rf3mio/AAD5v7W-EaypSSh7uP8D5FRVa?dl=0 sebagai acuan)
\end{itemize}
\end{frame}

\section{Counting Sort}
\frame{\sectionpage}

\begin{frame}
\frametitle{$<$TODO: Konten Wajib$>$}
\begin{itemize}
  \item ide dasar
  \item kapan counting sort bisa dipakai
  \item tekankan bahwa counting sort itu sangat cepat, hanya O(N)
  \item contoh kode
  \item analisis kompleksitas (best case, worst case, average case, semuanya sama)
  \item bisa gunakan berkas "TEKS : Sorting Standard.html"  (lihat https://www.dropbox.com/sh/zxcd1uct6rf3mio/AAD5v7W-EaypSSh7uP8D5FRVa?dl=0 sebagai acuan)
\end{itemize}
\end{frame}

\end{document}
