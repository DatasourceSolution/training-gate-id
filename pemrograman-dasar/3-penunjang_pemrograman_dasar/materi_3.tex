\documentclass{beamer}
\usetheme{tokitex}

\usepackage{graphics}
\usepackage{multirow}
\usepackage{tabto}

\usepackage[english,bahasa]{babel}
\newtranslation[to=bahasa]{Section}{Bagian}
\newtranslation[to=bahasa]{Subsection}{Subbagian}

\usepackage{listings, lstautogobble}
\usepackage{color}

\definecolor{dkgreen}{rgb}{0,0.6,0}
\definecolor{gray}{rgb}{0.5,0.5,0.5}
\definecolor{mauve}{rgb}{0.58,0,0.82}

\lstset{frame=tb,
  language=pascal,
  aboveskip=1mm,
  belowskip=1mm,
  showstringspaces=false,
  columns=fullflexible,
  keepspaces=true,
  basicstyle={\small\ttfamily},
  numbers=none,
  numberstyle=\tiny\color{gray},
  keywordstyle=\color{blue},
  commentstyle=\color{dkgreen},
  stringstyle=\color{mauve},
  breaklines=true,
  breakatwhitespace=true,
  autogobble=true
}

\title{Penunjang Pemrograman Dasar}
\author{Tim Olimpiade Komputer Indonesia}
\date{}

\begin{document}

\begin{frame}
\titlepage
\end{frame}

\begin{frame}
\frametitle{Pendahuluan}
Dokumen ini berisi tambahan pengetahuan yang dapat menunjang pemrograman dasar kalian.
\vfill
Melalui dokumen ini, kalian akan:
\begin{itemize}
  \item Mengenal komentar.
  \item Memahami pesan kesalahan.
  \item Memahami I/O \textit{redirection}.
\end{itemize}
\end{frame}

\section{Komentar}
\frame{\sectionpage}

\begin{frame}
\frametitle{Mengenal Komentar}
\begin{itemize}
  \item Program yang pendek seperti kuadrat.pas atau jumlah.pas yang sebelumnya telah kita jumpai memang sederhana dan mudah dipahami.
  \item Ketika program sudah mulai panjang dan kompleks, memahami alur kerja suatu program menjadi lebih sulit.
  \item Salah satu cara untuk membantu memahami alur kerja program adalah dengan menulis komentar.
\end{itemize}
\end{frame}

\begin{frame}[fragile]
\frametitle{Komentar}
\begin{itemize}
  \item Merupakan bagian dari program yang diabaikan oleh \textit{compiler}.
  \item Kita bisa menuliskan apapun di dalam komentar. Misalnya: apa yang dilakukan oleh suatu bagian program atau catatan tertentu.
  \item Pada Pascal, komentar dapat dituliskan dalam dua gaya:
  \begin{itemize}
    \item Satu baris, dituliskan dengan awalan dua \textit{slash}
    \begin{lstlisting}
      // ini adalah komentar, hanya bisa sebaris
      // jika perlu baris baru, tambahkan // lagi
    \end{lstlisting}
    \item Beberapa baris, dituliskan dengan mengawali dan mengakhiri komentar dengan kurawal (\{\}) atau kurung bintang ((**)).
    \begin{lstlisting}
      { ini adalah komentar, yang memungkinkan
        ditulis dalam beberapa baris }
      (* ini juga komentar,
         ditulis dengan cara lain *)
    \end{lstlisting}
  \end{itemize}
\end{itemize}
\end{frame}

\begin{frame} [fragile]
\frametitle{Contoh Program: kuadrat3.pas}
\begin{itemize}
  \item Perhatikan program berikut:
  \begin{lstlisting}
    var
      a, b, c, x, hasil: longint;
    begin
      // inisialisasi
      a := 1;
      b := 3;
      c := -2;

      // baca nilai x
      readln(x);

      // hitung hasil fungsi
      hasil := a*sqr(x) + b*x + c;

      // cetak
      writeln('ax^2 + bx + c = ', hasil);
    end.
  \end{lstlisting}
\end{itemize}
\end{frame}

\begin{frame}
\frametitle{Penjelasan Program: kuadrat3.pas}
\begin{itemize}
  \item Program kuadrat3.pas menjadi lebih deskriptif ketika kita menuliskan komentar.
  \item Ketika program yang dibuat sudah mulai panjang, komentar menjadi efektif untuk membantu kalian "mengingat kembali" apa yang telah diketikkan sebelumnya.
  \item Komentar juga berguna ketika program kalian akan dibaca oleh orang lain, sehingga orang lain bisa memahaminya dengan lebih mudah.
  \item Gunakan komentar secukupnya, jangan terlalu berlebihan juga.
\end{itemize}
\end{frame}

\section{Pesan Kesalahan (Error)}
\frame{\sectionpage}

\begin{frame}
\frametitle{Dua Jenis Error}
\begin{block}{Compilation Error}
  Kesalahan yang terjadi ketika program dikompilasi.\newline
  Contoh: terdapat kesalahan dalam pengetikan nama variabel, kurang tanda titik koma (;), atau salah penggunaan tipe data.
\end{block}
\begin{block}{Runtime Error}
  Kesalahan yang terjadi ketika program dieksekusi.\newline
  Contoh: saat program dieksekusi, tiba-tiba ada operasi pembagian dengan 0.
\end{block}

\begin{itemize}
  \item Mampu memahami pesan kesalahan yang disampaikan dapat membantu kita memperbaiki program secara lebih efisien.
\end{itemize}
\end{frame}

\begin{frame}[fragile]
\frametitle{\textit{Compilation Error}}
\begin{itemize}
  \item Pada kompilator Free Pascal, pesan kesalahan saat kompilasi biasanya disampaikan dengan format:
  \begin{lstlisting}
    <nama berkas>(<nomor baris>,<nomor kolom>) Error: <jenis error>
  \end{lstlisting}
  Contoh:
  \begin{lstlisting}
    tes.pas(4,13) Error: Identifier not found "nilai"
  \end{lstlisting}
  \item Artinya pada berkas tes.pas, baris 4, kolom 13, terdapat kesalahan berupa: sebuah \textit{identifier} bernama "nilai" tidak ditemukan. Untuk memperbaikinya, "nilai" harus dideklarasikan terlebih dahulu.
  \item Ketika suatu program memiliki \textit{compilation error}, kompilasi akan dibatalkan dan program tidak bisa dikompilasi sampai kesalahannya diperbaiki.
\end{itemize}
\end{frame}

\begin{frame}[fragile]
\frametitle{\textit{Compilation Warning}}
\begin{itemize}
  \item Selain \textit{error}, kompilator Free Pascal juga memberikan beberapa peringatan (\textit{warning}).
  \item Tidak seperti \textit{error}, \textit{warning} tidak membuat program batal dikompilasi. \textit{Warning} hanya seperti peringatan bahwa ada beberapa bagian dari program kalian yang mungkin bermasalah. Contoh:
  \begin{lstlisting}
    tes.pas(5,19) Warning: Variable "saldo" does not seem to be initialized
  \end{lstlisting}
  \item Artinya pada berkas tes.pas, baris 5, kolom 19, terdapat peringatan berupa: sebuah variabel bernama "saldo" belum diinisialisasi (belum diberi nilai awal). Untuk memperbaikinya, "saldo" dapat diinisialisasi terlebih dahulu, misalnya dengan 0 (jika "saldo" memiliki tipe data numerik).
\end{itemize}
\end{frame}

\begin{frame}
\frametitle{\textit{Runtime Error}}
\begin{itemize}
  \item Ketika program sudah berhasil dikompilasi, belum tentu program luput dari \textit{error} ketika dieksekusi.
  \item Program dapat mengalami \textit{error} ketika sedang dieksekusi karena berbagai hal:
  \begin{itemize}
    \item Melakukan pembagian dengan angka 0.
    \item Mengakses memori di luar yang telah dialokasikan.
    \item Mengalami \textit{stack overflow}.
  \end{itemize}
  \item Sebagian besar dari istilah dan masalah yang dijelaskan di atas mungkin kalian hadapi ketika sudah mempelajari tentang array dan rekursi.
\end{itemize}
\end{frame}

\begin{frame}
\frametitle{\textit{Error Code}}
\begin{itemize}
  \item Setiap \textit{runtime error} diasosiasikan dengan sebuah kode, biasa disebut dengan \alert{\textit{error code}}. Contoh:
  \begin{itemize}
    \item 200 Division by zero
    \item 201 Range check error
    \item 202 Stack overflow error
    \item 204 Invalid pointer operation
    \item 205 Floating point overflow
    \item 206 Floating point underflow
  \end{itemize}
  \item Dari \textit{error code} dan nama \textit{error} yang muncul, kalian bisa mendiagnosa masalah pada program dan memperbaikinya.
\end{itemize}
\end{frame}

\section{IO Redirection}
\frame{\sectionpage}

\begin{frame}[fragile]
\frametitle{IO Redirection}
\begin{itemize}
  \item Penjelasan tentang saluran input dan output sempat dijelaskan pada Sesi 2. Kali ini, kita akan memperdalamnya.
  \item Pada contoh yang lalu, kita sempat melakukan hal ini:
  \begin{lstlisting}
    jumlah < input.txt > output.txt
  \end{lstlisting}
  \item Ada dua hal yang dilakukan di sini:
  \begin{itemize}
    \item Memberikan STDIN kepada program jumlah yang akan dieksekusi dengan input.txt. Hal ini dilakukan dengan operator $<$.
    \item Memberikan STDOUT kepada program jumlah yang akan dieksekusi dengan output.txt. Hal ini dilakukan dengan operator $>$.
  \end{itemize}
\end{itemize}
\end{frame}

\begin{frame}[fragile]
\frametitle{IO Redirection (lanj.)}
\begin{itemize}
  \item Kita bisa melakukan hanya salah satu dari keduanya. Misalnya jika kita melakukan:
  \begin{lstlisting}
    jumlah < input.txt
  \end{lstlisting}
  \item Artinya program jumlah akan dijalankan dengan STDIN dari berkas input.txt, dan STDOUT ke layar.
  \item Hal ini akan membantu dalam mengurangi pengetikan berkas masukan terus menerus secara manual.
  \item Demikian pula dengan:
  \begin{lstlisting}
    jumlah > output.txt
  \end{lstlisting}
  \item Artinya program jumlah akan dijalankan dengan STDIN dari layar (kalian dapat mengetikkannya), dan STDOUT ke berkas output.txt.
  \end{itemize}
\end{frame}

\begin{frame}
\frametitle{Selanjutnya...}
\begin{itemize}
  \item Memasuki bagian yang menarik, yaitu struktur percabangan.
\end{itemize}
\end{frame}

\end{document}
