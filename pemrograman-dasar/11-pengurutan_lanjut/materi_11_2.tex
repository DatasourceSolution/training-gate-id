\documentclass{beamer}
\usetheme{tokitex}

\usepackage{graphics}
\usepackage{multirow}
\usepackage{tabto}

\usepackage[english,bahasa]{babel}
\newtranslation[to=bahasa]{Section}{Bagian}
\newtranslation[to=bahasa]{Subsection}{Subbagian}

\usepackage{listings, lstautogobble}
\usepackage{color}

\definecolor{dkgreen}{rgb}{0,0.6,0}
\definecolor{gray}{rgb}{0.5,0.5,0.5}
\definecolor{mauve}{rgb}{0.58,0,0.82}

\lstset{frame=tb,
  language=pascal,
  aboveskip=1mm,
  belowskip=1mm,
  showstringspaces=false,
  columns=fullflexible,
  keepspaces=true,
  basicstyle={\small\ttfamily},
  numbers=none,
  numberstyle=\tiny\color{gray},
  keywordstyle=\color{blue},
  commentstyle=\color{dkgreen},
  stringstyle=\color{mauve},
  breaklines=true,
  breakatwhitespace=true,
  autogobble=true
}
\title{Pencarian Lanjut - Quick Sort}
\author{Tim Olimpiade Komputer Indonesia}
\date{}

\begin{document}

\begin{frame}
\titlepage
\end{frame}

\begin{frame}
\frametitle{Pendahuluan}
Melalui dokumen ini, kalian akan:
\begin{itemize}
    \item Mempelajari konsep \textit{quick sort}.
    \item Mengimplementasikan \textit{quick sort}.
\end{itemize}
\end{frame}

\section{Konsep}
\frame{\sectionpage}

\begin{frame}
\frametitle{$<$TODO: Konten Wajib$>$}
\begin{itemize}
    \item ide dasar: divide (pilih 1 pivot, partisi semua yang $<=$ pivot ke kiri, yg $>=$ pivot ke kanan), conquer (kalau sudah tinggal 1 elemen berarti sudah sorted), combine (tinggal "tempel" saja arraynya)
    \item cara partisi array yg efisien (gunakan algoritma Hoare)
    \item kasih demonstrasi cara partisi
    \item analisis kompleksitas (best case, average case, worst case)
    \item jelaskan strategi pemilihan pivot: randomized, pilih median dari elemen {depan, tengah, belakang}, dsb,
    \item kemudian tekankan bahwa dalam soal programming, kasus worst case akan sangaaaaat jarang ditemui, sehingga menggunakan pivot elemen di tengah/depan/belakang sudah cukup
\end{itemize}
\end{frame}

\begin{frame}
\frametitle{$<$TODO: Konten Wajib$>$}
\begin{itemize}
    \item jelaskan bahwa quick sort bukan stable sort (sebelum itu, jelaskan apa itu stable sort)
    \item bisa gunakan berkas "TEKS : Sorting Standard.html" (lihat  https://www.dropbox.com/sh/zxcd1uct6rf3mio/AAD5v7W-EaypSSh7uP8D5FRVa?dl=0 sebagai acuan)
\end{itemize}
\end{frame}


\section{Implementasi}
\frame{\sectionpage}

\begin{frame}
\frametitle{$<$TODO: Konten Wajib$>$}
\begin{itemize}
    \item buat fungsi rekursif
    \item jelaskan per bagian kodenya
    \item jelaskan bahwa quick sort tidak butuh memori tambahan, karena partisi dapat dilakukan secara \textit{in-place}
\end{itemize}
\end{frame}

\section{Perbandingan Algoritma Sort}
\frame{\sectionpage}

\begin{frame}
\frametitle{$<$TODO: Konten Wajib$>$}
\begin{itemize}
    \item bandingkan selection sort, bubble sort, insertion sort, merge sort, dan quick sort. Tekankan bahwa merge/quick sort jauh lebih cepat dari sisanya
    \item tekankan bahwa counting sort tetap lebih cepat dari quick/merge sort, meskipun perlu ada karakteristik khusus pada data
    \item bandingkan quick sort dengan merge sort, biasa orang lebih sering menggunakan quick sort karena konstantanya lebih ringan dari merge sort, codingnya lebih pendek, dan tidak butuh memori tambahan.
\end{itemize}
\end{frame}

\end{document}
