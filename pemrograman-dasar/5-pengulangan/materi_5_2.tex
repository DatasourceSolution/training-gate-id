\documentclass{beamer}
\usetheme{tokitex}

\usepackage{graphics}
\usepackage{multirow}
\usepackage{tabto}

\usepackage[english,bahasa]{babel}
\newtranslation[to=bahasa]{Section}{Bagian}
\newtranslation[to=bahasa]{Subsection}{Subbagian}

\usepackage{listings, lstautogobble}
\usepackage{color}

\definecolor{dkgreen}{rgb}{0,0.6,0}
\definecolor{gray}{rgb}{0.5,0.5,0.5}
\definecolor{mauve}{rgb}{0.58,0,0.82}

\lstset{frame=tb,
  language=pascal,
  aboveskip=1mm,
  belowskip=1mm,
  showstringspaces=false,
  columns=fullflexible,
  keepspaces=true,
  basicstyle={\small\ttfamily},
  numbers=none,
  numberstyle=\tiny\color{gray},
  keywordstyle=\color{blue},
  commentstyle=\color{dkgreen},
  stringstyle=\color{mauve},
  breaklines=true,
  breakatwhitespace=true,
  autogobble=true
}

\title{Pengulangan Lanjut}
\author{Tim Olimpiade Komputer Indonesia}
\date{}

\begin{document}

\begin{frame}
\titlepage
\end{frame}

\begin{frame}
\frametitle{Pendahuluan}
Melalui dokumen ini, kalian akan:
\begin{itemize}
  \item Memahami penggunaan pengulangan yang bersarang.
  \item Memecahkan beberapa persoalan dengan pengulangan.
\end{itemize}
\end{frame}

\begin{frame}[fragile]
\frametitle{Motivasi: Pola 0}
\begin{itemize}
  \item Pak Dengklek sedang mengajari bebek-bebeknya yang baru lahir dari telur untuk menggambar.
  \item Pak Dengklek akan memberikan sebuah bilangan, misalnya \textbf{N}.
  \item Para bebek diminta untuk mencetak karakter bintang (*) yang tersusun \textbf{N} baris.
  \item Contoh untuk \textbf{N} = 3:
  \begin{lstlisting}
    *
    *
    *
  \end{lstlisting}
\end{itemize}
\end{frame}

\begin{frame}[fragile]
\frametitle{Motivasi (lanj.)}
\begin{itemize}
  \item Tentu saja solusinya sederhana, cukup gunakan salah satu struktur pengulangan yang kalian kuasai.
  \item Untuk kasus ini, lebih tepat digunakan \textbf{for}:
  \begin{lstlisting}
    for i := 1 to N do begin
      writeln('*');
    end;
  \end{lstlisting}
\end{itemize}
\end{frame}

\begin{frame}[fragile]
\frametitle{Motivasi: Pola 1}
\begin{itemize}
  \item Setelah para bebek mampu mengerjakan "Pola 0", Pak Dengklek memberikan persoalan yang sedikit lebih sulit.
  \item Diberikan dua bilangan, misalnya \textbf{N} dan \textbf{M}.
  \item Cetak karakter bintang (*) yang tersusun \textbf{N} baris dan \textbf{M} kolom!

  \item Contoh untuk \textbf{N} = 3 dan \textbf{M} = 5:
  \begin{lstlisting}
  *****
  *****
  *****
  \end{lstlisting}

  \item Kali ini, untuk setiap barisnya kita perlu melakukan pengulangan untuk mencetak \textbf{M} karakter bintang!
\end{itemize}
\end{frame}

\begin{frame}[fragile]
\frametitle{Contoh Program: pola1\_1.pas}
\begin{itemize}
  \item Kita bisa membuat "for di dalam for", sehingga membentuk struktur yang bersarang.
  \begin{lstlisting}
    var
      N, M : longint;
      i, j : longint;
    begin
      readln(N, M);

      for i := 1 to N do begin
        for j := 1 to M do begin
          write('*');
        end;
        writeln;
      end;
    end.
  \end{lstlisting}
\end{itemize}
\end{frame}

\begin{frame}[fragile]
\frametitle{Contoh Program: pola1\_2.pas}
\begin{itemize}
  \item Tentu saja kita bisa melakukannya dengan struktur pengulangan yang lain, misalnya \textbf{while}:
  \begin{lstlisting}
    var
      N, M : longint;
      i, j : longint;
    begin
      readln(N, M);

      for i := 1 to N do begin
        j := 1;
        while (j <= M) do begin
          write('*');
          j := j + 1;
        end;
        writeln;
      end;
    end.
  \end{lstlisting}
\end{itemize}
\end{frame}

\begin{frame}[fragile]
\frametitle{Contoh Lain: Pola 2}
\begin{itemize}
  \item Soal "Pola 1" dapat diselesaikan bebek-bebek muda dengan mudah. Dengan demikian Pak Dengklek memberikan soal yang lebih menantang.
  \item Diberikan sebuah bilangan, misalnya \textbf{N}.
  \item Cetak "struktur segitiga rata kiri" yang terdiri dari \textbf{N} baris.
  \item Misalnya untuk \textbf{N} = 5, hasilnya adalah:
  \begin{lstlisting}
    *
    **
    ***
    ****
    *****
  \end{lstlisting}
\end{itemize}
\end{frame}

\begin{frame}[fragile]
\frametitle{Contoh Solusi: pola2.pas}
\begin{itemize}
  \item Berikut ini adalah contoh solusinya, dimodifikasi dari pola1\_1.pas:
  \begin{lstlisting}
    var
      N : longint;
      i, j : longint;
    begin
      readln(N);

      for i := 1 to N do begin
        for j := 1 to i do begin
          write('*');
        end;
        writeln;
      end;
    end.
  \end{lstlisting}
\end{itemize}
\end{frame}

\begin{frame}[fragile]
\frametitle{Latihan: Pola 3}
\begin{itemize}
  \item Pak Dengklek kemudian memberikan tugas bagi para bebek muda untuk dikerjakan di rumah nanti.
  \item Diberikan sebuah bilangan, misalnya \textbf{N}.
  \item Cetak "struktur segitiga rata kanan" yang terdiri dari \textbf{N} baris.
  \item Misalnya untuk \textbf{N} = 5, hasilnya adalah:
  \begin{lstlisting}
        *
       **
      ***
     ****
    *****
  \end{lstlisting}
  \item Petunjuk: cetak spasi terlebih dahulu!
\end{itemize}
\end{frame}

\begin{frame}
\frametitle{Break \& Continue}
\begin{itemize}
  \item Kadang kala, kita membutuhkan suatu pengulangan untuk diberhentikan secara paksa atau lompat ke iterasi berikutnya pada \textbf{for}.
  \item Pascal menyediakan kedua fitur tersebut, yaitu dengan kata kunci \alert{\textbf{break}} dan \alert{\textbf{continue}}.
\end{itemize}
\end{frame}

\begin{frame}
\frametitle{Break \& Continue (lanj.)}
\begin{block}{Break}
Penggunaan \textbf{break} akan membuat program keluar dari pengulangan yang mengandung kata kunci tersebut.
\end{block}
\begin{block}{Continue}
Penggunaan \textbf{continue} akan membuat program kembali ke baris paling atas pengulangan, yaitu baris "for .. to .. do" (untuk \textbf{for}), "while ... do" (untuk \textbf{while}), dan "repeat ... until"(untuk \textbf{repeat}).
\end{block}
\end{frame}

\begin{frame}[fragile]
\frametitle{Contoh Soal: Berhitung 1}
\begin{itemize}
  \item Setelah para bebek mahir dalam menggambar pola, kini Pak Dengklek ingin mengajarkan mereka tentang berhitung.
  \item Pak Dengklek akan memberikan dua bilangan, yaitu \textbf{N} dan \textbf{M}.
  \item Para bebek diminta untuk menuliskan bilangan dari 1 sampai dengan \textbf{N}. Namun, ketika bilangan yang hendak ditulis adalah \textbf{M}, jangan cetak bilangan itu dan jangan cetak bilangan apapun lagi.
  \item Setelah selesai mencetak bilangan, cetak "selesai".
  \item Contoh untuk \textbf{N} = 10 dan \textbf{M} = 5:
  \begin{lstlisting}
    1
    2
    3
    4
    selesai
  \end{lstlisting}
\end{itemize}
\end{frame}

\begin{frame}[fragile]
\frametitle{Contoh Program: break.pas}
\begin{itemize}
  \item Berikut ini adalah contoh solusi dari soal "Berhitung 1".
  \begin{lstlisting}
    var
      N, M : longint;
      i : longint;
    begin
      readln(N, M);

      for i := 1 to N do begin
        if (i = M) then begin
          break;
        end;

        writeln(i);
      end;

      writeln('selesai');
    end.
  \end{lstlisting}
\end{itemize}
\end{frame}

\begin{frame}
\frametitle{Penjelasan Program: break.pas}
\begin{itemize}
  \item Ketika \textit{break} ditemui, pengulangan "for ... to ... do" akan diberhentikan secara paksa dan lanjut mengeksekusi perintah selanjutnya, yaitu mencetak tulisan "selesai".
\end{itemize}
\end{frame}

\begin{frame}[fragile]
\frametitle{Contoh Soal: Berhitung 2}
\begin{itemize}
  \item Kali ini Pak Dengklek mengubah soalnya: diberikan dua bilangan, yaitu \textbf{N} dan \textbf{M}.
  \item Para bebek diminta untuk menuliskan bilangan dari 1 sampai dengan \textbf{N}. Namun, ketika bilangan yang hendak ditulis adalah \alert{kelipatan} dari \textbf{M}, jangan cetak bilangan itu.
  \item Setelah selesai mencetak bilangan, cetak "selesai".
  \item Contoh untuk \textbf{N} = 10 dan \textbf{M} = 2:
  \begin{lstlisting}
    1
    3
    5
    7
    9
    selesai
  \end{lstlisting}
\end{itemize}
\end{frame}

\begin{frame}[fragile]
\frametitle{Contoh Program: continue.pas}
\begin{itemize}
  \item Berikut ini adalah contoh solusi dari soal "Berhitung 2".
  \begin{lstlisting}
    var
      N, M : longint;
      i : longint;
    begin
      readln(N, M);

      for i := 1 to N do begin
        if (i mod M = 0) then begin
          continue;
        end;

        writeln(i);
      end;

      writeln('selesai');
    end.
  \end{lstlisting}
\end{itemize}
\end{frame}

\begin{frame}
\frametitle{Penjelasan Program: continue.pas}
\begin{itemize}
  \item Ketika \textit{continue} ditemui, eksekusi perintah di dalam "for ... to ... do" untuk pencacah tersebut langsung dilewati dan lanjut ke pencacah selanjutnya.
  \item Artinya, untuk \textbf{N} = 10 dan \textbf{M} = 2, ketika nilai \textbf{i} = 2 dan \textbf{continue} ditemui, eksekusi akan dilewati dan lanjut ke perintah paling awal dalam pengulangan dengan \textbf{i} = 3.
\end{itemize}
\end{frame}


\begin{frame}
\frametitle{Contoh Soal: Tes Keprimaan}
\begin{itemize}
  \item Kemudian Pak Dengklek beralih persoalan yang lain, yaitu tentang teori bilangan.
  \item Diberikan sebuah bilangan positif yang lebih dari 1, misalnya \textbf{N}.
  \item Suatu bilangan \textbf{N} dikatakan prima apabila \textbf{N} positif dan hanya habis dibagi oleh 1 dan dirinya sendiri.
  \item Jika \textbf{N} prima, cetak "$<$N$>$ adalah bilangan prima" dan jika tidak, cetak "$<$N$>$ bukan bilangan prima".
\end{itemize}
Bagaimana kalian akan menyelesaikan persoalan ini?
\end{frame}

\begin{frame}
\frametitle{Solusi 1}
\begin{itemize}
  \item Salah satu solusi yang sederhana adalah: periksa semua bilangan di antara 2 sampai dengan \textbf{N}-1.
  \item Jika ada setidaknya satu bilangan yang habis membagi \textbf{N}, artinya \textbf{N} bukan prima.
\end{itemize}
\end{frame}


\begin{frame}[fragile]
\frametitle{Solusi 1: prima1\_1.pas}
  \begin{lstlisting}
    var
      N, i : longint;
      prima : boolean;
    begin
      readln(N);

      prima := TRUE;
      for i := 2 to N-1 do begin
        if (N mod i = 0) then begin
          prima := FALSE;
        end;
      end;

      if (prima) then begin
        writeln(N, ' adalah bilangan prima');
      end else begin
        writeln(N, ' bukan bilangan prima');
      end;
    end.
  \end{lstlisting}
\end{frame}

\begin{frame}
\frametitle{Solusi 2}
\begin{itemize}
  \item Solusi 1 melakukan pemeriksaan dari 2 sampai dengan \textbf{N}-1, artinya dibutuhkan pemeriksaan sebanyak \textbf{N}-2 kali.
  \item Sebetulnya pemeriksaan bisa dihentikan ketika ditemukan setidaknya satu saja bilangan yang habis membagi \textbf{N}.
  \item Dengan demikian bisa digunakan \textbf{break} untuk memberhentikan pengulangan begitu ditemukan bilangan yang habis membagi \textbf{N}.
\end{itemize}
\end{frame}

\begin{frame}[fragile]
\frametitle{Solusi 2: prima1\_2.pas}
\begin{lstlisting}
  var
    N, i : longint;
    prima : boolean;
  begin
    readln(N);

    prima := TRUE;
    for i := 2 to N-1 do begin
      if (N mod i = 0) then begin
        prima := FALSE;
        break;
      end;
    end;

    if (prima) then begin
      writeln(N, ' adalah bilangan prima');
    end else begin
      writeln(N, ' bukan bilangan prima');
    end;
  end.
\end{lstlisting}
\end{frame}

\begin{frame}
\frametitle{Penggunaan Break}
\begin{itemize}
  \item Dalam prakteknya, penggunaan \textbf{break} kurang dianjurkan karena membuat program menjadi "kotor".
  \item Jika membutuhkan pengulangan yang suatu ketika perlu diberhentikan (sudah memenuhi syarat tertentu), gunakan \textbf{while} atau \textbf{repeat}.
  \item Solusi yang lebih baik ditunjukkan pada "prima1\_3.pas".
\end{itemize}
\end{frame}

\begin{frame}[fragile]
\frametitle{Solusi 3: prima1\_3.pas}
\begin{lstlisting}
  var
    N, i : longint;
    found : boolean;
  begin
    readln(N);

    found := FALSE;
    i := 2;
    while ((i < N) and not found) do begin
      if (N mod i = 0) then begin
        found := TRUE;
      end;
      i := i + 1;
    end;
\end{lstlisting}
\end{frame}

\begin{frame}[fragile]
\frametitle{Solusi 3: prima1\_3.pas (lanj.)}
\begin{lstlisting}
    if (not found) then begin
      writeln(N, ' adalah bilangan prima');
    end else begin
      writeln(N, ' bukan bilangan prima');
    end;
  end.
\end{lstlisting}
\end{frame}

\begin{frame}[fragile]
\frametitle{Contoh Soal: Pembangkit Prima}
\begin{itemize}
  \item Kini para bebek sudah mengetahui bagaimana memeriksa keprimaan suatu bilangan. Pak Dengklek memutuskan untuk memberikan tambahan soal untuk pekerjaan rumah.
  \item Diberikan sebuah bilangan bulat \textbf{N}. Pak Dengklek meminta bebek-bebeknya untuk menuliskan \textbf{N} bilangan prima pertama.
  \item Contoh untuk \textbf{N} = 5:
  \begin{lstlisting}
    2
    3
    5
    7
    11
  \end{lstlisting}
\end{itemize}
\end{frame}

\begin{frame}
\frametitle{Solusi: Pembangkit Prima}
\begin{itemize}
  \item Salah satu strategi yang dapat kalian gunakan adalah "selama belum ditemukan \textbf{N} bilangan prima, cari bilangan prima!".
  \item Bagaimana mencari bilangan prima? Coba saja dari 2, 3, 4, dan seterusnya sampai ditemukan \textbf{N} bilangan prima.
\end{itemize}
\end{frame}

\begin{frame}[fragile]
\frametitle{Contoh Solusi: prima2.pas}
\begin{lstlisting}
var
  N, cur, i : longint;
  count : longint;
  found : boolean;
begin
  readln(N);

  count := 0; (* banyaknya prima yang sudah ditemukan *)
  cur := 2; (* nilai yang akan diperiksa keprimaannya *)
  while (count < N) do begin
    found := FALSE;
    i := 2;
    while ((i < cur) and not found) do begin
      if (cur mod i = 0) then begin
        found := TRUE;
      end;
      i := i + 1;
    end;
\end{lstlisting}
\end{frame}

\begin{frame}[fragile]
\frametitle{Contoh Solusi: prima2.pas (lanj.)}
\begin{lstlisting}
    if (not found) then begin
      (* ditemukan prima!
      cetak dan tambahkan prima yg sudah ditemukan *)
      writeln(cur);
      count := count + 1;
    end;

    (* entah ini prima atau bukan, lanjut untuk
     memeriksa bilangan berikutnya *)
    cur := cur + 1;
  end;
  (* keluar dari while, dipastikan count = N *)
end.
\end{lstlisting}
\end{frame}

\begin{frame}
\frametitle{Penutup}
\begin{itemize}
  \item Percabangan dan pengulangan merupakan dua struktur kontrol yang sangat penting pada pemrograman.
  \item Kalian diharapkan berlatih sampai lancar di kedua hal tersebut, baru lanjut untuk mempelajari materi selanjutnya.
\end{itemize}
\end{frame}

\end{document}
