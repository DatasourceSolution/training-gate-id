\documentclass{beamer}
\usetheme{tokitex}

\usepackage[utf8]{inputenc}
\usepackage[T1]{fontenc}
\usepackage{arev}

\title{Petunjuk Makan Siang di Kantin}
\subtitle{Sambil Menyelam Minum Air}
\date{\today}
\institute{Gedung Sate Bundar}

\begin{document}

\begin{frame}
\titlepage
\end{frame}

\section{Introduction}    
\frame{\sectionpage}
\begin{frame}
\frametitle{Judul Ngasal}
\framesubtitle{Ini subjudul}
\begin{itemize}
\item Bulet
	\begin{itemize}
	\item Bulet kecil
	\item Bulet kecil 2
	\end{itemize}
\item Bulat
	\begin{enumerate}
	\item Angka 1
	\item Angka 2
	\item \alert{wakacau}
	\end{enumerate}
\end{itemize}
\end{frame}

\subsection{Text}
\begin{frame}
\frametitle{Judul Lain}
Lorem ipsum dolor sit amet, consectetur adipisicing elit, sed do eiusmod tempor incididunt ut labore et dolore magna aliqua. Ut enim ad minim veniam, quis nostrud exercitation ullamco laboris nisi ut aliquip ex ea commodo consequat.
\end{frame}

\subsection{Teorema}
\begin{frame}
\frametitle{Blocks}
\begin{block}{Phytagoras}
$a^2 + b^2 = c^2$
\end{block}

\begin{block}{Definisi: Lucu}
\begin{itemize}
\item lu.cu 1 a menggelikan hati; menimbulkan tertawa; jenaka: cerita ini ~ sekali; tukang lawaknya tidak ~
\item ber.lu.cu v ark 1 melucu
\item me.lu.cu v 1 mengucapkan (berbuat) sesuatu yg menggelikan hati: ia pandai ~
\item ke.lu.cu.an n 1 kejenakaan
\end{itemize}
\end{block}
\tez{asd}
\end{frame}
\end{document}
