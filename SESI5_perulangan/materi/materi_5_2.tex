\documentclass{beamer}
\usepackage{graphics}
\usepackage{multirow}
\usepackage{tabto}

\usepackage{listings}
\usepackage{color}

\definecolor{dkgreen}{rgb}{0,0.6,0}
\definecolor{gray}{rgb}{0.5,0.5,0.5}
\definecolor{mauve}{rgb}{0.58,0,0.82}

\lstset{frame=tb,
  language=pascal,
  aboveskip=3mm,
  belowskip=3mm,
  showstringspaces=false,
  columns=flexible,
  basicstyle={\small\ttfamily},
  numbers=none,
  numberstyle=\tiny\color{gray},
  keywordstyle=\color{blue},
  commentstyle=\color{dkgreen},
  stringstyle=\color{mauve},
  breaklines=true,
  breakatwhitespace=true,
  tabsize=3
}

\title{Perulangan Lanjut}
\author{Tim Olimpiade Komputer Indonesia}

\begin{document}

\begin{frame}
\titlepage
\end{frame}

\begin{frame}
\frametitle{Pendahuluan}
Melalui dokumen ini, kalian akan:
\begin{itemize}
	\item Memahami penggunaan perulangan yang bersarang.
	\item Memecahkan beberapa persoalan dengan perulangan.
\end{itemize}
\end{frame}

\begin{frame}[fragile]
\frametitle{Motivasi}
\begin{itemize}
	\item Diberikan sebuah bilangan, misalnya N.
	\item Cetak karakter bintang (*) yang tersusun N baris!
	
	\item Contoh untuk N = 3:
	\begin{lstlisting}
	*
	*
	*
	\end{lstlisting}
\end{itemize}
\end{frame}

\begin{frame}[fragile]
\frametitle{Motivasi (lanj.)}
\begin{itemize}
	\item Tentu saja solusinya sederhana, cukup gunakan salah satu struktur perulangan yang kalian kuasai.
	\item Untuk kasus ini, lebih tepat digunakan for:
	\begin{lstlisting}
		for i := 1 to N do begin
		    writeln('*');
		end;
	\end{lstlisting}	
\end{itemize}
\end{frame}

\begin{frame}[fragile]
\frametitle{Motivasi: Pola 1}
\begin{itemize}
	\item Diberikan dua bilangan, misalnya N dan M.
	\item Cetak karakter bintang (*) yang tersusun N baris dan M kolom!
	
	\item Contoh untuk N = 3 dan M = 5:
	\begin{lstlisting}
	*****
	*****
	*****
	\end{lstlisting}
	
	\item Kali ini, untuk setiap barisnya kita perlu melakukan perulangan untuk mencetak M karakter bintang!
\end{itemize}
\end{frame}

\begin{frame}[fragile]
\frametitle{Contoh Program: pola1\_1.pas}
\begin{itemize}
	\item Kita bisa membuat "for di dalam for", sehingga membentuk struktur yang bersarang.
	\begin{lstlisting}
		var
		    N,M:longint;
		    i,j:longint;
		begin
		    readln(N, M);
		
		    for i := 1 to N do begin
		        for j := 1 to M do begin
		            write('*');
		        end;
		        writeln;
		    end;
		end.
	\end{lstlisting}
\end{itemize}
\end{frame}

\begin{frame}[fragile]
\frametitle{Contoh Program: pola1\_2.pas}
\begin{itemize}
	\item Tentu saja kita bisa melakukannya dengan struktur perulangan yang lain, misalnya while:
	\begin{lstlisting}
		var
		    N,M:longint;
		    i,j:longint;
		begin
		    readln(N, M);
		
		    for i := 1 to N do begin
		        j := 1;
		        while (j <= M) do begin
		            write('*');
		            j := j + 1;
		        end;
		        writeln;
		    end;
		end.
	\end{lstlisting}
\end{itemize}
\end{frame}

\begin{frame}[fragile]
\frametitle{Contoh Lain: Pola 2}
\begin{itemize}
	\item Diberikan sebuah bilangan, misalnya N.
	\item Cetak "struktur segitiga rata kiri" yang terdiri dari N baris.
	\item Misalnya untuk N = 5, hasilnya adalah:
	\begin{lstlisting}
	*
	**
	***
	****
	*****
	\end{lstlisting} 
\end{itemize}
\end{frame}

\begin{frame}[fragile]
\frametitle{Contoh Solusi: pola2.pas}
\begin{itemize}
	\item Berikut ini adalah contoh solusinya, dimodifikasi dari "pola1\_1.pas":
	\begin{lstlisting}
		var
		    N:longint;
		    i,j:longint;
		begin
		    readln(N);
		
		    for i := 1 to N do begin
		        for j := 1 to i do begin
		            write('*');
		        end;
		        writeln;
		    end;
		end.
	\end{lstlisting}
\end{itemize}
\end{frame}

\begin{frame}[fragile]
\frametitle{Latihan: Pola 3}
Coba pikirkan solusi dan buat program dengan deskripsi berikut:
\begin{itemize}
	\item Diberikan sebuah bilangan, misalnya N.
	\item Cetak "struktur segitiga rata kanan" yang terdiri dari N baris.
	\item Misalnya untuk N = 5, hasilnya adalah:
	\begin{lstlisting}
	    *
	   **
	  ***
	 ****
	*****
	\end{lstlisting} 
	\item Petunjuk: cetak spasi terlebih dahulu!
\end{itemize}
\end{frame}

\begin{frame}
\frametitle{Contoh Soal: Tes Keprimaan}
Kita beralih ke persoalan yang lebih nyata: memeriksa apakah suatu bilangan prima.
\begin{itemize}
	\item Diberikan sebuah bilangan, misalnya N (1 $<$ N $\leq$ 100).
	\item Suatu bilangan N dikatakan prima apabila N positif dan hanya habis dibagi oleh 1 dan dirinya sendiri.
	\item Jika N prima, cetak "$<$N$>$ adalah bilangan prima" dan jika tidak, cetak "$<$N$>$ bukan bilangan prima".
\end{itemize}
Bagaimana kalian akan menyelesaikan persoalan ini?
\end{frame}

\begin{frame}[fragile]
\frametitle{Solusi 1: prima1\_1.pas}
	\begin{lstlisting}
		var
		    N,i:longint;
		    prima:boolean;
		begin
		    readln(N);
		
		    prima := TRUE;
		    for i := 2 to N-1 do begin
		        if (N mod i = 0) then begin
		            prima := FALSE;
		        end;
		    end;
		
		    if (prima) then begin
		        writeln(N, ' adalah bilangan prima');
		    end else begin
		        writeln(N, ' bukan bilangan prima');
		    end;
		end.
	\end{lstlisting}
\end{frame}

\begin{frame}[fragile]
\frametitle{}
\begin{itemize}
	\item 
\end{itemize}
\end{frame}

\begin{frame}
\frametitle{}
\begin{itemize}
	\item 
\end{itemize}
\end{frame}

\end{document}