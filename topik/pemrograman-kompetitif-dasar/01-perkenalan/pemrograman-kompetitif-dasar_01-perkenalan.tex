\documentclass{beamer}
\usetheme{tokitex}

\usepackage{graphics}
\usepackage{multirow}
\usepackage{tabto}

\usepackage[english,bahasa]{babel}
\newtranslation[to=bahasa]{Section}{Bagian}
\newtranslation[to=bahasa]{Subsection}{Subbagian}

\usepackage{listings, lstautogobble}
\usepackage{color}

\definecolor{dkgreen}{rgb}{0,0.6,0}
\definecolor{gray}{rgb}{0.5,0.5,0.5}
\definecolor{mauve}{rgb}{0.58,0,0.82}

\lstset{frame=tb,
  language=pascal,
  aboveskip=1mm,
  belowskip=1mm,
  showstringspaces=false,
  columns=fullflexible,
  keepspaces=true,
  basicstyle={\small\ttfamily},
  numbers=none,
  numberstyle=\tiny\color{gray},
  keywordstyle=\color{blue},
  commentstyle=\color{dkgreen},
  stringstyle=\color{mauve},
  breaklines=true,
  breakatwhitespace=true,
  autogobble=true
}

\title{Perkenalan}
\author{Tim Olimpiade Komputer Indonesia}
\date{}

\begin{document}

\begin{frame}
\titlepage
\end{frame}

\begin{frame}
\frametitle{Panduan Konten}
\begin{itemize}
  \item Apa yang akan dipelajari pada topik pemrograman kompetitif dasar
  \item Demonstrasi problem solving
  \item Pemanasan dengan soal ad hoc
  \item Slide ini tidak butuh materi, hanya perkenalan
\end{itemize}
\end{frame}

\begin{frame}
\frametitle{Tips}
Demi keseragaman dan kemudahan, gunakan command yang sudah didefinisikan di ../config.tex:
% Komentar: meskipun ada yang menghasilkan format yang sama, tetapi menggunakan command ini mempermudah kalau ke depannya mau mengubah format, tidak perlu find and replace 1 per 1. 
\begin{itemize}
  \item Gunakan \textbackslash id\{\} untuk menulis identifier (nama fungsi, variabel), contoh: \textbackslash id\{primeCount\} akan mencetak \id{primeCount}.
  \item Gunakan \textbackslash progTerm\{\} untuk menulis istilah dalam pemrograman (tipe data, jenis error, for), contoh: \textbackslash progTerm\{integer\} akan mencetak \progTerm{integer}.
  \item Gunakan \textbackslash foreignTerm\{\} untuk menulis istilah asing lainnya, contoh: \textbackslash foreignTerm\{default\} akan mencetak \foreignTerm{default}.
\end{itemize}
\end{frame}

\begin{frame}
\frametitle{Tips (lanj.)}
\begin{itemize}
  \item Jika diperlukan, gunakan \textbackslash newTerm\{\} untuk menulis istilah yang baru diperkenalkan, contoh: \textbackslash newTerm\{rekursi\} akan mencetak \newTerm{rekursi}.
  \item Gunakan \textbackslash emp\{\} untuk memberi penekanan, contoh: \textbackslash emp\{sangat lambat\} akan mencetak \emp{sangat lambat}.
  \item Gunakan \textbackslash statement\{\} untuk merujuk pada statement program, contoh: \textbackslash statement\{count := count + 1\} akan mencetak \statement{count := count + 1}.
\end{itemize}
\end{frame}

\end{document}
