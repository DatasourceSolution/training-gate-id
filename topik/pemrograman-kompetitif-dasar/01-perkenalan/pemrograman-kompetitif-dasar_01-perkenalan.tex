\documentclass{beamer}
\usetheme{tokitex}

\usepackage{graphics}
\usepackage{multirow}
\usepackage{tabto}

\usepackage[english,bahasa]{babel}
\newtranslation[to=bahasa]{Section}{Bagian}
\newtranslation[to=bahasa]{Subsection}{Subbagian}

\usepackage{listings, lstautogobble}
\usepackage{color}

\definecolor{dkgreen}{rgb}{0,0.6,0}
\definecolor{gray}{rgb}{0.5,0.5,0.5}
\definecolor{mauve}{rgb}{0.58,0,0.82}

\lstset{frame=tb,
  language=pascal,
  aboveskip=1mm,
  belowskip=1mm,
  showstringspaces=false,
  columns=fullflexible,
  keepspaces=true,
  basicstyle={\small\ttfamily},
  numbers=none,
  numberstyle=\tiny\color{gray},
  keywordstyle=\color{blue},
  commentstyle=\color{dkgreen},
  stringstyle=\color{mauve},
  breaklines=true,
  breakatwhitespace=true,
  autogobble=true
}

\title{Perkenalan}
\author{Tim Olimpiade Komputer Indonesia}
\date{}

\begin{document}

\begin{frame}
\titlepage
\end{frame}

\begin{frame}
\frametitle{Perkenalan}
Selamat datang di topik \newTerm{Pemrograman Kompetitif Dasar}!
\newline
\newline
Anda diharapkan telah menguasai pemrograman dasar dan mampu:
\begin{itemize}
  \item Mengetahui setidaknya satu bahasa pemrograman.
  \item Mampu membaca dan menulis program.
  \item Mampu memahami \newTerm{pseudocode}.
\end{itemize}
\end{frame}

\begin{frame}
\frametitle{Pseudocode}
\begin{itemize}
  \item Merupakan bahasa informal serupa dengan bahasa pemrograman untuk mendeskripsikan program.
  \item Biasa digunakan pada materi pembelajaran algoritma.
  \item Pseudocode sendiri bukanlah bahasa pemrograman sungguhan.
\end{itemize}
\end{frame}


\begin{frame}
\frametitle{Contoh Pseudocode}
\begin{codebox}
\Procname{$\proc{insertionSort}(A)$}
\li \For $i \gets 2$ \To $\attrib{A}{length}$
    \Do
\li   $j \gets i$
\li   \While $(j > 1)$ and $(A[j] < A[j-1])$
      \Do
\li     $\proc{swap}(A[j], A[j-1])$
\li     $j \gets j-1$
      \End
    \End
\end{codebox}

\begin{itemize}
  \item Memahami pseudocode pada awalnya mungkin sulit.
  \item Seiring berjalannya waktu, Anda akan terbiasa dan memahami betapa mudahnya menggunakan pseudocode.
\end{itemize}
\end{frame}

\begin{frame}
\frametitle{Tentang Pemrograman Kompetitif}
\begin{block}{Pemrograman Kompetitif}
\foreignTerm{Competitive programming is solving well-defined problems by writing computer programs under specified limits.}

-Ashar Fuadi-
\end{block}
\end{frame}

\begin{frame}
\frametitle{Tentang Pemrograman Kompetitif (lanj.)}
\begin{itemize}
  \item Pemrograman kompetitif sering dijadikan ajang "adu otak" dan asah kemampuan \foreignTerm{problem solving}.
  \item Peserta ditantang untuk:
  \begin{itemize}
    \item Menganalisa permasalahan
    \item Merancang algoritma solusi
    \item Menuliskannya dalam bentuk program
  \end{itemize}
\end{itemize}
\end{frame}

\begin{frame}
\frametitle{Ajang Pemrograman Kompetitif - Nasional}
\begin{itemize}
  \item Di Indonesia, pemrograman kompetitif menjadi konsep dalam Olimpiade Sains Nasional (OSN) bidang komputer/informatika.
  \item Selain OSN, terdapat kompetisi tingkat nasional yang diselenggarakan beberapa Universitas di Indonesia, seperti CompFest (UI), NPC (ITS), BNPCHS (Binus), dan ILPC (Ubaya).
  \item Sebagai sarana berlatih, ada juga TOKI Open Contest yang biasa dilaksanakan per bulan.
\end{itemize}
\end{frame}

\begin{frame}
\frametitle{Ajang Pemrograman Kompetitif - Internasional}
\begin{itemize}
  \item International Olympiad in Informatics (IOI) merupakan kompetisi bagi siswa SMA dari seluruh dunia.
  \item Untuk mahasiswa, terdapat ACM-ICPC (ACM International Collegiate Programming Contest) yang pesertanya terdiri dari tim-tim beranggotakan tiga orang.
\end{itemize}
\end{frame}

\begin{frame}
\frametitle{Ajang Pemrograman Kompetitif - Lainnya}
\begin{itemize}
  \item Terdapat pula kompetisi tingkat regional yang diselenggarakan bagi negara-negara dalam suatu bagian, seperti Asia-Pacific Informatics Olympiad (APIO).
  \item Untuk sekedar hobi dan latihan rutin, Anda dapat mengikuti \textcolor{blue}{\href{http://codeforces.com/}{Codeforces}}, \textcolor{blue}{\href{https://www.topcoder.com/}{Topcoder}}, dan \textcolor{blue}{\href{https://www.codechef.com/}{Codechef}}.
\end{itemize}
\end{frame}

\begin{frame}
\frametitle{Contoh Pemrograman Kompetitif}
\begin{itemize}
  \item Masalah yang diberikan adalah \foreignTerm{well-defined problems}.
  \item \foreignTerm{Well-defined problem} adalah sebuah masalah yang telah terdefinsi dengan baik, seperti asumsi yang diperlukan dan batasan masalah.
  \item Solusi atas masalah ditulis dalam bentuk program program dan memenuhi batas-batas yang ditentukan.
  \item Batas yang ditentukan: waktu, memori, dan lainnya.
\end{itemize}
\end{frame}

\begin{frame}
\frametitle{Contoh Soal Pemrograman Kompetitif}
\begin{itemize}
  \item Terdapat $N$ buah ruangan yang dinomori dari 1 sampai $N$
  \item Ruangan ke-$i$ memiliki sebuah lampu dan sebuah tombol.
  \item Bila tombol itu ditekan, keadaan lampu pada seluruh ruangan ke-$x$ akan berubah (dari mati menjadi menyala, atau sebaliknya), yang mana $x$ habis dibagi $i$.
  \item Contoh, bila $N$ = 10 dan tombol di ruangan ke-2 ditekan, maka keadaan lampu pada ruangan 2, 4, 6, 8, dan 10 akan berubah.
  \item Bila pada awalnya seluruh lampu berada pada keadaan mati, dan tombol pada setiap ruangan ditekan tepat sekali, bagaimanakah keadaan lampu pada ruangan ke-$N$?
\end{itemize}
\end{frame}

\begin{frame}
\frametitle{Contoh Soal Pemrograman Kompetitif (lanj.)}
\begin{itemize}
  \item Batas waktu: 1 detik.
  \item Batas memori: 32 MB.
  \item Diketahui $1 \leq N \leq 10^{14}$.
\end{itemize}
\end{frame}

\begin{frame}
\frametitle{Solusi Naif}
Salah satu cara penyelesaiannya adalah dengan mensimulasikan skenario pada deskripsi soal:
\begin{itemize}
  \item Mulai dengan ruangan ke-1, dipastikan keadaan lampu pada ruangan ke-$N$ akan berubah ($N$ habis dibagi 1).
  \item Lanjut ke ruangan ke-2, periksa apakah 2 habis membagi $N$. Bila ya, ubah keadaan lampunya.
  \item Lanjut ke ruangan ke-3, periksa apakah 3 habis membagi $N$. Bila ya, ubah keadaan lampunya.
  \item ... dan seterusnya sampai ruangan ke-$N$.
\end{itemize}
\end{frame}

\begin{frame}
\frametitle{Solusi Naif (lanj.)}
\begin{itemize}
  \item Setelah selesai disimulasikan, tinggal keadaan lampu ruangan ke-N dan cetak jawabannya.
  \item Kompleksitas solusi ini adalah $O(N)$, dan hanya akan bekerja untuk nilai $N$ yang kecil.
  \item Untuk $N$ yang lebih besar, misalnya $N = 10^9$, kemungkinan besar diperlukan waktu lebih dari 1 detik.
\end{itemize}
\end{frame}

\begin{frame}
\frametitle{Solusi yang Lebih Baik}
\begin{itemize}
  \item Dengan observasi, yang sebenarnya perlu dilakukan adalah memeriksa banyaknya pembagi dari $N$.
  \item Apabila banyaknya pembagi ganjil, berarti pada akhirnya lampu di ruangan ke-$N$ akan menyala.
  \item Bila genap, berarti lampu di ruangan ke-$N$ akan tetap mati.
\end{itemize}
\end{frame}

\begin{frame}
\frametitle{Solusi yang Lebih Baik (lanj.)}
\begin{itemize}
  \item Untuk mencari banyaknya pembagi dari N dengan lebih efisien, lakukan faktorisasi prima terlebih dahulu.
  \item Misalkan $N$ = 12, maka faktorisasi primanya adalah $2^2 \times 3$.
  \newline
  \item Untuk menjadi pembagi dari 12, suatu bilangan hanya boleh:
  \begin{itemize}
    \item Memiliki faktor 2 maksimal sebanyak 2.
    \item Memiliki faktor 3 maksimal sebanyak 1.
    \item Tidak boleh memiliki faktor lainnya.
  \end{itemize}
\end{itemize}
\end{frame}

\begin{frame}
\frametitle{Solusi yang Lebih Baik (lanj.)}
Sebagai contoh, berikut daftar seluruh pembagi dari 12:
\begin{itemize}
  \item $1 = 2^0 \times 3^0$
  \item $2 = 2^1 \times 3^0$
  \item $3 = 2^0 \times 3^1$
  \item $4 = 2^2 \times 3^0$
  \item $6 = 2^1 \times 3^1$
  \item $12 = 2^2 \times 3^1$
\end{itemize}
\end{frame}

\begin{frame}
\frametitle{Solusi yang Lebih Baik (lanj.)}
\begin{itemize}
  \item Banyaknya pembagi dari 12 sebenarnya sama saja dengan banyaknya kombinasi yang bisa dipilih dari {$2^0$, $2^1$, $2^2$} dan {$3^0$, $3^1$}.
  \item Banyaknya kombinasi sama dengan mengkalikan banyaknya elemen pada tiap-tiap himpunan.
  \item Sehingga banyaknya cara ada $3 \times 2 = 6$ cara.
\end{itemize}
\end{frame}

\begin{frame}
\frametitle{Solusi yang Lebih Baik (lanj.)}
\begin{itemize}
  \item Cara ini juga berlaku untuk nilai $N$ yang lain.
  \item Misalnya $N = 172.872 = 2^3 \times 3^2 \times 7^4$.
  \item Berarti banyak pembaginya adalah $4 \times 3 \times 5 = 60$.
\end{itemize}
\end{frame}

\begin{frame}
\frametitle{Solusi yang Lebih Baik (lanj.)}
\begin{itemize}
  \item Secara umum, banyaknya pembagi dari:
  \newline
  \begin{center}
    $N = a_1^{p_1} \times a_2^{p_2} \times a_3^{p_3} \times ... \times a_k^{p_k}$
  \end{center}
  adalah:
  \newline
  \begin{center}
    $(1+p_1) \times (1+p_2) \times (1+p_3) \times ... \times (1+p_k)$
    \newline
  \end{center}
  \item Jadi cukup faktorkan $N$, lalu periksa banyak pembaginya.
  \item Faktorisasi bilangan bisa diimplementasikan dengan efisien dan 1 detik cukup untuk $N$ sampai $10^{14}$.
\end{itemize}
\end{frame}

\begin{frame}
\frametitle{Manfaat Pemrograman Kompetitif}
\begin{itemize}
  \item Mengasah kemampuan menganalisa permasalahan dan pemecahan masalah.
  \item Bertemu dengan teman-teman sehobi!
  \item Kadang, soal \foreignTerm{interview} untuk masuk ke perusahaan teknologi terkemuka juga membutuhkan kemampuan \foreignTerm{problem solving}.
\end{itemize}
\end{frame}

\begin{frame}
\frametitle{Dan Tentunya...}
\begin{center}
  {\LARGE Menantang dan menyenangkan!}
\end{center}
\end{frame}

\begin{frame}
\frametitle{Latihan}
\begin{itemize}
  \item Sebagai pemanasan, silakan mengerjakan soal latihan yang diberikan.
  \item Baca juga beberapa materi pengayaan yang diberikan.
\end{itemize}
\end{frame}

\end{document}
