\documentclass{beamer}
\usetheme{tokitex}

\usepackage{graphics}
\usepackage{multirow}
\usepackage{tabto}

\usepackage[english,bahasa]{babel}
\newtranslation[to=bahasa]{Section}{Bagian}
\newtranslation[to=bahasa]{Subsection}{Subbagian}

\usepackage{listings, lstautogobble}
\usepackage{color}

\definecolor{dkgreen}{rgb}{0,0.6,0}
\definecolor{gray}{rgb}{0.5,0.5,0.5}
\definecolor{mauve}{rgb}{0.58,0,0.82}

\lstset{frame=tb,
  language=pascal,
  aboveskip=1mm,
  belowskip=1mm,
  showstringspaces=false,
  columns=fullflexible,
  keepspaces=true,
  basicstyle={\small\ttfamily},
  numbers=none,
  numberstyle=\tiny\color{gray},
  keywordstyle=\color{blue},
  commentstyle=\color{dkgreen},
  stringstyle=\color{mauve},
  breaklines=true,
  breakatwhitespace=true,
  autogobble=true
}

\title{Dynamic Programming}
\author{Tim Olimpiade Komputer Indonesia}
\date{}

\usepackage{qtree}
\begin{document}

\begin{frame}
\titlepage
\end{frame}

\begin{frame}
\frametitle{Pendahuluan}
Melalui dokumen ini, kalian akan:
\begin{itemize}
  \item Memahami konsep \foreignTerm{Dynamic Programming} (DP).
  \item Menyelesaikan contoh persoalan DP sederhana.
\end{itemize}
\end{frame}

\begin{frame}
\frametitle{Motivasi}
\begin{itemize}
  \item Diberikan $M$ jenis koin, masing-masing jenis bernilai $a_1, a_2, ..., a_M$ rupiah.
  \item Asumsikan terdapat tak hingga koin untuk setiap nominal koin yang ada. 
  \item Tentukan berapa banyaknya minimum koin untuk membayar sebesar $N$ rupiah!
\end{itemize}
\end{frame}

\begin{frame}
\frametitle{Solusi Greedy}
\begin{itemize}
  \item Mari kita coba menyelesaikan masalah ini secara \fGreedy.
  \item Salah satu algoritma \fGreedy yang mungkin adalah dengan menggunakan koin terbesar yang $\leq$ sisa uang yang harus dibayar.
\end{itemize}
\end{frame}

\begin{frame}
\frametitle{Solusi Greedy (lanj.)}
\begin{itemize}
  \item Misalkan kita memiliki nominal koin 1 rupiah, 6 rupiah, dan 10 rupiah dan ingin membayar 12 rupiah.
  \item Dengan algoritma sebelumnya, kita akan menggunakan koin 10 rupiah terlebih dahulu. 
  \item Karena tersisa 2 rupiah, berikutnya kita akan menggunakan 2 koin 1 rupiah, sehingga totalnya kita menggunakan 3 koin.
  \item Namun, ada solusi lebih baik: 2 koin 6 rupiah.
  \item Algoritma \fGreedy ini tidak memberikan solusi terbaik.
\end{itemize}
\end{frame}

\begin{frame}
\frametitle{Observasi}
\begin{itemize}
  \item Untuk membayar $N$ rupiah, kita dapat memilih salah satu koin terlebih dahulu.
  \item Jika nilai koin itu adalah $a_k$, maka sisa uang uang perlu kita bayar adalah $N-a_k$.
  \item Dalam kasus ini, terdapat $M$ pilihan koin untuk $a_k$.
\end{itemize}
\end{frame}

\begin{frame}
\frametitle{Observasi (lanj.)}
\begin{itemize}
  \item Perhatikan bahwa penukaran $N - a_k$ merupakan suatu sub-persoalan yang serupa dengan persoalan awalnya.
  \item Artinya, cara yang sama untuk menyelesaikan sub-persoalan dapat digunakan.
  \item Kita akan menggunakan strategi penyelesaian secara rekursif.
\end{itemize}
\end{frame}

\begin{frame}
\frametitle{Solusi Rekursif}
\begin{itemize}
  \item Definisikan sebuah fungsi $f(x)$ sebagai banyaknya koin minimum yang dibutuhkan untuk membayar $x$ rupiah.
  \item Kita dapat mencoba-coba koin yang ingin kita gunakan.
  \item Jika suatu koin $a_k$ digunakan, maka kita membutuhkan $f(x-a_k)$ koin ditambah satu koin $a_k$.
  \item Atau dapat ditulis $f(x) = f(x-a_k) + 1$
  \item Pencarian nilai $f(x-a_k)$ dilakukan secara rekursif, kita kembali mencoba-coba koin yang ingin digunakan.
\end{itemize}
\end{frame}

\begin{frame}
\frametitle{Solusi Rekursif (lanj.)}
\begin{itemize}
  \item Dari semua kemungkinan $a_k$, mana pilihan yang terbaik?
  \item Pilihan yang terbaik akan memberikan nilai $f(x - a_k) + 1$ sekecil mungkin.
  \item Jadi kita cukup mencoba semua kemungkinan $a_k$, dan ambil yang hasil $f(x - a_k) + 1$ terkecil.
\end{itemize}
\end{frame}

\begin{frame}
\frametitle{Solusi Rekursif (lanj.)}
\begin{itemize}
  \item Jika $f(x)$ dihitung secara rekursif, apa yang menjadi \fbasecase?
  \item Kasus terkecilnya adalah $f(0)$, yang artinya kita hendak membayar 0 rupiah.
  \item Membayar 0 rupiah tidak membutuhkan satu pun koin, sehinga $f(0) = 0$.
\end{itemize}
\end{frame}

\begin{frame}
\frametitle{Solusi rekursif}
Secara matematis, hubungan rekursif ini dituliskan:
\[f(x) = \left\{\begin{array}{lr}
    0, & n = 0\\
    \min_{1 \leq k \leq M, a_k \leq x} {f(x - a_k) + 1}, & n > 0\\
    \end{array}\right. \]
\end{frame}

\begin{frame}
\frametitle{Implementasi Solusi Rekursif}
Kita implementasikan $f(x)$ sebagai fungsi $\proc{solve}(x)$:
\begin{codebox}
\Procname{$\proc{solve}(x)$}
\li \If $(x \isequal 0)$ \Then
\li   \Return $0$
\li \Else
\li   $best \gets \infty$
\li   \For $k \gets 1$ \To $M$ \Do
\li     \If $(a_k \leq x)$ \Then
\li       $best \gets \min(best, \proc{solve}(x - a_k) + 1)$
        \End
      \End
\li   \Return $best$
    \End
\end{codebox}

Jawaban akhirnya ada pada $\proc{solve}(N)$.
\end{frame}

\begin{frame} [fragile]
\frametitle{Solusi rekursif (lanj.)}
Mari kita lihat pohon rekursi yang dihasilkan oleh fungsi $f$. 

Berikut untuk $f(12)$ dengan nominal koin 1, 6, dan 10 rupiah.

\begin{center}
\scalebox{0.9}{
\Tree [.$f(12)$
  [.$f(2)$
    [.$f(1)$
      [.$f(0)$ ]
    ]
  ]
  [.$f(6)$
    [.$f(0)$ ]
    [.$f(5)$
      [.$f(4)$
        [.$...$
        ]
      ]
    ]
  ]
  [.$f(11)$
    [.$f(1)$
      [.$f(0)$ ]
    ]
    [.$f(5)$
      [.$f(4)$
        [.$...$          
        ]
      ]
    ]
    [.$f(10)$ 
      [.$f(0)$ 
        [.$...$ ]      
      ]
      [.$f(4)$ 
        [.$...$ ]      
      ]
      [.$f(9)$ 
        [.$...$ ]      
      ]
    ]
  ]
]
}
\end{center}
\end{frame}

\begin{frame}
\frametitle{Solusi rekursif (lanj.)}
\begin{itemize}
  \item Jika diperhatikan pada pohon rekursi, terdapat $O(M)$ cabang untuk setiap pemanggilan $f$.
  \item Untuk menghitung nilai $f(N)$, kita akan memiliki pohon rekursi yang kira-kira sedalam $O(N)$.
  \item Berarti kira-kira dilakukan $O(M^N)$ pemanggilan fungsi.
  \item Karena itu, solusi ini membutuhkan $O(M^N)$ operasi, yang mana banyaknya operasi ini eksponensial.
\end{itemize}
\end{frame}

\begin{frame}
\frametitle{Solusi rekursif (lanj.)}
\begin{itemize}
  \item Biasanya solusi eksponensial berjalan sangat lambat.
  \item Cobalah Anda hitung nilai $O(M^N)$ dengan $M=3$ dan $N=100$, untuk menyadari betapa lambatnya solusi ini!
  \item Kita tidak ingin memiliki solusi eksponensial pada pemrograman kompetitif, kecuali pada soal-soal tertentu  yang tidak memiliki solusi polinomial. 
  \item Karena itu, kita harus melakukan sebuah optimisasi.
\end{itemize}
\end{frame}

\begin{frame}
\frametitle{Optimisasi}
Jika diperhatikan, ternyata banyak $f(x)$ yang dihitung berkali-kali. Sebagai contoh, $f(5)$ dan $f(4)$.
\begin{center}
\scalebox{0.8}{
\Tree [.$f(12)$
  [.$f(2)$
    [.$f(1)$
      [.$f(0)$ ]
    ]
  ]
  [.$f(6)$
    [.$f(0)$ ]
    [.$f(5)$
      [.$f(4)$
        [.$...$
        ]
      ]
    ]
  ]
  [.$f(11)$
    [.$f(1)$
      [.$f(0)$ ]
    ]
    [.$f(5)$
      [.$f(4)$
        [.$...$          
        ]
      ]
    ]
    [.$f(10)$ 
      [.$f(0)$ 
        [.$...$ ]      
      ]
      [.$f(4)$ 
        [.$...$ ]      
      ]
      [.$f(9)$ 
        [.$...$ ]      
      ]
    ]
  ]
]
}
\end{center}
\end{frame}

\begin{frame}
\frametitle{Optimisasi (lanj.)}
\begin{itemize}
  \item Perhatikan bahwa hanya ada $N + 1$ kemungkinan $x$ untuk $f(x)$, yaitu $0$ sampai $N$.
  
  \item Kita dapat melakukan \newTerm{memoisasi}, yaitu mencatat hasil perhitungan $f(x)$ setelah menghitungnya.
  \item Jika suatu ketika kita kembali memerlukan nilai $f(x)$, kita tidak perlu menghitungnya kembali.
\end{itemize}
\end{frame}

\begin{frame}
\frametitle{Solusi Rekursif dengan Memoisasi}
\begin{codebox}
\Procname{$\proc{solve}(x)$}
\li \If $(x \isequal 0)$ \Then
\li   \Return $0$
\li \ElseIf $hasComputed[x]$ \Then
\li   \Comment Langsung kembalikan
\li   \Return $memo[x]$ 
\li \Else
\li   $best \gets \infty$
\li   \For $k \gets 1$ \To $M$ \Do
\li     \If $(a_k \leq x)$ \Then
\li       $best \gets \min(best, \proc{solve}(x - a_k) + 1)$
        \End
      \End
\li   \Comment Catat hasilnya      
\li   $hasComputed[x] \gets true$
\li   $memo[x] \gets best$
\li   \Return $best$
    \End
\end{codebox}
\end{frame}

\begin{frame}
\frametitle{Solusi Rekursif dengan Memoisasi (lanj.)}
\begin{itemize}
  \item Untuk menghitung suatu nilai $f(x)$, kita membutuhkan $O(M)$ iterasi.
  \item Sehingga untuk menghitung nilai $f(x)$ untuk seluruh $x$, kita membutuhkan $O(NM)$ operasi.
  \item Banyaknya operasi ini polinomial terhadap $N$ dan $M$, dan \\ \emp{jauh lebih cepat} daripada solusi rekursif sebelumnya.
\end{itemize}
\end{frame}

\begin{frame} 
\frametitle{Dynamic Programming}
\begin{itemize}
  \item Merupakan metode penyelesaian persoalan yang melibatkan pengambilan keputusan dengan memanfaatkan informasi dari penyelesaian sub-persoalan yang sama namun lebih kecil.
  \item Solusi sub-persoalan tersebut hanya dihitung satu kali dan disimpan di memori.
  \item Jika sebuah persoalan adalah masalah optimisasi, maka biasanya kita mencoba semua kemungkinan solusi sub-problem yang dihasilkan, dan  mengambil yang hasilnya paling optimal. 
\end{itemize}
\end{frame}

\begin{frame} 
\frametitle{Dynamic Programming}
Terdapat dua cara mengimplementasikan DP
\begin{itemize}
  \item \newTerm{\fTopdown}: diimplementasikan secara rekursif sambil mencatat nilai yang sudah ditemukan (memoisasi).
  \item \newTerm{\fBottomup}: diimplementasikan secara iteratif dengan menghitung mulai dari kasus yang kecil ke besar.
  \newline
\end{itemize}
\end{frame}

\begin{frame} 
\frametitle{Top Down}
\begin{itemize}
  \item Cara yang sebelumnya kita gunakan adalah \ftopdown.
  \item Kata memoisasi berasal dari "memo", yang artinya catatan.
  \item Pada \ftopdown, penyelesaian masalah dimulai dari kasus yang besar.
  \item Untuk menyelesaikan kasus yang besar, dibutuhkan solusi dari kasus yang lebih kecil.
  \item Karena solusi kasus yang lebih kecil belum ada, maka kita akan mencarinya terlebih dahulu, lalu mencatat hasilnya.
  \item Hal ini dilakukan secara rekursif.
\end{itemize}
\end{frame}

\begin{frame} 
\frametitle{Bottom Up}
\begin{itemize}
  \item Pada \fbottomup, penyelesaian masalah dimulai dari kasus yang kecil.
  \item Ketika merumuskan formula rekursif, kita mengetahui jawaban kasus yang paling kecil, yaitu \fbasecase.
  \item Informasi ini digunakan untuk menyelesaikan kasus yang \\ lebih besar.
  \item Biasanya dianalogikan dengan pengisian "tabel DP".
  \item Hal ini dilakukan secara iteratif.
\end{itemize}
\end{frame}

\begin{frame}
\frametitle{Solusi Coin Change dengan Bottom Up}
Secara \fbottomup, kita hitung semua nilai $f(x)$ untuk \newline semua nilai $x$ dari $0$ sampai $N$ secara menaik.

Nilai dari $f(x)$ disimpan dalam \farray $f[x]$.
\begin{codebox}
\Procname{$\proc{solve}()$}
\li $f[0] \gets 0$
\li \For $x \gets 1$ \To $N$ \Do
\li   $best \gets \infty$
\li   \For $k \gets 1$ \To $M$ \Do
\li     \If $(a_k \leq x)$ \Then
\li       $best \gets \min(best, f[x - a_k] + 1)$
        \End
      \End
\li   $f[x] \gets best$
    \End
\li \Return $f[N]$
\end{codebox}
\end{frame}

\begin{frame} 
\frametitle{Kompleksitas?}
\begin{itemize}
  \item Dengan mudah Anda dapat memperhatikan bahwa kompleksitas solusi dengan \fbottomup adalah $O(NM)$.
  \item Kompleksitas ini sama seperti dengan cara \ftopdown.
  \item Kenyataannya, sebenarnya keduanya merupakan algoritma \emp{yang sama}, hanya berbeda di arah pencarian jawaban.
\end{itemize}
\end{frame}

\begin{frame}
\frametitle{Mengisi "Tabel"}
Cara \fbottomup yang dijelaskan sebelumnya terkesan seperti "mengisi tabel".\newline

\begin{tabular}{|c|c|c|c|c|c|c|c|c|c|c|c|c|c|}
\hline $x$ & 0 & 1 & 2 & 3 & 4 & 5 & 6 & 7 & 8 & 9 & 10 & 11 & 12 \\ 
\hline $f(x)$ &  &  &  &  &  &  &  &  &  &  & & & \\ 
\hline 
\end{tabular} 
\end{frame}

\begin{frame}
\frametitle{Mengisi "Tabel" (lanj.)}
Awalnya, diisi $f(0) \gets 0$.\newline

\begin{tabular}{|c|c|c|c|c|c|c|c|c|c|c|c|c|c|}
\hline $x$ & 0 & 1 & 2 & 3 & 4 & 5 & 6 & 7 & 8 & 9 & 10 & 11 & 12 \\ 
\hline $f(x)$ &  0 &  &  &  &  &  &  &  &  &  & & & \\ 
\hline 
\end{tabular} 
\end{frame}

\begin{frame}
\frametitle{Mengisi "Tabel" (lanj.)}
Berikutnya, diisi $f(1)$.

Satu-satunya pilihan adalah menukarkan dengan koin 1, karena kita tidak bisa menggunakan koin 6 atau 10. Jadi:

$f(1) = f(1-1) + 1 = f(0) + 1$
\newline

\begin{tabular}{|c|c|c|c|c|c|c|c|c|c|c|c|c|c|}
\hline $x$ & 0 & 1 & 2 & 3 & 4 & 5 & 6 & 7 & 8 & 9 & 10 & 11 & 12 \\ 
\hline $f(x)$ &  0 & 1 &  &  &  &  &  &  &  &  & & & \\ 
\hline 
\end{tabular} 
\newline \newline \newline Masuk akal, untuk membayar 1 memang kita membutuhkan 1 koin.
\end{frame}

\begin{frame}
\frametitle{Mengisi "Tabel" (lanj.)}
Hal serupa terjadi ketika kita mengisi $f(2), f(3), f(4), dan f(5)$.

Satu-satunya pilihan adalah menukarkan dengan koin 1, karena kita tidak bisa menggunakan koin 6 atau 10.\newline

\begin{tabular}{|c|c|c|c|c|c|c|c|c|c|c|c|c|c|}
\hline $x$ & 0 & 1 & 2 & 3 & 4 & 5 & 6 & 7 & 8 & 9 & 10 & 11 & 12 \\ 
\hline $f(x)$ &  0 & 1 & 2 & 3 & 4 & 5 &  &  &  &  & & & \\ 
\hline 
\end{tabular} 
\end{frame}

\begin{frame}
\frametitle{Mengisi "Tabel" (lanj.)}
Berikutnya adalah mengisi $f(6)$.

Terdapat pilihan untuk menggunakan koin 1 atau 6 terlebih dahulu, sehingga:
\begin{align*}
  f(6) &= \min(f(6-1) + 1, f(6-6) + 1) \\
  &= \min(f(5) + 1, f(0) + 1) \\
  &= \min(5 + 1, 0 + 1) \\
  &= \min(6, 1) \\
  &= 1
\end{align*}
\end{frame}

\begin{frame}
\frametitle{Mengisi "Tabel" (lanj.)}
Memang benar, untuk membayar 6 kita hanya membutuhkan 1 koin.\newline

\begin{tabular}{|c|c|c|c|c|c|c|c|c|c|c|c|c|c|}
\hline $x$ & 0 & 1 & 2 & 3 & 4 & 5 & 6 & 7 & 8 & 9 & 10 & 11 & 12 \\ 
\hline $f(x)$ &  0 & 1 & 2 & 3 & 4 & 5 & 1 &  &  &  & & & \\ 
\hline 
\end{tabular} 
\end{frame}

\begin{frame}
\frametitle{Mengisi "Tabel" (lanj.)}
Lakukan hal serupa untuk $x = 7$ sampai $x = 9$.\newline

\begin{tabular}{|c|c|c|c|c|c|c|c|c|c|c|c|c|c|}
\hline $x$ & 0 & 1 & 2 & 3 & 4 & 5 & 6 & 7 & 8 & 9 & 10 & 11 & 12 \\ 
\hline $f(x)$ &  0 & 1 & 2 & 3 & 4 & 5 & 1 & 2 & 3 & 4 & & & \\ 
\hline 
\end{tabular} 
\end{frame}

\begin{frame}
\frametitle{Mengisi "Tabel" (lanj.)}
Berikutnya adalah mengisi $f(10)$.

Terdapat pilihan untuk menggunakan koin 1, 6, atau 10 terlebih dahulu, sehingga:
\begin{align*}
  f(10) &= \min(f(10-1) + 1, \ f(10-6) + 1, \ f(10-10) + 1) \\
  &= \min(f(9) + 1, \ f(4) + 1, \ f(0) + 1) \\
  &= \min(4 + 1, \ 4 + 1, \ 0 + 1) \\
  &= \min(5, 5, 1) \\
  &= 1
\end{align*}
\end{frame}

\begin{frame}
\frametitle{Mengisi "Tabel" (lanj.)}
Kembali, memang benar bahwa untuk membayar 10 kita hanya membutuhkan 1 koin, yaitu langsung menggunakan koin 10 \\ (tanpa 1 dan 6).\newline

\begin{tabular}{|c|c|c|c|c|c|c|c|c|c|c|c|c|c|}
\hline $x$ & 0 & 1 & 2 & 3 & 4 & 5 & 6 & 7 & 8 & 9 & 10 & 11 & 12 \\ 
\hline $f(x)$ &  0 & 1 & 2 & 3 & 4 & 5 & 1 & 2 & 3 & 4 & 1 & & \\ 
\hline 
\end{tabular} 
\end{frame}

\begin{frame}
\frametitle{Mengisi "Tabel" (lanj.)}
Berikutnya adalah mengisi $f(11)$.

\begin{align*}
  f(11) &= \min(f(11-1) + 1, \ f(11-6) + 1, \ f(11-10) + 1) \\
  &= \min(f(10) + 1, \ f(5) + 1, \ f(1) + 1) \\
  &= \min(1 + 1, \ 5 + 1, \ 1 + 1) \\
  &= \min(2, 6, 2) \\
  &= 2
\end{align*}
\begin{tabular}{|c|c|c|c|c|c|c|c|c|c|c|c|c|c|}
\hline $x$ & 0 & 1 & 2 & 3 & 4 & 5 & 6 & 7 & 8 & 9 & 10 & 11 & 12 \\ 
\hline $f(x)$ &  0 & 1 & 2 & 3 & 4 & 5 & 1 & 2 & 3 & 4 & 1 & 2 & \\ 
\hline 
\end{tabular} 
\end{frame}

\begin{frame}
\frametitle{Mengisi "Tabel" (lanj.)}
Terakhir, isi $f(12)$.
\begin{align*}
  f(12) &= \min(f(12-1) + 1, \ f(12-6) + 1, \ f(12-10) + 1) \\
  &= \min(f(11) + 1, \ f(6) + 1, \ f(2) + 1) \\
  &= \min(2 + 1, \ 1 + 1, \ 2 + 1) \\
  &= \min(3, 2, 3) \\
  &= 2
\end{align*}

Dari sini, terlihat bahwa menggunakan koin 10 terlebih dahulu (pilihan paling kanan) mengakibatkan banyaknya koin yang dibutuhkan adalah 3.\newline

Sementara menggunakan koin 6 terlebih dahulu (pilihan tengah) mengakibatkan banyaknya koin yang dibutuhkan adalah 2.
\end{frame}

\begin{frame}
\frametitle{Mengisi "Tabel" (lanj.)}
Jadi kita selesai mengisi "tabel" DP.\newline

\begin{tabular}{|c|c|c|c|c|c|c|c|c|c|c|c|c|c|}
\hline $x$ & 0 & 1 & 2 & 3 & 4 & 5 & 6 & 7 & 8 & 9 & 10 & 11 & 12 \\ 
\hline $f(x)$ &  0 & 1 & 2 & 3 & 4 & 5 & 1 & 2 & 3 & 4 & 1 & 2 & 2 \\ 
\hline 
\end{tabular}
\newline \newline
\begin{itemize}
  \item Jika Anda menggunakan \ftopdown, pada akhirnya tabel $memo$ juga akan berisi nilai-nilai ini.
  \item Coba implementasikan secara \ftopdown dan \fbottomup dan lihat hasilnya!
\end{itemize}
\end{frame}

\begin{frame}
\frametitle{Top Down dan Bottom Up}
Top Down
\begin{itemize}
  \item Sebuah transformasi natural dari formula rekursif, biasanya mudah diimplementasikan.
  \item Urutan pengisian tabel tidak penting.
  \item Hanya menghitung nilai dari fungsi jika hanya diperlukan.
  \item Ketika seluruh tabel memo pada akhirnya terisi, bisa saja lebih lambat karena adanya 
  \foverhead pemanggilan fungsi.
\end{itemize}
\end{frame}

\begin{frame}
\frametitle{Top Down dan Bottom Up (lanj.)}
Bottom Up
\begin{itemize}
  \item Tidak mengalami perlambatan dari \foverhead pemanggilan fungsi.
  \item Memungkinkan penggunaan teknik DP lanjutan seperti \foreignTerm{flying table}, kombinasi dengan struktur data \foreignTerm{tree}, dsb.
  \item Harus memikirkan urutan pengisian nilai tabel.
  \item Semua tabel harus diisi nilainya walaupun tidak dibutuhkan akhirnya.
\end{itemize}
\end{frame}

\begin{frame}
\frametitle{Top Down dan Bottom Up (lanj.)}
\begin{itemize}
  \item Beberapa orang lebih alami untuk menggunakan \ftopdown, sementara sisanya lebih terbiasa dengan \fbottomup.
  \item Bergantung dari cara berpikir Anda, salah satunya mungkin lebih mudah Anda pelajari.
  \item Untuk orang yang telah berpengalaman, penggunaan \fbottomup dan \ftopdown dapat disesuaikan dengan soal yang dihadapi.
\end{itemize}
\end{frame}

\begin{frame}
\frametitle{Penutup}
\begin{itemize}
  \item Terdapat dua versi DP, yaitu \ftopdown dan \fbottomup.
  \item Keduanya memiliki keuntungan dan kerugian, pilih yang tepat sesuai dengan kebutuhan soal.
  \item Kunci dari mengerjakan soal DP adalah mengidentifikasi pilihan keputusan yang bisa diambil, dan merumuskannya menjadi rumus rekursif.
  \item Anda perlu banyak latihan soal DP untuk menjadi terbiasa dengan melakukan formulasi rekursif ini.
\end{itemize}
\end{frame}

\end{document}
