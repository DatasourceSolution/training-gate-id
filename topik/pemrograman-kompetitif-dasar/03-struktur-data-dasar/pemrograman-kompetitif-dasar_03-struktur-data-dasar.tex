\documentclass{beamer}
\usetheme{tokitex}

\usepackage{graphics}
\usepackage{multirow}
\usepackage{tabto}

\usepackage[english,bahasa]{babel}
\newtranslation[to=bahasa]{Section}{Bagian}
\newtranslation[to=bahasa]{Subsection}{Subbagian}

\usepackage{listings, lstautogobble}
\usepackage{color}

\definecolor{dkgreen}{rgb}{0,0.6,0}
\definecolor{gray}{rgb}{0.5,0.5,0.5}
\definecolor{mauve}{rgb}{0.58,0,0.82}

\lstset{frame=tb,
  language=pascal,
  aboveskip=1mm,
  belowskip=1mm,
  showstringspaces=false,
  columns=fullflexible,
  keepspaces=true,
  basicstyle={\small\ttfamily},
  numbers=none,
  numberstyle=\tiny\color{gray},
  keywordstyle=\color{blue},
  commentstyle=\color{dkgreen},
  stringstyle=\color{mauve},
  breaklines=true,
  breakatwhitespace=true,
  autogobble=true
}

\title{Struktur Data Dasar}
\author{Tim Olimpiade Komputer Indonesia}
\date{}

\begin{document}

\begin{frame}
\titlepage
\end{frame}

\begin{frame}
\frametitle{Panduan Konten}
\begin{itemize}
  \item Pengertian struktur data \& apa pentingnya
  \item Review: tunjukkan bahwa array adalah contoh struktur data sederhana yang mendukung operasi read \& update secara random access
  \item Bahas linked list, tekankan apa keuntungan/kerugian dari linked list dibandingkan dengan array
  \item Bahas stack dan queue, bisa dimulai dengan diberikan permasalahan, lalu cari struktur data yang tepat kira-kira apa. Contohnya sederhana saja, misalnya tumpukan piring cucian dan antrian kantor pajak
  \item Jelaskan tentang Binary Search Tree, supaya ada variasi yang lebih jelas. Biasanya kalau hanya belajar stack/queue/linked list, orang tidak terlalu paham abstraksi dari struktur data. Cukup jelaskan operasi insert dan find saja. Operasi delete cukup disebut, dan beri link supaya peserta bisa baca kalau berminat.
\end{itemize}
\end{frame}

\begin{frame}
\frametitle{Tips}
\begin{itemize}
  \item Setiap penjelasan tentang struktur data ada baiknya menjabarkan operasi apa saja yang didukung
  \item Misalnya untuk BST, operasi yang didukung adalah insert(x), dan find(x).
  \item Jelaskan kompleksitas untuk setiap operasi yang didukung.
\end{itemize}
\end{frame}

\begin{frame}
\frametitle{Tips}
Demi keseragaman dan kemudahan, gunakan command yang sudah didefinisikan di ../config.tex:
% Komentar: meskipun ada yang menghasilkan format yang sama, tetapi menggunakan command ini mempermudah kalau ke depannya mau mengubah format, tidak perlu find and replace 1 per 1. 
\begin{itemize}
  \item Gunakan \textbackslash id\{\} untuk menulis identifier (nama fungsi, variabel), contoh: \textbackslash id\{primeCount\} akan mencetak \id{primeCount}.
  \item Gunakan \textbackslash progTerm\{\} untuk menulis istilah dalam pemrograman (tipe data, jenis error, for), contoh: \textbackslash progTerm\{integer\} akan mencetak \progTerm{integer}.
  \item Gunakan \textbackslash foreignTerm\{\} untuk menulis istilah asing lainnya, contoh: \textbackslash foreignTerm\{default\} akan mencetak \foreignTerm{default}.
\end{itemize}
\end{frame}

\begin{frame}
\frametitle{Tips (lanj.)}
\begin{itemize}
  \item Jika diperlukan, gunakan \textbackslash newTerm\{\} untuk menulis istilah yang baru diperkanalkan, contoh: \textbackslash newTerm\{rekursi\} akan mencetak \newTerm{rekursi}.
  \item Gunakan \textbackslash emp\{\} untuk memberi penekanan, contoh: \textbackslash emp\{sangat lambat\} akan mencetak \emp{sangat lambat}.
  \item Gunakan \textbackslash statement\{\} untuk merujuk pada statement program, contoh: \textbackslash statement\{count := count + 1\} akan mencetak \statement{count := count + 1}.
\end{itemize}
\end{frame}

\end{document}
