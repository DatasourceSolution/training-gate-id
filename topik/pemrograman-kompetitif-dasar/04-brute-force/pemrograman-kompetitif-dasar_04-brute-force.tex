\documentclass{beamer}
\usetheme{tokitex}

\usepackage{graphics}
\usepackage{multirow}
\usepackage{tabto}

\usepackage[english,bahasa]{babel}
\newtranslation[to=bahasa]{Section}{Bagian}
\newtranslation[to=bahasa]{Subsection}{Subbagian}

\usepackage{listings, lstautogobble}
\usepackage{color}

\definecolor{dkgreen}{rgb}{0,0.6,0}
\definecolor{gray}{rgb}{0.5,0.5,0.5}
\definecolor{mauve}{rgb}{0.58,0,0.82}

\lstset{frame=tb,
  language=pascal,
  aboveskip=1mm,
  belowskip=1mm,
  showstringspaces=false,
  columns=fullflexible,
  keepspaces=true,
  basicstyle={\small\ttfamily},
  numbers=none,
  numberstyle=\tiny\color{gray},
  keywordstyle=\color{blue},
  commentstyle=\color{dkgreen},
  stringstyle=\color{mauve},
  breaklines=true,
  breakatwhitespace=true,
  autogobble=true
}

\title{Brute Force}
\author{Tim Olimpiade Komputer Indonesia}
\date{}

\begin{document}

\begin{frame}
\titlepage
\end{frame}

\begin{frame}
\frametitle{Pendahuluan}
Melalui dokumen ini, kalian akan:
\begin{itemize}
  \item Mempelajari konsep \fBruteForce.
  \item Mampu mengerjakan persoalan dengan pendekatan \fBruteForce.
\end{itemize}
\end{frame}

\begin{frame}
\frametitle{Konsep}
\begin{itemize}
  \item \fBruteForce bukan suatu algoritma khusus, melainkan suatu strategi penyelesaian masalah.
  \item Sebutan lainnya adalah \fCompleteSearch dan \fExhaustiveSearch.
  \item Prinsip dari strategi ini hanya satu, yaitu...
\end{itemize}
\end{frame}

\begin{frame}
\frametitle{Konsep (lanj.)}
\begin{center}
  \huge Coba semua kemungkinan,\\ambil yang terbaik!
\end{center}
\end{frame}

\begin{frame}
\frametitle{Sifat Brute Force}
\begin{itemize}
  \item \fBruteForce \emp{menjamin} solusi pasti benar, karena seluruh kemungkinan dijelajahi.
  \item Akibatnya, umumnya \fBruteForce bekerja dengan lambat.
  \item Terutama ketika banyak kemungkinan solusi yang perlu dicoba.
\end{itemize}
\end{frame}

\begin{frame}
\frametitle{Soal: Subset Sum}
\begin{itemize}
  \item Diberikan $N$ buah bilangan $(a_1, a_2, ..., a_N)$ dan bilangan $K$
  \item Apakah terdapat subset dari bilangan-bilangan tersebut sehingga jumlahan dari elemen subset tersebut sama dengan $K$? 
  \item Bila iya, maka keluarkan "YA". Selain itu keluarkan "TIDAK"
\end{itemize}

Batasan:
\begin{itemize} 
  \item $1 \leq N \leq 15$
  \item $1 \leq K \leq 10^9$
  \item $1 \leq a_i \leq 10^9$
\end{itemize}
\end{frame}

\begin{frame}
\frametitle{Solusi}
\begin{itemize}
  \item Untuk setiap elemen, kita memiliki 2 pilihan yaitu memilih elemen tersebut atau tidak memilihnya.
  \item Kita akan menelusuri semua kemungkinan pilihan.
  \item Jika jumlahan dari elemen-elemen yang dipilih sama dengan $K$, maka terdapat solusi.
  \item Hal ini dapat dengan mudah diimplementasikan secara rekursif.
\end{itemize}
\end{frame}

\begin{frame}
\frametitle{Performa?}
\begin{itemize}
  \item Terdapat $2^N$ kemungkinan konfigurasi "pilih - tidak pilih".
  \item Kompleksitas solusi adalah $O(2^N)$.
  \item Untuk nilai $N$ terbesar, $2^N = 2^{15} = 32.768$.
  \item Masih jauh di bawah 100 juta, yaitu banyaknya operasi komputer per detik pada umumnya.
\end{itemize}
\end{frame}

\begin{frame}
\frametitle{Implementasi}
\begin{codebox}
\Procname{$\proc{solve}(i, sum, N, K)$}
\li \If $i > N$ \Then
\li   \Return $(sum \isequal K)$
\li \Else 
\li   $noPick \gets \proc{solve}(i+1, sum, N, K)$
\li   $pick \gets \proc{solve}(i+1, sum + a_i, N, K)$
\li   \Comment Cukup setidaknya salah satu bernilai $true$
\li   \Return $noPick$ or $pick$
    \End
\end{codebox}

\begin{codebox}
\Procname{$\proc{solveSubsetSum}(a, N, K)$}
\li $result \gets \proc{solve}(1, 0, N, K)$
\li \Comment Cetak keluaran sesuai nilai $result$
\end{codebox}
\end{frame}

\begin{frame}
\frametitle{Optimisasi}
\begin{itemize}
  \item Bisakah solusi tersebut menjadi lebih cepat?
  \item Perhatikan kasus ketika nilai $sum$ telah melebihi $K$.
  \item Karena semua $a_i$ bernilai positif, maka $sum$ tidak akan mengecil.
  \item Karena itu, bila $sum$ sudah melebihi $K$, \emp{dipastikan} tidak akan tercapai sebuah solusi.
\end{itemize}
\end{frame}

\begin{frame}
\frametitle{Solusi Optimisasi}
\begin{codebox}
\Procname{$\proc{solve}(i, sum, N, K)$}
\li \If $sum > K$ \Then
\li   \Return $false$
    \End
\zi
\li \If $i > N$ \Then
\li   \Return $(sum \isequal K)$
\li \Else 
\li   $noPick \gets \proc{solve}(i+1, sum, N, K)$
\li   $pick \gets \proc{solve}(i+1, sum + a_i, N, K)$
\li   \Comment Cukup setidaknya salah satu bernilai $true$
\li   \Return $noPick$ or $pick$
    \End
\end{codebox}
\end{frame}

\begin{frame}
\frametitle{Pruning}
Hal ini biasa disebut sebagai \newTerm{pruning}.
\begin{block}{Pruning}
  Merupakan optimisasi dengan mengurangi ruang pencarian dengan cara menghindari pencarian yang sudah pasti salah.
\end{block}
\end{frame}

\begin{frame}
\frametitle{Pruning (lanj.)}
\begin{itemize}
  \item Meskipun mengurangi ruang pencarian, \fpruning umumnya tidak mengurangi kompleksitas solusi.
  \item Pada kasus ini, solusi dapat dianggap tetap bekerja dalam $O(2^N)$.
\end{itemize}
\end{frame}

\begin{frame}
\frametitle{Soal: Mengatur Persamaan}
\begin{itemize}
  \item Diberikan $N$ buah bilangan $(a_1, a_2, ..., a_N)$.
  \item Diberikan sebuah persamaan: $p + q + r = 0$.
  \item Nilai $p$, $q$, dan $r$ harus merupakan anggota $\{ a_1, a_2, ..., a_N \}$
  \item Berapa banyak ($p, q, r$) berbeda yang memenuhi persamaan tersebut?
\end{itemize}

Batasan:
\begin{itemize} 
  \item $1 \leq N \leq 2.000$
  \item $-10^5 \leq a_i \leq 10^5$
\end{itemize}
\end{frame}

\begin{frame}
\frametitle{Solusi Sederhana}
Masukkan nilai-nilai $\{ a_1, a_2, ..., a_N \}$ ke $p$, $q$, dan $r$, lalu periksa apakah persamaannya terpenuhi.

\begin{codebox}
\Procname{$\proc{countTriplet}(a, N)$}
\li $count \gets 0$
\li \For $i \gets 1$ \To $N$ \Do
\li   \For $j \gets 1$ \To $N$ \Do
\li     \For $k \gets 1$ \To $N$ \Do
\li       $p \gets a_i$
\li       $q \gets a_j$
\li       $r \gets a_k$
\li       \If $(p + q + r) \isequal 0$ \Then
\li         $count \gets count + 1$.
          \End
        \End
      \End
    \End
\li \Return $count$
\end{codebox}
\end{frame}

\begin{frame}
\frametitle{Solusi Sederhana (lanj.)}
\begin{itemize}
  \item Kompleksitas waktu cara ini adalah $O(N^3)$.
  \item Tentunya terlalu besar untuk nilai $N$ mencapai 2.000.
  \item Ada cara yang lebih baik?
\end{itemize}
\end{frame}

\begin{frame}
\frametitle{Observasi}
\begin{itemize}
  \item Jika kita sudah menentukan nilai $p$ dan $q$, maka nilai $r$ haruslah $-(p + q)$.
  \item Jadi cukup tentukan nilai $p$ dan $q$, lalu periksa apakah nilai $-(p + q)$ ada pada bilangan-bilangan yang diberikan.
  \item Pemeriksaan ini dapat dilakukan dengan \fBinarySearch.
  \item Kompleksitas solusi menjadi $O(N^2 \log{N})$
\end{itemize}
\end{frame}

\begin{frame}
\frametitle{Solusi Lebih Baik}
\begin{codebox}
\Procname{$\proc{countTripletFast}(a, N)$}
\li $count \gets 0$
\li \For $i \gets 1$ \To $N$ \Do
\li   \For $j \gets 1$ \To $N$ \Do
\li       $p \gets a_i$
\li       $q \gets a_j$
\li       $r \gets -(p + q)$
\li       \If $\proc{exist}(a, N, r)$ \Then
\li         $count \gets count + 1$.
          \End
        \End
      \End
    \End
\li \Return $count$
\end{codebox}

Dengan $\proc{exist}(a, N, r)$ adalah algoritma \fBinarySearch untuk memeriksa keberadaan $r$ di $\{ a_1, a_2, ..., a_N \}$.
\end{frame}

\begin{frame}
\frametitle{Solusi Lebih Baik (lanj.)}
\begin{itemize}
  \item Kompleksitas $O(N^2 \log{N})$ sudah cukup untuk $N$ yang mencapai 2.000.
  \item Dari sini kita belajar bahwa optimisasi pada pencarian kadang diperlukan, meskipun ide dasarnya adalah \fBruteForce.
\end{itemize}
\end{frame}

\begin{frame}
\frametitle{Penutup}
\begin{itemize}
  \item Ide dari \fBruteForce biasanya sederhana, Anda hanya perlu menjelajahi seluruh kemungkinan solusi.
  \item Biasanya merupakan ide pertama yang didapatkan saat menghadapi masalah.
  \item Lakukan analisis algoritma, jika kompleksitasnya cukup, maka \fBruteForce saja :)
  \item Bila tidak cukup cepat, coba lakukan observasi. 
  \item Bisa jadi kita dapat melakukan \fBruteForce dari "sudut pandang yang lain" dan lebih cepat.
  \item Bila tidak berhasil juga, baru coba pikirkan strategi lainnya.
\end{itemize}
\end{frame}

\end{document}
