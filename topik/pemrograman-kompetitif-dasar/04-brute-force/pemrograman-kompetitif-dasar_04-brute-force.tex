\documentclass{beamer}
\usetheme{tokitex}

\usepackage{graphics}
\usepackage{multirow}
\usepackage{tabto}

\usepackage[english,bahasa]{babel}
\newtranslation[to=bahasa]{Section}{Bagian}
\newtranslation[to=bahasa]{Subsection}{Subbagian}

\usepackage{listings, lstautogobble}
\usepackage{color}

\definecolor{dkgreen}{rgb}{0,0.6,0}
\definecolor{gray}{rgb}{0.5,0.5,0.5}
\definecolor{mauve}{rgb}{0.58,0,0.82}

\lstset{frame=tb,
  language=pascal,
  aboveskip=1mm,
  belowskip=1mm,
  showstringspaces=false,
  columns=fullflexible,
  keepspaces=true,
  basicstyle={\small\ttfamily},
  numbers=none,
  numberstyle=\tiny\color{gray},
  keywordstyle=\color{blue},
  commentstyle=\color{dkgreen},
  stringstyle=\color{mauve},
  breaklines=true,
  breakatwhitespace=true,
  autogobble=true
}

\title{Brute Force}
\author{Tim Olimpiade Komputer Indonesia}
\date{}

\begin{document}

\begin{frame}
\titlepage
\end{frame}

\begin{frame}
\frametitle{Apa Itu Brute Force?}

\begin{block}{Brute Force}
\foreignTerm{brute-force search or exhaustive search, also known as generate and test, is a very general problem-solving technique that consists of systematically enumerating all possible candidates for the solution and checking whether each candidate satisfies the problem's statement.}

-Wikipedia-
\end{block}

\end{frame}

\begin{frame}
\frametitle{Panduan Konten}
\begin{itemize}
  \item Jelaskan bahwa brute force adalah suatu pendekatan penyelesaian masalah, dengan mencari semua kemungkinan dan mengambil yang tepat. Ini yang paling dasar
  \item Tekankan bahwa brute force \emp{pasti benar}, meskipun bisa jadi lambat
  \item Anggap saja brute force sebagai foreign term, jadi tulis dengan \textbackslash foreignTerm\{brute force\}
  \item Berikan contoh permasalahan, selesaikan dengan brute force
  \item Berikan contoh lagi yang lebih sulit, dan tunjukkan bahwa kadang2 kita harus pintar dalam menentukan apa yang mau di-brute force
  \item Ada baiknya salah satu permasalahan perlu menggunakan rekursi (backtracking) untuk menyelesaikannya
  \item Sekalian juga perkenalkan tentang backtracking, pruning, dsb
  \item Ajarkan untuk hitung kompleksitasnya. Kalau cukup untuk AC, langsung coding aja
\end{itemize}
\end{frame}

\begin{frame}
\frametitle{Tips}
Demi keseragaman dan kemudahan, gunakan command yang sudah didefinisikan di ../config.tex:
% Komentar: meskipun ada yang menghasilkan format yang sama, tetapi menggunakan command ini mempermudah kalau ke depannya mau mengubah format, tidak perlu find and replace 1 per 1. 
\begin{itemize}
  \item Gunakan \textbackslash id\{\} untuk menulis identifier (nama fungsi, variabel), contoh: \textbackslash id\{primeCount\} akan mencetak \id{primeCount}.
  \item Gunakan \textbackslash progTerm\{\} untuk menulis istilah dalam pemrograman (tipe data, jenis error, for), contoh: \textbackslash progTerm\{integer\} akan mencetak \progTerm{integer}.
  \item Gunakan \textbackslash foreignTerm\{\} untuk menulis istilah asing lainnya, contoh: \textbackslash foreignTerm\{default\} akan mencetak \foreignTerm{default}.
\end{itemize}
\end{frame}

\begin{frame}
\frametitle{Tips (lanj.)}
\begin{itemize}
  \item Jika diperlukan, gunakan \textbackslash newTerm\{\} untuk menulis istilah yang baru diperkanalkan, contoh: \textbackslash newTerm\{rekursi\} akan mencetak \newTerm{rekursi}.
  \item Gunakan \textbackslash emp\{\} untuk memberi penekanan, contoh: \textbackslash emp\{sangat lambat\} akan mencetak \emp{sangat lambat}.
  \item Gunakan \textbackslash statement\{\} untuk merujuk pada statement program, contoh: \textbackslash statement\{count := count + 1\} akan mencetak \statement{count := count + 1}.
\end{itemize}
\end{frame}

\end{document}
