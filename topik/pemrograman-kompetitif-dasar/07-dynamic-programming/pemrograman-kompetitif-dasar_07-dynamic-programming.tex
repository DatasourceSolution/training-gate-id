\documentclass{beamer}
\usetheme{tokitex}

\usepackage{graphics}
\usepackage{multirow}
\usepackage{tabto}

\usepackage[english,bahasa]{babel}
\newtranslation[to=bahasa]{Section}{Bagian}
\newtranslation[to=bahasa]{Subsection}{Subbagian}

\usepackage{listings, lstautogobble}
\usepackage{color}

\definecolor{dkgreen}{rgb}{0,0.6,0}
\definecolor{gray}{rgb}{0.5,0.5,0.5}
\definecolor{mauve}{rgb}{0.58,0,0.82}

\lstset{frame=tb,
  language=pascal,
  aboveskip=1mm,
  belowskip=1mm,
  showstringspaces=false,
  columns=fullflexible,
  keepspaces=true,
  basicstyle={\small\ttfamily},
  numbers=none,
  numberstyle=\tiny\color{gray},
  keywordstyle=\color{blue},
  commentstyle=\color{dkgreen},
  stringstyle=\color{mauve},
  breaklines=true,
  breakatwhitespace=true,
  autogobble=true
}

\title{Dynamic Programming}
\author{Tim Olimpiade Komputer Indonesia}
\date{}

\begin{document}

\begin{frame}
\titlepage
\end{frame}

\begin{frame}
\frametitle{Panduan Konten}
\begin{itemize}
  \item Disarankan untuk tidak langsung menjelaskan konsepnya
  \item Disarankan untuk mulai dengan coin change, koinnya \{1, 6, 10\} dan mau tukar 12
  \item Jelaskan mengapa greedy menjadi salah
  \item Bahas dengan mencoba diselesaikan secara rekursif
  \item Jelaskan bahwa kompleksitasnya eksponensial
  \item Jelaskan kalau ada banyak kasus overlap, dan pada akhirnya hanya ada O(N) kasus (state)
  \item Aplikasikan tabel memo DP, dan jelaskan bahwa sekarang kompleksitasnya polinomial
  \item Disarankan untuk menjelaskan dengan DP top down (rekursif + memo)
\end{itemize}
\end{frame}

\begin{frame}
\frametitle{Panduan Konten}
\begin{itemize}
  \item Jelaskan tentang DP bottom up (lanjutkan pembahasan coin change sebelumnya)
  \item Barulah bahas konsep DP: ada pilihan, kalau salah satu dipilih bakal muncul subproblem serupa dengan cakupan lebih kecil. Solusinya adalah coba saja semua kemungkinan, dan pilih yg hasil akhirnya terbaik
  \item Boleh disarankan bahwa untuk menyelesaikan soal DP, dapat cari dulu hubungan rekursifnya lalu temukan base case. Sebenarnya ini tergantung preferensi, ada juga yang lebih suka cari base case, baru disambung2 dan ditemukan hubungan rekursifnya 
  \item Lanjut ke studi kasus knapsack dan matrix chain multiplication. Gunakan pola yang sebelumnya: coba solve secara rekursif, cari statenya, berikan memo, hitung kompleksitas, contoh implementasi
\end{itemize}
\end{frame}

\begin{frame}
\frametitle{Catatan}
\begin{itemize}
  \item Jika isi slide ini menjadi terlalu banyak, boleh dipecah menjadi 2 atau 3 slide sesuai subtopik pembahasan.
\end{itemize}
\end{frame}

\begin{frame}
\frametitle{Tips}
Demi keseragaman dan kemudahan, gunakan command yang sudah didefinisikan di ../config.tex:
% Komentar: meskipun ada yang menghasilkan format yang sama, tetapi menggunakan command ini mempermudah kalau ke depannya mau mengubah format, tidak perlu find and replace 1 per 1. 
\begin{itemize}
  \item Gunakan \textbackslash id\{\} untuk menulis identifier (nama fungsi, variabel), contoh: \textbackslash id\{primeCount\} akan mencetak \id{primeCount}.
  \item Gunakan \textbackslash progTerm\{\} untuk menulis istilah dalam pemrograman (tipe data, jenis error, for), contoh: \textbackslash progTerm\{integer\} akan mencetak \progTerm{integer}.
  \item Gunakan \textbackslash foreignTerm\{\} untuk menulis istilah asing lainnya, contoh: \textbackslash foreignTerm\{default\} akan mencetak \foreignTerm{default}.
\end{itemize}
\end{frame}

\begin{frame}
\frametitle{Tips (lanj.)}
\begin{itemize}
  \item Jika diperlukan, gunakan \textbackslash newTerm\{\} untuk menulis istilah yang baru diperkanalkan, contoh: \textbackslash newTerm\{rekursi\} akan mencetak \newTerm{rekursi}.
  \item Gunakan \textbackslash emp\{\} untuk memberi penekanan, contoh: \textbackslash emp\{sangat lambat\} akan mencetak \emp{sangat lambat}.
  \item Gunakan \textbackslash statement\{\} untuk merujuk pada statement program, contoh: \textbackslash statement\{count := count + 1\} akan mencetak \statement{count := count + 1}.
\end{itemize}
\end{frame}

\end{document}
