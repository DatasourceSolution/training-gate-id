 \documentclass{beamer}
\usetheme{tokitex}

\usepackage{graphics}
\usepackage{multirow}
\usepackage{tabto}

\usepackage[english,bahasa]{babel}
\newtranslation[to=bahasa]{Section}{Bagian}
\newtranslation[to=bahasa]{Subsection}{Subbagian}

\usepackage{listings, lstautogobble}
\usepackage{color}

\definecolor{dkgreen}{rgb}{0,0.6,0}
\definecolor{gray}{rgb}{0.5,0.5,0.5}
\definecolor{mauve}{rgb}{0.58,0,0.82}

\lstset{frame=tb,
  language=pascal,
  aboveskip=1mm,
  belowskip=1mm,
  showstringspaces=false,
  columns=fullflexible,
  keepspaces=true,
  basicstyle={\small\ttfamily},
  numbers=none,
  numberstyle=\tiny\color{gray},
  keywordstyle=\color{blue},
  commentstyle=\color{dkgreen},
  stringstyle=\color{mauve},
  breaklines=true,
  breakatwhitespace=true,
  autogobble=true
}

\title{Greedy}
\author{Tim Olimpiade Komputer Indonesia}
\date{}

\begin{document}

\begin{frame}
	\titlepage
\end{frame}

\begin{frame}
	\frametitle{Pendahuluan}
	Melalui dokumen ini, kalian akan:

\end{frame}
	
\begin{frame}
	\frametitle{Menaklukkan OSN}
	Kompetisi pemrograman, OSN pada khususnya membutuhkan persiapan yang matang.
	
	
\end{frame}

\begin{frame}
	\frametitle{Mengenali medan}
	Mengenali kondisi kompetisi yang akan anda ikuti adalah hal yang penting. Kondisi ini antara lain:
	\begin{itemize}
		\item Perhitungan poin \ cara pemilihan pemenang
		\item Jumlah dan tingkat kesulitan soal
		\item Waktu yang diberikan
		\item \textit{Resource} yang ada
	\end{itemize}
\end{frame}


\begin{frame}
	\frametitle{Perhitungan poin}
	Ada dua jenis cara perhitungan poin yang umum digunakan pada kompetisi pemrograman:
	\begin{itemize}
		\item IOI \textit{style}
		~
		pada jenis IOI, anda bisa pendapatkan poin secara parsial, tergantung \textit{subtask} yang and kerjakan. Waktu pengerjaan tidak berpengaruh pada hasil akhir. OSN menggunakan sistem penilaian ini.
		
		\item ACM \textit{style}
		~
		pada jenis ini, anda harus menyelesaikan masalah secara keseluruhan. Dengan kata lain, nilai anda adalah 0 atau 100. Penentuan pemenang dihitung dari total soal yang diselesaikan.  Waktu pengerjaan dihitung sebagai penalti.
	\end{itemize}
\end{frame}

\begin{frame}
	\frametitle{Jumlah dan tingkat kesulitan soal}
		
	\begin{itemize}
		\item OSN terdiri dari 3 soal perhari, dengan total 6 soal
		\item kesulitan soal OSN bervariasi mulai dari mudah sampai sulit.
		\item Biasanya akan ada soal yang bertipe \textit{open problem}. Soal ini tidak memiliki solusi pasti. Artinya Anda akan membuat program yang mendekati kebenaran sebaik mungkin. Penjelasan mendalam seputar \textit{open problem} akan dijelaskan di beberapa slide berikutnya.
	\end{itemize}
\end{frame}

\begin{frame}
	\frametitle{Waktu dan \textit{resource}}
	\begin{itemize}
		\item waktu kontes OSN adalah 5 jam perhari
		\item Perhatikan \textit{resouce} berupa:
		\begin{itemize}
			\item \textit{compiler} yang disediakan
			\item \textit{editor} yang disediakan
			\item sistem operasi yang disediakan
			\item kemampuan mesin
		\end{itemize}
		\item biasakan diri dengan sistem operasi, editor serta kompiler yang akan dipakai
		\item informasi kemampuan mesin bisa digunakan untuk memprediksi \textit{running time}
	\end{itemize}
\end{frame}

\begin{frame}
	\frametitle{Persiapan sebelum kompetisi}
	\begin{itemize}
		\item 	Lakukan latihan secara berkala sebelum kompetisi. Latihan bisa dilakukan di:
		\begin{itemize}
			\item TLX
			\item codeforces
			\item topoder
			\item SPOJ
			\item COCI
			\item dll
		\end{itemize}
		\item Simulasikan OSN. Bisa menyelesaikan soal setingkat OSN saja tidak cukup. Anda harus dapat menyelesaikannya dalam situasi yang serupa saat kompetisi. Buatlah simulasi OSN dengan memilih 3 buah soal, kemudian kerjakan sebaik-baiknya selama 5 jam tanpa bantuan.
	\end{itemize}
\end{frame}

\begin{frame}
	\frametitle{Waktu dan \textit{resource}}
	\begin{itemize}
		\item waktu kontes OSN adalah 5 jam perhari
		\item Perhatikan \textit{resouce} berupa:
		\begin{itemize}
			\item \textit{compiler} yang disediakan
			\item \textit{editor} yang disediakan
			\item sistem operasi yang disediakan
			\item kemampuan mesin
		\end{itemize}
		\item biasakan diri dengan sistem operasi, editor serta kompiler yang akan dipakai
		\item informasi kemampuan mesin bisa digunakan untuk memprediksi \textit{running time}
	\end{itemize}
\end{frame}

\begin{frame}
	\frametitle{Menaklukkan \textit{open problem}}

\end{frame}
\end{document}
