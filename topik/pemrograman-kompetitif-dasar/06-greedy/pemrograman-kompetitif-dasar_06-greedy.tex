 \documentclass{beamer}
\usetheme{tokitex}

\usepackage{graphics}
\usepackage{multirow}
\usepackage{tabto}

\usepackage[english,bahasa]{babel}
\newtranslation[to=bahasa]{Section}{Bagian}
\newtranslation[to=bahasa]{Subsection}{Subbagian}

\usepackage{listings, lstautogobble}
\usepackage{color}

\definecolor{dkgreen}{rgb}{0,0.6,0}
\definecolor{gray}{rgb}{0.5,0.5,0.5}
\definecolor{mauve}{rgb}{0.58,0,0.82}

\lstset{frame=tb,
  language=pascal,
  aboveskip=1mm,
  belowskip=1mm,
  showstringspaces=false,
  columns=fullflexible,
  keepspaces=true,
  basicstyle={\small\ttfamily},
  numbers=none,
  numberstyle=\tiny\color{gray},
  keywordstyle=\color{blue},
  commentstyle=\color{dkgreen},
  stringstyle=\color{mauve},
  breaklines=true,
  breakatwhitespace=true,
  autogobble=true
}

\title{Greedy}
\author{Tim Olimpiade Komputer Indonesia}
\date{}

\begin{document}

\begin{frame}
	\titlepage
\end{frame}

\begin{frame}
	\frametitle{Pendahuluan}
	Melalui dokumen ini, kalian akan:
	\begin{itemize}
		\item Memahami konsep \foreignTerm{greedy}
		\item Menyelesaikan beberapa contoh persoalan \foreignTerm{greedy} sederhana
	\end{itemize}
\end{frame}
	
\begin{frame}
	\frametitle{Greedy}
	Suatu persoalan dapat diselesaikan dengan teknik \foreignTerm{greedy} jika persoalan tersebut memiliki memiliki dua properti berikut:
	\begin{itemize}

		\item Solusi optimal dari persoalan dapat ditentukan dari solusi optimal sub-persoalan tersebut
		\item Pada setiap sub-persoalan, ada suatu langkah yang bisa dilakukan yang mana langkah tersebut menghasilkan solusi maksimum pada sub-persoalan tersebut. Langkah ini disebut juga \foreignTerm{greedy choice}. 
	\end{itemize}
\end{frame}

\begin{frame}
	\frametitle{Contoh Soal: Activity Selection}
	\begin{itemize}
		\item Diberikan $N$ buah aktivitas. Aktivitas ke-$i = <a_i.start, a_i.end>$,  dimulai pada waktu $a_i.start$ dan berakhir pada waktu $a_i.end$
		\item Pada setiap satuan waktu, Anda dapat mengikuti paling banyak satu aktivitas.
		\item Anda ingin mengatur jadwal sedemikian sehingga Anda bisa ikut aktivitas sebanyak mungkin.
		\item Sebagai contoh, diberikan 4 buah aktivitas: $[ <1, 3> , <2, 6>, <4, 7>,  <8, 9> ]$. Anda dapat hadir di 3 aktivitas berbeda yang tidak saling tindih, yaitu $<1, 3>, <4, 7>, dan <8, 9>$. 
	\end{itemize}
\end{frame}



\begin{frame}
	\frametitle{Contoh Soal: Activity Selection (solusi)}
	Solusi dari persoalan ini bisa diperoleh dari solusi sub-persoalan yang lebih kecil:
	
	\begin{itemize}
		\item Kita definisikan $f(t)$ sebagai jumlah aktivitas terbanyak yang dapat diikuti dari waktu ke-0 sampai waktu ke-$t$.
		\item Jika kita hendak memilih suatu aktivitas $<t' , t>$, maka kita dapat menghitung nilai $f(t) = 1 + f(t' - 1)$ 
		\item Artinya, untuk dapat menemukan solusi optimal pada $f(t)$, kita perlu menentukan solusi optimal untuk waktu-waktu yang lebih awal.
		\item Solusi akhir dari persoalan ini adalah $f(max(e))$, yang mana $max(e)$ adalah nilai $a_i.end$ terbesar dari semua pilihan aktivitas yang ada.
	\end{itemize}
\end{frame}

\begin{frame}
	\frametitle{Contoh Soal: Activity Selection (solusi)}
		
	\begin{itemize}
		\item Asumsikan kita sudah menemukan solusi untuk $f(t)$. 
		\item Kita memiliki beberapa pilihan aktivitas yang dapat dipilih, yang mana waktu mulai aktivitas tersebut harus lebih besar dari $t$.
		\item  \foreignTerm{Greedy choice} untuk permasalah ini adalah dengan memilih aktivitas ke-$k$ yang mana aktivitas tersebut memiliki waktu selesai yang paling cepat. 
		\item Kita bisa hitung solusi dari sub-problem yang lebih besar yaitu $f(a_k.end) = f(t) + 1$. Kita bisa perbarui nilai $t$ menjadi $a_k.end$.
		\item Jika pada waktu $t$ sudah tidak ada lagi aktivitas yang dapat dipilih, maka nilai $f(t)$ merupakan solusi akhir.
	\end{itemize}
\end{frame}

\begin{frame}
	\frametitle{Contoh Soal: Activity Selection (solusi)}
	Mencari \foreignTerm{greedy choice} yang tepat
	\\~
	\\
	Pada paparan sebelumnya, kita memilih aktivitas yang memiliki waktu selesai paling cepat. Apakah ada pilihan  \foreignTerm{greedy choice} yang lainnya? Perhatikan pilihan berikut:
	\begin{itemize}
		\item Memilih aktivitas dengan waktu mulai paling awal
		\item Memilih aktivitas dengan durasi paling singkat
		\item Memilih aktivitas dengan waktu akhir paling awal
	\end{itemize}
\end{frame}

\begin{frame}
	\frametitle{Contoh Soal: Activity Selection (solusi)}
	Memilih aktivitas dengan waktu mulai paling awal
	\begin{itemize}
		\item Pilihan ini tidak menghasilkan solusi optimal.
		\item Bisa jadi ada aktivitas  yang mulai lebih awal, namun memiliki durasi yang sangat panjang sehingga menyita waktu
		\item Contoh kasus: $[ <1,5>, <2, 3> , <4, 5> ]$. Aktivitas paling singkat ialah $<1, 5>$. Namun dengan memilih aktivitas tersebut, kita kehilangan kesempatan untuk mengambil aktivitas $<2, 3> dan <4, 5> $
	\end{itemize}
\end{frame}

\begin{frame}
	\frametitle{Contoh Soal: Activity Selection (solusi)}
	Memilih aktivitas dengan durasi paling singkat
	\begin{itemize}
		\item Pilihan ini juga tidak menghasilkan solusi optimal.
		\item Bisa jadi aktivitas dengan durasi paling singkat ini memotong dua aktivitas lain.
		\item Contoh kasus: $[ <1,10>, <9, 11> , <11, 20> ]$. Aktivitas paling singkat ialah $<9, 11>$. Namun dengan memilih aktivitas tersebut, kita kehilangan kesempatan untuk mengambil aktivitas $<1, 10> dan <11, 20> $
	\end{itemize}
\end{frame}

\begin{frame}
	\frametitle{Contoh Soal: Activity Selection (solusi)}
	Memilih aktivitas dengan waktu akhir paling awal
	\begin{itemize}
		\item Pilihan ini akan menghasilkan solusi optimal.
		\item Dengan memilih aktivitas yang selesai lebih awal, kita mempunyai sisa waktu lebih banyak, untuk memilih aktivitas lainnya.
	\end{itemize}
\end{frame}

\begin{frame} [fragile]
	\frametitle{Contoh Soal: Activity Selection (Pseudocode)}
	\begin{lstlisting}
	f = 0
	t = -1
	sort(a)  // sort menaik berdasarkan a[i].end
	for i = 1 to N
		if (t >= a[i].start)
			continue
		f = f + 1
		t = a[i].end
	return f
	\end{lstlisting}
	Nilai $f$ bisa disimpan dengan satu variabel saja, karena ketika kita sudah memiliki nilai $f(a_i.end)$, nilai $f(t)$ sudah tidak diperlukan lagi.
\end{frame}


\begin{frame}
	\frametitle{permasalahan pada algoritma \foreignTerm{greedy}}
	Perhatikan contoh soal berikut:
	\begin{itemize}
		\item Anda ingin menukar uang $\$12$ dengan pecahan koin $\$5$, $\$2$, dan $\$1$.
		\item Anda ingin menukar dengan jumlah koin sekecil mungkin.
	\end{itemize}
	\textit{Greedy choice} yang bisa dilakukan adalah dengan memilih pecahan koin dengan nominal terbesar yang mungkin untuk tiap sub-persoalan. Pertama-tama kita pilih pecahan $\$5$, sehingga tersisa $\$7$ lagi yang harus dipecah. Selanjutnya kita pilih $\$5$ lagi dan menyisakan $\$2$ untuk dipecah. Akhirnya, kita pilih $\$2$ sebagai pecahan terakhir. Solusi dari kasus ini adalah dengan menggunakan 3 keping koin.
\end{frame}

\begin{frame}
	\frametitle{permasalahan pada algoritma \foreignTerm{greedy}}
	\begin{itemize}
		\item Dengan soal yang sama, bagaimana jika pecahan koin yang tersedia bernilai $\$5$, $\$4$, dan $\$1$?
		\item Dengan algoritma \foreignTerm{greedy}, kita akan menukar  $\$12$ dengan pecahan $\$5$, $\$5$, $\$1$, dan $\$1$. Padahal ada solusi yang lebih baik, yaitu menggunakan 3 keping koin pecahan $\$4$.
		\item Pada kasus tersebut, \textit{greedy choice} yang tidak selalu dapat menghasilkan solusi optimal. Permasalahan ini tidak dapat diselesaikan oleh algoritma \foreignTerm{greedy}.
	\end{itemize}
\end{frame}

\begin{frame}
	\frametitle{permasalahan pada algoritma \foreignTerm{greedy}}
	\begin{itemize}
		\item Pembuktian kebenaran algoritma \foreignTerm{greedy} tidaklah mudah.
		\item Biasanya akan ada beberapa pilihan  \foreignTerm{greedy choice} yang ada, yang mana tidak semuanya bisa menghasilkan solusi optimal.
		\item Untuk contoh soal  \foreignTerm{activity selection}, ada beberapa pilihan  \foreignTerm{greedy choice} antara lain:
		\begin{itemize}
			\item memilih aktivitas dengan waktu mulai paling awal
			\item memilih aktivitas dengan waktu selesai paling awal
			\item memilih aktivitas dengan durasi paling cepat
		\end{itemize}
		\item Namun, hanya pilihan nomor 2 yang menghasilkan solusi optimal.
	\end{itemize}
\end{frame}

\begin{frame}
	\frametitle{permasalahan pada algoritma \foreignTerm{greedy}}
	\begin{itemize}
		\item Ketika menemukan suatu \foreignTerm{greedy choice}, sangat dianjurkan untuk menguji kebenaran dari pilihan tersebut sebelum diimplementasikan.
		\item Pengujian yang dapat dilakukan adalah dengan mencoba membuat contoh-contoh kasus yang mungkin dapat menggagalkan \foreignTerm{greedy choice} tersebut. Teknik ini biasa disebut \foreignTerm{proof by counter-example}.
		\item Jika ditemukan satu saja contoh kasus yang mana \foreignTerm{greedy choice} yang diajukan tidak menghasilkan solusi optimal, maka \foreignTerm{greedy choice} tersebut dinyatakan salah.
	\end{itemize}
\end{frame}

\begin{frame}
\frametitle{Penutup}
\begin{itemize}
		\item Algoritma \textit{greedy} terkadang mudah untuk dipikirkan dan mudah untuk diimplementasikan, namun sulit untuk dibuktikan kebenarannya.
		\item Pembuktian kebenaran algoritma \foreignTerm{greedy} bisa jadi membutuhkan pembuktian matematis yang kompleks dan memakan waktu. 
		\item Pada suasana kompetisi, intuisi dan pengalaman sangat membantu untuk menyelesaikan soal bertipe \foreignTerm{greedy}
		\item  berhati-hati dan telili saat mengerjakan soal bertipe \foreignTerm{greedy}. Perhatikan setiap detil yang ada, karena  bisa berakibat fatal.
		
\end{itemize}
\end{frame}

\end{document}
