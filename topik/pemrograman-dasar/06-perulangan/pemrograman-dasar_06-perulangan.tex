\documentclass{beamer}
\usetheme{tokitex}

\usepackage{graphics}
\usepackage{multirow}
\usepackage{tabto}

\usepackage[english,bahasa]{babel}
\newtranslation[to=bahasa]{Section}{Bagian}
\newtranslation[to=bahasa]{Subsection}{Subbagian}

\usepackage{listings, lstautogobble}
\usepackage{color}

\definecolor{dkgreen}{rgb}{0,0.6,0}
\definecolor{gray}{rgb}{0.5,0.5,0.5}
\definecolor{mauve}{rgb}{0.58,0,0.82}

\lstset{frame=tb,
  language=pascal,
  aboveskip=1mm,
  belowskip=1mm,
  showstringspaces=false,
  columns=fullflexible,
  keepspaces=true,
  basicstyle={\small\ttfamily},
  numbers=none,
  numberstyle=\tiny\color{gray},
  keywordstyle=\color{blue},
  commentstyle=\color{dkgreen},
  stringstyle=\color{mauve},
  breaklines=true,
  breakatwhitespace=true,
  autogobble=true
}

\title{Perulangan}
\author{Tim Olimpiade Komputer Indonesia}
\date{}

\begin{document}

\begin{frame}
\titlepage
\end{frame}

\begin{frame}
\frametitle{Pendahuluan}
Melalui dokumen ini, kalian akan:
\begin{itemize}
  \item Memahami konsep perulangan.
  \item Mempelajari struktur \textbf{for}, \textbf{while}, dan \textbf{repeat} pada Pascal.
\end{itemize}
\end{frame}

\begin{frame}[fragile]
\frametitle{Motivasi}
\begin{itemize}
  \item Hari ini, Pak Dengklek ingin menyambut \textbf{N} ekor bebeknya yang baru lahir dari telur.
  \item Diberikan \textbf{N}, cetak tulisan "halo dunia!" sebanyak \textbf{N} kali!
  \item Contoh untuk \textbf{N} = 3:
  \begin{lstlisting}
    halo dunia!
    halo dunia!
    halo dunia!
  \end{lstlisting}
\end{itemize}
\end{frame}

\begin{frame}
\frametitle{Motivasi (lanj.)}
\begin{itemize}
  \item Solusi "if (N = 1) then $<$cetak satu kali$>$, else if (N = 2) then $<$cetak dua kali$>$, ..." tidak mungkin digunakan, karena \textbf{N} bisa jadi sangat besar.
  \item Kita membutuhkan suatu struktur yang memungkinkan untuk mengulangi serangkaian pekerjaan!
\end{itemize}
\end{frame}

\begin{frame}
\frametitle{Perulangan}
\begin{itemize}
  \item Umumnya setiap bahasa pemrograman memberikan struktur perulangan yang bentuknya bisa berupa:
  \begin{itemize}
    \item Perulangan dengan pencacah. Contoh: "untuk i dari 1 sampai N, cetak 'halo dunia!'".
    \item Perulangan selama dicapai suatu kondisi. Contoh: "selama i $<$ N, cetak 'halo dunia!' dan tambah i dengan 1".
    \item Perulangan sampai dicapai suatu kondisi. Contoh: "cetak 'halo dunia!' dan tambah i dengan 1, hingga tercapai i = N".
  \end{itemize}
  \item Pada Pascal, ada ketiga struktur itu yang masing-masing diwakili dengan \alert{\textbf{for}}, \alert{\textbf{while}}, dan \alert{\textbf{repeat}}.
\end{itemize}
\end{frame}

\begin{frame}[fragile]
\frametitle{Perulangan: for}
\begin{itemize}
  \item Biasanya digunakan ketika kita tahu berapa kali perulangan perlu dilakukan.
  \item Pada Pascal, strukturnya:
  \begin{lstlisting}
    for <pencacah> := <awal> to <akhir> do begin
      <perintah 1>;
      <perintah 2>;
      ...
    end;
  \end{lstlisting}
  \item Dengan $<$pencacah$>$ adalah suatu variabel bertipe data \alert{ordinal}, $<$awal$>$ dan $<$akhir$>$ adalah nilai awal dan akhir untuk pencacah, dan $<$perintah x$>$ adalah perintah yang akan diulang.
  \item "Pencacah" mungkin istilah yang asing bagi kalian. Penjelasan berikutnya dengan contoh akan meningkatkan pemahaman kalian.
\end{itemize}
\end{frame}

\begin{frame}[fragile]
\frametitle{Contoh Program: for.pas}
\begin{itemize}
  \item Ketikkan program berikut dan coba jalankan:
  \begin{lstlisting}
    var
      N, i: longint;
    begin
      write('Masukkan nilai N: ');
      readln(N);

      for i := 1 to N do begin
        writeln('tulisan ini dicetak saat i = ', i);
      end;

      writeln('akhir dari program');
    end.
  \end{lstlisting}
  \item Masukkan berbagai nilai N, misalnya 1, 2, 10, dan 0.
\end{itemize}
\end{frame}

\begin{frame}
\frametitle{Penjelasan Program: for.pas}
\begin{itemize}
  \item Pada contoh tersebut, \textbf{i} merupakan variabel pencacah yang bertipe \textbf{longint}.
  \item Pertama kali dijalankan, \textbf{i} akan bernilai 1 dan tulisan dicetak saat \textbf{i} = 1.
  \item Setelah itu, \textbf{end} dari struktur \textbf{for} ditemukan. Pascal akan menambahkan \textbf{i} dengan 1, lalu kembali ke awal dari \textbf{for}. Jika \textbf{i} masih belum lebih dari \textbf{N}, maka perintah di dalamnya akan kembali dilaksanakan.
  \item Dengan demikian, tercetaklah tulisan saat \textbf{i} = 2, 3, dan seterusnya hingga \textbf{N}.
\end{itemize}
\end{frame}

\begin{frame}
\frametitle{Penjelasan Program: for.pas (lanj.)}
\begin{itemize}
  \item Jika \textbf{i} sudah lebih dari \textbf{N}, perulangan akan berhenti dan Pascal akan menjalankan perintah selanjutnya.
  \item Pada contoh ini, mencetak tulisan "akhir dari program".
  \item Dalam kasus ini, variabel \textbf{i} mencacah 1, 2, 3, ..., \textbf{N}.
\end{itemize}
\end{frame}

\begin{frame}[fragile]
\frametitle{Perulangan: for (lanj.)}
\begin{itemize}
  \item Struktur \textbf{for} hanya bisa mencacah dari suatu nilai awal ke suatu nilai akhir, dan pencacah akan "bertambah satu tingkat" setiap waktunya.
  \item Untuk keperluan pencacah "turun satu tingkat" setiap waktunya, terdapat sebuah variasi lain dari \textbf{for}, yaitu \textbf{for downto} dengan struktur:
  \begin{lstlisting}
    for <pencacah> := <awal> downto <akhir> do begin
      <perintah 1>;
      <perintah 2>;
      ...
    end;
  \end{lstlisting}
  \item Ya, cukup ubah \alert{\textbf{to}} menjadi \alert{\textbf{downto}}. Kini \textbf{for} akan bekerja secara menurun. Tentu saja jika $<$nilai awal$>$ dan $<$nilai akhir$>$ nilainya disesuaikan pula.
\end{itemize}
\end{frame}

\begin{frame}[fragile]
\frametitle{Contoh Program: fordownto.pas}
\begin{itemize}
  \item Berikut ini contoh dari penggunaan \textbf{for downto}:
  \begin{lstlisting}
    var
      N, i: longint;
    begin
      write('Masukkan nilai N: ');
      readln(N);

      for i := N downto 1 do begin
        writeln('tulisan ini dicetak saat i = ', i);
      end;

      writeln('akhir dari program');
    end.
  \end{lstlisting}
\end{itemize}
\end{frame}

\begin{frame}
\frametitle{Hal Penting untuk Perulangan for}
Penting untuk diketahui:
\begin{itemize}
  \item Tipe data dari pencacah, nilai awal, dan nilai akhir harus sama dan bersifat \alert{ordinal}.
  \item Karena tipe data pencacah yang penting merupakan tipe data ordinal, artinya pencacah boleh saja berupa \textbf{char}.
  \item Nilai pencacah tidak boleh diubah saat perulangan dikerjakan.
  \item Pascal tidak bisa melayani \textbf{for} yang pencacahnya "bertambah dua tingkat" pada setiap perulangan.
\end{itemize}
\end{frame}

\begin{frame}[fragile]
\frametitle{Contoh Program: jumlahfor.pas}
\begin{itemize}
  \item Berikut adalah contoh program untuk menjumlahkan bilangan di antara dua bilangan:
  \begin{lstlisting}
    var
      awal, akhir, i: longint;
      jumlah: longint;
    begin
      write('Nilai awal: '); readln(awal);
      write('Nilai akhir: '); readln(akhir);

      jumlah := 0;
      for i := awal to akhir do begin
        jumlah := jumlah + i;
      end;

      writeln('jumlah bilangan bulat di antara awal dan akhir (inklusif) adalah ', jumlah);
    end.
  \end{lstlisting}
\end{itemize}
\end{frame}

\begin{frame}[fragile]
\frametitle{Perulangan: while}
\begin{itemize}
  \item Biasa digunakan ketika tidak diketahui harus berapa kali serangkaian perintah dilaksanakan, tetapi diketahui perintah-perintah itu perlu dilaksanakan selama suatu kondisi terpenuhi.
  \item Pada Pascal, strukturnya:
  \begin{lstlisting}
    while <kondisi> do begin
      <perintah 1>;
      <perintah 2>;
      ...
    end;
  \end{lstlisting}
  \item Seperti pada \textbf{if}, $<$kondisi$>$ adalah suatu nilai boolean. Selama nilainya \textbf{TRUE}, seluruh $<$perintah x$>$ di dalamnya akan dieksekusi secara berurutan.
\end{itemize}
\end{frame}

\begin{frame}[fragile]
\frametitle{Contoh Program: while.pas}
\begin{itemize}
  \item Berikut adalah contoh penggunaan \textbf{while} untuk kasus yang sama dengan for.pas:
  \begin{lstlisting}
    var
      N, i: longint;
    begin
      write('Masukkan nilai N: ');
      readln(N);

      i := 1;
      while (i <= N) do begin
        writeln('tulisan ini dicetak saat i = ', i);
        i := i + 1;
      end;

      writeln('akhir dari program');
    end.
  \end{lstlisting}
\end{itemize}
\end{frame}

\begin{frame}
\frametitle{Penjelasan Program: while.pas}
\begin{itemize}
  \item Misalkan dimasukkan nilai \textbf{N} = 5.
  \item Pada awalnya, nilai \textbf{i} diinisialisasi dengan 1.
  \item Kemudian diperiksa apakah dipenuhi \textbf{i} $<$= \textbf{N}. Karena dipenuhi, perintah mencetak saat \textbf{i} = 1 dilaksanakan. Demikian pula dengan perintah "i := i + 1". Kini nilai \textbf{i} = 2.
  \item Selanjutnya ditemukan \textbf{end} dari \textbf{while}. Pascal akan kembali ke awal dari \textbf{while} dan memeriksa apakah masih dipenuhi kondisi yang diberikan.
  \item Jika masih, maka perintah di dalam while akan kembali dilaksanakan.
  \item Jika sudah tidak dipenuhi, maka Pascal akan mengeksekusi perintah-perintah di bawah \textbf{end} dari \textbf{while}, yakni mencetak tulisan "akhir dari program".
\end{itemize}
\end{frame}

\begin{frame}
\frametitle{Perulangan: while (lanj.)}
\begin{itemize}
  \item Perhatikan bahwa perintah "i := i + 1" diperlukan, supaya suatu saat nanti $<$kondisi$>$ pada while akan tidak dipenuhi.
  \item Sekarang coba hapus perintah "i := i + 1" pada while.pas, dan jalankan kembali programnya.
  \item Apa yang terjadi? Program akan terjebak dalam \alert{\textit{infinite loop}}, atau \alert{perulangan yang tidak akan pernah berhenti}! Gunakan tombol CTRL+C pada \textit{keyboard} untuk memberhentikan program secara paksa.
  \item Pastikan suatu ketika "kondisi pada \textbf{while}" tidak dipenuhi, atau program tidak akan pernah berhenti $:)$
\end{itemize}
\end{frame}

\begin{frame}[fragile]
\frametitle{Contoh Program: jumlahwhile.pas}
\begin{itemize}
  \item Berikut ini contoh program dengan while yang melakukan hal serupa dengan jumlahfor.pas:
  \begin{lstlisting}
    var
      awal, akhir, i, jumlah: longint;
    begin
      write('Nilai awal: '); readln(awal);
      write('Nilai akhir: '); readln(akhir);

      jumlah := 0;
      i := awal;
      while (i <= akhir) do begin
        jumlah := jumlah + i;
        i := i + 1;
      end;

      writeln('jumlah bilangan bulat di antara awal dan akhir (inklusif) adalah ', jumlah);
    end.
  \end{lstlisting}
\end{itemize}
\end{frame}

\begin{frame}[fragile]
\frametitle{Perulangan: while (lanj.)}
\begin{itemize}
  \item Struktur \textbf{while} bisa jadi lebih fleksibel dari struktur \textbf{for}, karena tipe datanya tidak harus ordinal.
  \item Kalian bisa membuat pencacah yang tidak sekedar "naik satu tingkat". Contoh:
  \begin{lstlisting}
    i := 1;
    while (i <= N) do begin
      writeln('i = ', i);
      i := i + 2;
    end;
  \end{lstlisting}

  \item Tidak terbatas pada penjumlahan, hal semacam ini pun bisa dilakukan:
  \begin{lstlisting}
    i := 1;
    while (i <= N) do begin
      writeln('i = ', i);
      i := i * 5;
    end;
  \end{lstlisting}
\end{itemize}
\end{frame}

\begin{frame}[fragile]
\frametitle{Perulangan: while (lanj.)}
\begin{itemize}
  \item Selain itu, dengan menggunakan \textbf{while}, kalian tidak perlu tahu berapa kali suatu perulangan perlu dilakukan.
  \item Contoh soal: Pak Dengklek memberikan serangkaian bilangan, satu pada setiap barisnya, dan cetak bilangan itu. Ketika menemukan bilangan negatif, berhenti membaca dan akhiri program.
  \item Hal ini tidak dapat dilakukan dengan \textbf{for}, karena kita tidak tahu berapa nilai akhir dari \textbf{for} yang harus dilakukan.
  \item Dengan \textbf{while}, hal ini dapat dilakukan dengan mudah:
  \begin{lstlisting}
    readln(x);
    while (x >= 0) do begin
      writeln('bilangan yang dibaca: ', x);
      readln(x);
    end;
  \end{lstlisting}
\end{itemize}
\end{frame}

\begin{frame}[fragile]
\frametitle{Perulangan: repeat}
\begin{itemize}
  \item Struktur perulangan yang lain adalah \textbf{repeat}.
  \item Mirip dengan \textbf{while}, tetapi kali ini bentuknya adalah "lakukan perintah berikut, sampai dicapai suatu kondisi".
  \item Perhatikan bedanya:
  \begin{itemize}
    \item \textbf{while}: selama suatu kondisi terpenuhi
    \item \textbf{repeat}: sampai suatu kondisi terpenuhi
  \end{itemize}
  \item Pada Pascal, strukturnya:
  \begin{lstlisting}
    repeat
      <perintah 1>;
      <perintah 2>;
      ...
    until <kondisi>;
  \end{lstlisting}
\end{itemize}
\end{frame}

\begin{frame}[fragile]
\frametitle{Contoh Program: repeat.pas}
\begin{itemize}
  \item Berikut ini adalah contoh program dengan \textbf{repeat} yang menjalankan tugas serupa dengan for.pas dan while.pas.
  \begin{lstlisting}
    var
      N, i: longint;
    begin
      write('Masukkan nilai N: ');
      readln(N);

      i := 1;
      repeat
        writeln('tulisan ini dicetak saat i = ', i);
        i := i + 1;
      until (i > N);

      writeln('akhir dari program');
    end.
  \end{lstlisting}
\end{itemize}
\end{frame}

\begin{frame}
\frametitle{Penjelasan Program: repeat.pas}
\begin{itemize}
  \item Misalkan \textbf{N} = 5.
  \item Awalnya \textbf{i} diinisialisasi dengan 1.
  \item Memasuki struktur repeat, perintah mencetak saat \textbf{i} = 1 dilaksanakan. Kemudian setelah dilaksanakan "i := i + 1", \textbf{i} bernilai 2.
  \item Memasuki \textbf{until}, diperiksa apakah \textbf{i} $>$ \textbf{N}. Karena belum dipenuhi, maka seluruh perintah sesudah kata kunci \textbf{repeat} akan kembali dilaksanakan.
  \item Ketika \textbf{i} = 5, perintah mencetak saat \textbf{i} = 5 akan dilaksanakan. Setelah itu, \textbf{i} ditambah 1 menjadi 6.
  \item Karena sudah dipenuhi \textbf{i} $>$ \textbf{N}, maka perulangan berhenti dan dilaksanakan perintah di bawah kata \textbf{until}. Dalam contoh ini, mencetak tulisan "akhir dari program".
\end{itemize}
\end{frame}

\begin{frame}
\frametitle{Perulangan: repeat (lanj.)}
\begin{itemize}
  \item Seperti pada \textbf{while}, pastikan suatu saat nanti kondisi berhenti dicapai supaya tidak terjebak pada \textit{infinite loop}.
  \item Perhatikan bahwa seluruh perintah di dalam "repeat ... until" \alert{pasti} dilaksanakan setidaknya satu kali. Karena pemeriksaan pada kondisi baru di lakukan setelah seluruh perintah di dalamnya dilaksanakan.
\end{itemize}
\end{frame}

\begin{frame}[fragile]
\frametitle{Contoh Program Lagi: jumlahrepeat.pas}
\begin{itemize}
  \item Berikut adalah contoh yang serupa dengan jumlahfor.pas.
  \begin{lstlisting}
    var
      awal, akhir, i: longint;
      jumlah: longint;
    begin
      write('Nilai awal: '); readln(awal);
      write('Nilai akhir: '); readln(akhir);

      jumlah := 0;
      i := awal;
      repeat
        jumlah := jumlah + i;
        i := i + 1;
      until (i > akhir);

      writeln('jumlah bilangan bulat di antara awal dan akhir (inklusif) adalah ', jumlah);
    end.
  \end{lstlisting}
\end{itemize}
\end{frame}

\begin{frame}
\frametitle{Sejauh ini...}
Kalian sudah belajar tentang:
\begin{itemize}
  \item Konsep perulangan pada pemrograman.
  \item Struktur \textbf{for}, \textbf{while}, dan \textbf{repeat} beserta kegunaan dan perbedaannya.
\end{itemize}
Selanjutnya kita akan memasuki tentang:
\begin{itemize}
  \item Penggunaan perulangan yang lebih kompleks, yaitu perulangan bersarang.
  \item Membuat program dengan apa yang telah dipelajari sejauh ini.
\end{itemize}
\end{frame}

\end{document}
