\documentclass{beamer}
\usetheme{tokitex}

\usepackage{graphics}
\usepackage{multirow}
\usepackage{tabto}

\usepackage[english,bahasa]{babel}
\newtranslation[to=bahasa]{Section}{Bagian}
\newtranslation[to=bahasa]{Subsection}{Subbagian}

\usepackage{listings, lstautogobble}
\usepackage{color}

\definecolor{dkgreen}{rgb}{0,0.6,0}
\definecolor{gray}{rgb}{0.5,0.5,0.5}
\definecolor{mauve}{rgb}{0.58,0,0.82}

\lstset{frame=tb,
  language=pascal,
  aboveskip=1mm,
  belowskip=1mm,
  showstringspaces=false,
  columns=fullflexible,
  keepspaces=true,
  basicstyle={\small\ttfamily},
  numbers=none,
  numberstyle=\tiny\color{gray},
  keywordstyle=\color{blue},
  commentstyle=\color{dkgreen},
  stringstyle=\color{mauve},
  breaklines=true,
  breakatwhitespace=true,
  autogobble=true
}

\title{Rekursi}
\author{Tim Olimpiade Komputer Indonesia}
\date{}

\begin{document}

\begin{frame}
\titlepage
\end{frame}

\begin{frame}
\frametitle{Pendahuluan}
Melalui dokumen ini, kalian akan:
\begin{itemize}
  \item Memahami konsep rekursi.
  \item Mempelajarai rekursi tak bercabang.
\end{itemize}
\end{frame}

\begin{frame}
\frametitle{Pendahuluan (lanj.) }
\begin{itemize}
  \item Rekursi adalah keadaan dimana kita membuat sebuah fungsi untuk menyelesaikan sebuah permasalahan dengan cara memanggil diri sendiri secara berulang-ulang
  \item Ketika Anda memanggil fungsi, fungsi itu akan menyelesaikan masalah. Jika terlalu besar, fungsi itu akan memanggil diri sendiri tapi dengan permasalahan yang lebih kecil
  \item Fungsi itu akan memanggil diri sendiri sampai kita berhasil mengecilkan permasalahan itu ke base-case atau permasalah paling simpel
\end{itemize}
\end{frame}

\begin{frame}
\frametitle{$<$TODO: Konten Wajib$>$}
\begin{itemize}
  \item Mengapa perlu ada rekursi
\end{itemize}
\end{frame}

\begin{frame}
\frametitle{Cara mengerjakan soal dengan rekursi }
Ketika kita mengerjakan soal dengan rekursi, kita harus memikirkan 3 perihal:
\begin{itemize}
  \item $Base Case$ Apakah base-case (atau soal paling simpel) dari pertanyaan ini?
  \item Apa yang harus kita lakukan jika soal yang kita terima bukan $base case$?
  \item Bagaimana cara implementasinya dalam fungsi tersebut?
\end{itemize}
\end{frame}

\begin{frame}
\frametitle{Soal: Faktorial}
Deskripsi:
\begin{itemize}
  \item Operasi faktorial (dikenal dengan notasi $N$!) adalah operasi mengali bilangan dari 1 sampai N.
  \item Contoh: Jika $N$ = 4, maka 4!=1 x 2 x 3 x 4 = 24
  \item Bantu Pak Dengklek cari hasil $N$!
\end{itemize}
\end{frame}

\begin{frame}
\frametitle{Soal: Faktorial (lanj.) }
Format masukan:
\begin{itemize}
  \item Sebuah baris berisi sebuah bilangan $N$
\end{itemize}
Format keluaran:
\begin{itemize}
  \item Sebuah baris berisi hasil $N$!
\end{itemize}
Batasan:
\begin{itemize}
  \item $1 \le N \le 10$
\end{itemize}
\end{frame}

\begin{frame}
\frametitle{Solusi}
\begin{itemize}
  \item Permasalahan: Bagaimana cara mengalihkan bilangan dari 1 sampai $N$?
  \item Ide 1: Loop
  \begin{itemize}
    \item Kita set variabel $jawaban$ := 1
    \item Buat sebuah loop dari $i$ sampai $N$
    \item Untuk setiap $i$, $jawaban$ dikali dengan $i$
   \end{itemize}
   \item Ide 2: Rekursi (Akan kita jelaskan)
\end{itemize}
\end{frame}

\begin{frame}[fragile]
\frametitle{Contoh Solusi: faktorial\_loop.pas}
Implementasinya cukup sederhana:
\begin{lstlisting}
  (* Faktorial dengan loop : *)
  var i,N,jawaban:longint;
  begin
    readln(N);
      jawaban:=1; (*Initialisasi*)
    for i:=1 to N do
        jawaban:= jawaban * i;
    writeln(jawaban);
  end.
\end{lstlisting}
\end{frame}

\begin{frame}
\frametitle{Penjelasan solusi Rekursi}
Seperti yang kami jelaskan tadi, mengerjakan soal rekursi ada 3 tahap
\begin{itemize}
  \item Tahap 1 : Base Case
  \begin {itemize}
    \item Jika kita lihat di batasan soal, soal kita akan menjawab soal dari 1! sampai 10!
    \item Pertanyaan : Dari 1! sampai 10!, mana yang paling gampang dihitung?
    \item Yang paling gampang dihitung adalah 1!, jadi 1! adalah base casenya.
    \item 1! = 1
  \end {itemize}
\end{itemize}
\end{frame}

\begin{frame}
\frametitle{Penjelasan solusi Rekursi (lanj.) }
\begin{itemize}
  \item Tahap 2 : Bukan Base Case
  \begin {itemize}
    \item Bagaimana kalau fungsi kita menerima soal yang bukan 1! ?
    \item Sesuai penjelasan di pendahuluan, kita harus mengecilkan masalah!
    \item Setelah dipikir-pikir untuk semua N lebih dari 1. N! = N X (N-1)!
    \item Contoh: 5! = 5 X 4!
    \item Dengan ini, kita telah mengecilkan permasalahan dari 5! jadi 4!
    \item 4! lalu akan kita kecilkan terus menerus sampai menjadi $base case$ yaitu 1!
  \end {itemize}
\end{itemize}
\end{frame}

\begin{frame}[fragile]
\frametitle{Contoh Solusi: faktorial\_rekursi.pas}
Perhatikan contoh berikut:
\begin{lstlisting}
  (* Faktorial dengan rekursi : *)
  function faktorial(Z:longint):longint;
  begin
    if(Z==1)then (* Base Case *)
      f:=1;
    else
      f:=Z*f(Z-1); (* Proses Mengecilkan Masalah*)
   end;
\end{lstlisting}
\end{frame}

\begin{frame}
\frametitle{Penjelasan solusi rekursi}
Perhatikan maksud kami:
\begin {itemize}
  \item Anggap program kita ditanya nilai dari 5! (N=5)
  \item Kita akan panggil faktorial(5)
  \item faktorial(5) akan mengecek, apakah N=5 adalah $base case$
  \item Ternyata bukan! Karena baru $base case$ kalau N=1
  \item Tadi kita ingat kalau 5! = 5 X 4!
  \item Untuk menjawab soal dari nilai 5!, faktorial(5)  butuh nilai dari 4!
  \item Karena dia butuh nilai dari 4! , dia akan panggil faktorial(4)
\end{itemize}
\end{frame}

\begin{frame}
\frametitle{Penjelasan solusi rekursi (lanj.) }
\begin {itemize}
  \item faktorial(4) akan mengecek apa sudah base case! Belum! Kita ingat 4!= 4 X 3! Dia butuh nilai dari 3! maka dia akan panggil faktorial(3)
  \item faktorial(3) akan mengecek apa sudah base case! Belum! Kita ingat 3!= 3 X 2! Dia butuh nilai dari 2! maka dia akan panggil faktorial(2)
  \item faktorial(2) akan mengecek apa sudah base case! Belum! Kita ingat 2!= 2 X 1! Dia butuh nilai dari 1! maka dia akan panggil faktorial (1)
\end{itemize}
\end{frame}

\begin{frame}
\frametitle{Penjelasan solusi rekursi (lanj.) }
\begin {itemize}
  \item faktorial(1) akan mengecek apa sudah base case! Sudah! Karena 1! = 1, dia kembalikan angka 1
  \item faktorial(2) menangkap jawaban 1! dari fakotrial(1). 2! = 2 X 1! = 2 X 1 =2. faktorial(2) mengembalikan angka 2
  \item faktorial(3) menangkap jawaban 2! dari faktorial(2). 3!= 3 X 2! = 3 X 2 =6. faktorial(3) mengembalikan angka 6
  \item Dst sampai faktorial(5) menangkap jawaban 4! dari faktorial(4). 5!= 5 X 4! = 5 X 24 =120.
\end{itemize}
\end{frame}


\begin{frame}
\frametitle{Kompleksitas solusi}
\begin {itemize}
  \item Kompleksitas solusi adalah O($N$). Karena jika kita panggil \alert{\textbf{faktorial(N)}}, pasti terpanggil semua dari \alert{\textbf{faktorial(1)}} sampai \alert{\textbf{f(N)}}.
  \item Jumlah pemanggilan adalah $N$ kali. Makanya O($N$).
\end {itemize}
\end{frame}

\begin{frame}
\frametitle{Materi selanjutnya}
\begin{itemize}
  \item Rekursi bercabang. Di contoh faktorial tadi, program menangkap cuma 1 variabel. Bagaimana kalau program menangkap 2 variabel? (atau bahkan lebih)
\end {itemize}
\end{frame}

\end{document}
