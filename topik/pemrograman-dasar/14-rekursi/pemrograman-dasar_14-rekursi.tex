\documentclass{beamer}
\usetheme{tokitex}

\usepackage{graphics}
\usepackage{multirow}
\usepackage{tabto}

\usepackage[english,bahasa]{babel}
\newtranslation[to=bahasa]{Section}{Bagian}
\newtranslation[to=bahasa]{Subsection}{Subbagian}

\usepackage{listings, lstautogobble}
\usepackage{color}

\definecolor{dkgreen}{rgb}{0,0.6,0}
\definecolor{gray}{rgb}{0.5,0.5,0.5}
\definecolor{mauve}{rgb}{0.58,0,0.82}

\lstset{frame=tb,
  language=pascal,
  aboveskip=1mm,
  belowskip=1mm,
  showstringspaces=false,
  columns=fullflexible,
  keepspaces=true,
  basicstyle={\small\ttfamily},
  numbers=none,
  numberstyle=\tiny\color{gray},
  keywordstyle=\color{blue},
  commentstyle=\color{dkgreen},
  stringstyle=\color{mauve},
  breaklines=true,
  breakatwhitespace=true,
  autogobble=true
}

\title{Pengenalan rekursi}
\author{Tim Olimpiade Komputer Indonesia}
\date{}

\begin{document}

\begin{frame}
\titlepage
\end{frame}

\begin{frame}
\frametitle{Pendahuluan}
Melalui dokumen ini, kalian akan:
\begin{itemize}
  \item Memahami konsep rekursi.
  \item Mempelajari rekursi sederhana.
\end{itemize}
\end{frame}

\begin{frame}
\frametitle{Pengenalan Rekursi}
\begin{itemize}
  \item Rekursi adalah keadaan yang mana sebuah fungsi menyelesaikan sebuah permasalahan dengan cara memanggil diri sendiri secara berulang kali.
  \item Jika masalah sudah cukup kecil, maka fungsi rekursi dapat langsung menghasilkan jawaban.
  \item Jika masalah terlalu besar, maka fungsi akan memanggil diri sendiri dengan cakupan masalah yang lebih kecil.
\end{itemize}
\end{frame}

\begin{frame}
\frametitle{Mengapa Perlu Ada Rekursi}
\begin{itemize}
  \item Banyak permasalahan yang lebih mudah diselesaikan (dan pendek kodenya) jika menggunakan pendekatan rekursif.
  \item Pada dasarnya, strategi iteratif (misalnya dengan \textit{for loop}) dan rekursif sama-sama melakukan sesuatu yang berulang-ulang. 
  \item Namun, terkadang solusi iteratif  untuk suatu masalah sangat sulit untuk dipikirkan dan memerlukan teknik khusus.
  \item Dengan solusi rekursif, mungkin saja lebih mudah untuk melihat dan merancang alur penyelesaiannya.
\end{itemize}
\end{frame}

\begin{frame}
\frametitle{Strategi Rekursif}
Terdapat dua hal yang perlu dipikirkan ketika menggunakan strategi rekursif:
\begin{itemize}
  \item \textit{Base case}
  
  Apa kasus paling sederhana dari permasalahan ini?
  
  \item \textit{Recurrence relation}
  
  Bagaimana hubungan rekursif dari persoalan ini dengan persoalan serupa yang lebih kecil?
\end{itemize}
\end{frame}

\begin{frame}
\frametitle{Contoh Soal: Faktorial}
Deskripsi:
\begin{itemize}
  \item Pak Dengklek baru mempelajari konsep matematika baru, yaitu faktorial.
  \item Operasi faktorial pada $N$, atau ditulis dengan notasi $N$!, adalah operasi mengalikan bilangan dari 1 sampai N.
  \item Contoh: Jika $N = 4$, maka $4! = 1 \times 2 \times 3 \times 4 = 24$
  \item Diberikan $N$, bantu Pak Dengklek mencari hasil $N$!
\end{itemize}
\end{frame}

\begin{frame}
\frametitle{Contoh Soal: Faktorial (lanj.) }
Format masukan:
\begin{itemize}
  \item Sebuah baris berisi sebuah bilangan $N$
\end{itemize}
Format keluaran:
\begin{itemize}
  \item Sebuah baris berisi hasil $N$!
\end{itemize}
Batasan:
\begin{itemize}
  \item $1 \le N \le 10$
\end{itemize}
\end{frame}

\begin{frame}
\frametitle{Solusi}
\begin{itemize}
  \item Ide 1:
  \begin{itemize}
    \item Cukup gunakan \textit{for loop} biasa
    \item Solusi ini bekerja secara iteratif.
  \end{itemize}
  \item Ide 2: Rekursi 
\end{itemize}
\end{frame}

\begin{frame}[fragile]
\frametitle{Contoh Solusi Iteratif}
Implementasi solusi secara iteratif cukup sederhana:
\begin{lstlisting}    
  function faktorial(x: longint): longint;
  var
    jawaban: longint;
  begin
    jawaban := 1; 
    for i := 1 to x do begin
      jawaban := jawaban * i;
    end;
    
    faktorial := jawaban;
  end;
\end{lstlisting}
\end{frame}

\begin{frame}
\frametitle{Solusi Rekursif}
\textit{Base Case}
\begin{itemize}
  \item Pada batasan soal, nilai $N$ berkisar antara 1 sampai 10.
  \item Dari batasan tersebut, kasus terkecilnya adalah $N=1$.
  \item Jadi $N=1$ adalah \textit{base case}, dan memang jelas diketahui bahwa 1! = 1.
\end{itemize}
\end{frame}

\begin{frame}
\frametitle{Solusi Rekursif (lanj.) }
\textit{Recurrence Relation}
\begin{itemize}
  \item Bagaimana jika $N > 1$?
  \item Sadari bahwa untuk mencari $N!$, diperlukan informasi $(N-1)!$.
  \item Jika $(N-1)!$ diketahui, bisa dihitung $N! = N \times (N-1)!$
  \item Dengan observasi ini, kita mengetahui hubungan rekursif dari $N!$, yaitu $N! = N \times (N-1)!$.
\end{itemize}
\end{frame}

\begin{frame}
\frametitle{Contoh Pengerjaan}
\begin{itemize}
  \item Jadi untuk menghitung $4!$, nilainya adalah $4 \times 3\!$.
  \item Berhubung program tidak tahu nilai dari $3!$, maka kasus diperkecil lebih lanjut: $3 \times 2!$
  \item Program juga tidak tahu nilai dari $2!$, sehingga kasus diperkecil lebih lanjut: $2 \times 1!$.
\end{itemize}
\end{frame}

\begin{frame}
\frametitle{Contoh Pengerjaan (lanj.)}
\begin{itemize}
  \item $1!$ sudah diketahui nilainya, yaitu $1$. Jadi $2! = 2 \times 1 = 2$.
  \item $2!$ sudah diketahui nilainya, yaitu $2$. Jadi $3! = 3 \times 2 = 6$.
  \item $3!$ sudah diketahui nilainya, yaitu $6$. Jadi $4! = 4 \times 6 = 24$.
  \item Kini diketahui $4! = 24$, dan persoalan selesai.
\end{itemize}
\end{frame}

\begin{frame}[fragile]
\frametitle{Contoh Solusi: faktorial\_rekursif.pas}
Berikut implementasi pencarian faktorial secara rekursif:
\begin{lstlisting}
  function faktorial(x: longint): longint;
  begin
    if (x = 1) then begin
      faktorial := 1
    end else begin
      faktorial := x * faktorial(x-1); 
    end;
  end;
\end{lstlisting}
\end{frame}

\begin{frame}
\frametitle{Contoh Eksekusi Fungsi}
\begin{itemize}
  \item Misalkan hendak dicari nilai $4!$ ($N=4$)
  \item Panggil faktorial(4)
  \item faktorial(4) akan memeriksa, apakah $N=1$
  \item Ternyata bukan, sehingga masuk ke bagian \textit{recurrence relation}.
  \item Untuk mencari faktorial(4), diperlukan nilai $3!$, sehingga dipanggil faktorial(3).
\end{itemize}
\end{frame}

\begin{frame}
\frametitle{Contoh Eksekusi Fungsi (lanj.) }
\begin{itemize}
  \item faktorial(3) juga masih bukan \textit{base case},
  
  sehingga panggil faktorial(2).
  \item faktorial(2) juga masih bukan \textit{base case}, 
  
  sehingga panggil faktorial(1).
  \item faktorial(1) merupakan \textit{base case}, 
  
  sehingga langsung dikembalikan hasilnya, yaitu 1.
  
  \item Nilai ini dikembalikan ke fungsi yang memanggil faktorial(1), yaitu faktorial(2).
 
\end{itemize}
\end{frame}

\begin{frame}
\frametitle{Contoh Eksekusi Fungsi (lanj.) }
\begin{itemize}
  \item faktorial(2) menerima jawaban dari faktorial(1), sehingga "faktorial(2) = 2 $\times$ faktorial(1)" kini bisa dikerjakan dan dihasilkan nilai 2.
  \item faktorial(3) menangkap jawaban dari faktorial(2), sehingga "faktorial(3) = 3 $\times$ faktorial(2)" kini bisa dikerjakan dan dihasilkan nilai 6.
  \item faktorial(4) menangkap jawaban dari faktorial(3), sehingga "faktorial(4) = 4 $\times$ faktorial(3)" kini bisa dikerjakan dan dihasilkan nilai 24.
  \item Pemanggilan fungsi selesai, dan disimpulkan faktorial(4) = 24.
\end{itemize}
\end{frame}

\begin{frame}
\frametitle{Kompleksitas solusi}
\begin{itemize}
  \item Baik secara iteratif maupun rekursif, kompleksitasnya adalah $O(N)$.
  \item Setiap pemanggilan rekursif membutuhkan alokasi memori, sehingga jika pemanggilannya semakin dalam, semakin banyak tambahan memori yang digunakan.
  \item Waktu untuk mengalokasikan memori juga menyebabkan solusi rekursif cenderung bekerja lebih lambat dibandingkan solusi iteratif.
\end{itemize}
\end{frame}

\begin{frame}
\frametitle{Materi Selanjutnya}
\begin{itemize}
  \item Pada pembelajaran ini, rekursi yang digunakan masih sangat sederhana.
  \item Bahkan belum terasa bahwa solusi rekursi lebih mudah dan pendek kodenya dibandingkan solusi iteratif.
  \item Pembelajaran selanjutnya tentang rekursi yang lebih kompleks akan menunjukkan hal tersebut.
\end{itemize}
\end{frame}

\end{document}
