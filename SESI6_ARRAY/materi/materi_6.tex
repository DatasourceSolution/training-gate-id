\documentclass{beamer}
\usetheme{tokitex}
\usepackage{graphics}
\usepackage{multirow}
\usepackage{multicol}
\usepackage{tabto}
\usepackage[english,bahasa]{babel}
\newtranslation[to=bahasa]{Section}{Bagian}
\newtranslation[to=bahasa]{Subsection}{Subbagian}

\usepackage{listings}
\usepackage{color}

\definecolor{dkgreen}{rgb}{0,0.6,0}
\definecolor{gray}{rgb}{0.5,0.5,0.5}
\definecolor{mauve}{rgb}{0.58,0,0.82}

\lstset{frame=tb,
  language=pascal,
  showstringspaces=false,
  columns=flexible,
  basicstyle={\small\ttfamily},
  numbers=none,
  numberstyle=\tiny\color{gray},
  keywordstyle=\color{blue},
  commentstyle=\color{dkgreen},
  stringstyle=\color{mauve},
  breaklines=true,
  breakatwhitespace=true,
  tabsize=2
}

\title{Pengenalan Array}
\author{Tim Olimpiade Komputer Indonesia}

\begin{document}

\begin{frame}
\titlepage
\end{frame}

\begin{frame}
\frametitle{Pendahuluan}
Melalui dokumen ini, kalian akan:
\begin{itemize}
	\item Memahami konsep array.
	\item Mengimplementasikan array pada bahasa Pascal.
	\item Menggunakan array untuk penyelesaian beberapa contoh masalah.
\end{itemize}
\end{frame}

\begin{frame}
\frametitle{Motivasi}
\begin{itemize}
	\item Pak Dengklek memiliki sebuah tumpukan berisi $N$ kartu, yang dipenuhi $1 \le N \le 100$.
	\item Setiap kartu bertuliskan suatu bilangan bulat.
	\item Sekarang Pak Dengklek ingin tahu urutan angka-angka pada kartu tersebut bila tumpukan kartu itu dibalik.
	\item Contoh: jika diberikan 5 kartu dengan angka-angka dari atasnya [1, 5, 3, 20, 4], maka setelah dibalik urutannya menjadi: [4, 20, 3, 5, 1].
	\item Bantulah Pak Dengklek menentukan urutan angka-angka tersebut setelah tumpukan kartu dibalik!
\end{itemize}
\end{frame}

\begin{frame}[fragile]
\frametitle{Solusi?}
\begin{itemize}
	\item Sederhana, idenya adalah dengan menampung seluruh bilangan terlebih dahulu, baru dicetak dalam urutan terbalik.
	\item Misalnya jika $N$ selalu 3, kita bisa membuat 3 variabel (misalnya a, b, c), lalu:
	\begin{lstlisting}
	readln(a);
	readln(b);
	readln(c);
	
	writeln(c);
	writeln(b);
	writeln(a);
	\end{lstlisting}
	\item Sayangnya nilai $N$ tidak tetap! Dibutuhkan suatu mekanisme lain untuk menggunakan dan mengakses variabel!
\end{itemize}
\end{frame}

\section{Konsep Array}
\frame{\sectionpage}

\begin{frame}
\frametitle{Pengertian Array}
\begin{block}{Array}
Variabel dengan satu nama, tetapi mengandung banyak nilai.
Akses nilai-nilainya dilakukan dengan indeks.
\end{block}
\vfill
Perhatikan contoh berikut!
\vfill
\begin{tabular}{|c|c|c|c|c|c|c|c|c|c|c|}
\hline indeks & 1 & 2 & 3 & 4 & 5 & 6 & 7 & 8 & 9 & 10 \\ 
\hline A & 3 & 10 & 11 & 23 & 35 & 12 & 31 & 53 & 0 & 19 \\ 
\hline 
\end{tabular} 

\begin{itemize}
	\item A[1] = 3
	\item A[2] = 10
	\item A[5] = 35
\end{itemize}
\end{frame}

\begin{frame}
\frametitle{Penjelasan}
\begin{itemize}
	\item Pada contoh sebelumnya, kita memiliki sebuah variabel bernama \textbf{A}.
	\item \textbf{A} memiliki 10 nilai, yang masing-masing dapat diakses dengan indeks.
	\item Untuk mengakses nilai \textbf{A} yang ke-x, digunakan \textbf{A[x]}.
	\item Lebih jauh lagi, sebenarnya \textbf{A[x]} bisa dianggap sebagai sebuah variabel yang berdiri sendiri.
	\item Konsep inilah yang disebut sebagai array!
\end{itemize}
\end{frame}

\section{Implementasi Array Pada Pascal}
\frame{\sectionpage}

\begin{frame}[fragile]
\frametitle{Deklarasi}
\begin{itemize}
	\item Karena array merupakan variabel, diperlukan deklarasi seperti variabel lainnya.
	\item Format deklarasi array adalah:
	\begin{lstlisting}
	<nama>: array[<nilai awal>..<nilai akhir>] of <tipe>;
	\end{lstlisting}
	\item Dengan $<$nama$>$ adalah nama dari array (aturan penamaan sama seperti variabel biasanya), $<$nilai awal$>$ dan $<$nilai akhir$>$ adalah rentang indeks array terdefinisi, dan $<$tipe$>$ adalah tipe data dari array.
	\item Tentu saja, tipe data di sini bisa berupa \textbf{longint}, \textbf{double}, \textbf{string}, \textbf{boolean} atau suatu \textbf{record}.
\end{itemize}
\end{frame}

\begin{frame}[fragile]
\frametitle{Contoh Deklarasi}
Berikut ini adalah contoh deklarasi array pada Pascal:
\begin{lstlisting}
var
  tabel: array[0..100] of boolean;
  bukuTelepon: array[1..1000] of string;
  frekuensi: array[-1000..1000] of longint;
\end{lstlisting}
\begin{itemize}
	\item Untuk contoh array \textbf{tabel}, hanya tabel[0], tabel[1], tabel[2], ..., tabel[100] yang terdefinisi.
	\item Mengakses nilai tabel[0], tabel[-2], atau tabel[500] akan menyebabkan \textbf{runtime error}.
	\item Untuk itu, tentukan rentang indeks yang akan kalian gunakan saat deklarasi dengan tepat (sesuai kebutuhan).
\end{itemize}
\end{frame}

\begin{frame}[fragile]
\frametitle{Array dan Variabel}
\begin{itemize}
	\item Karena suatu elemen dari array juga bisa dianggap variabel, tentu saja kita bisa melakukan perintah \textbf{readln} padanya.
	\item Sebagai contoh, jika kita memiliki array bernama \textbf{tabel} yang terdefinisi dari 1 sampai dengan \textbf{100}, kita bisa melakukan:
	\begin{lstlisting}
	readln(tabel[2]);
	\end{lstlisting}
\end{itemize}
\end{frame}

\begin{frame}[fragile]
\frametitle{Array dan Variabel (lanj.)}
\begin{itemize}
	\item Jika diberikan \textbf{5} bilangan, dan kita perlu menyimpan masing-masing bilangan di tabel, kita bisa melakukan:
	\begin{lstlisting}
	readln(tabel[1]);
	readln(tabel[2]);
	readln(tabel[3]);
	readln(tabel[4]);
	readln(tabel[5]);		
	\end{lstlisting}
	\item Tentu saja hal ini sangat tidak efisien!
	\item Untungnya, kita sudah mempelajari sebuah teknik yang sangat penting, yaitu \textbf{perulangan}.
\end{itemize}
\end{frame}

\begin{frame}[fragile]
\frametitle{Array dan Variabel (lanj.)}
\begin{itemize}
	\item Proses membaca 5 bilangan pada 5 baris kini bisa dilakukan dengan cara:
	\begin{lstlisting}
	for i := 1 to 5 do begin
	   readln(tabel[i]);
	end;
	\end{lstlisting}
	\item Untuk kasus umum, yaitu ketika diberikan $N$ bilangan, cukup ganti angka 5 dengan variabel $N$.
	\begin{lstlisting}
	for i := 1 to N do begin
	   readln(tabel[i]);
	end;
	\end{lstlisting}
\end{itemize}
\end{frame}

\begin{frame}[fragile]
\frametitle{Array dan Variabel (lanj.)}
\begin{itemize}
	\item Demikian pula untuk pencetakan secara terbalik, kita bisa menggunakan perulangan sebagai berikut:
	\begin{lstlisting}
	for i := N downto 1 do begin
	   readln(tabel[i]);
	end;
	\end{lstlisting}
	\item Sekarang masalah Pak Dengklek terpecahkan!
\end{itemize}
\end{frame}

\begin{frame}[fragile]
\frametitle{Contoh Solusi: balik.pas}
Berikut contoh solusi lengkap untuk permasalahan motivasi:
\begin{lstlisting}
var
   N, i: longint;
   tabel: array[1..100] of longint;

begin
   readln(N);

   for i := 1 to N do begin
      readln(tabel[i]);
   end;

   for i := N downto 1 do begin
      writeln(tabel[i]);
   end;
end.
\end{lstlisting}
\end{frame}


\begin{frame}
\frametitle{Array dan Memori}
\begin{itemize}
	\item Setiap elemen pada array membutuhkan memori, bergantung pada tipe data yang digunakan.
	\item Total memori yang dibutuhkan untuk sebuah array sama dengan banyaknya elemennya dikali ukuran memori satu elemennya.
	\item Sebagai contoh, array dengan 100 elemen dan memiliki tipe \textbf{longint} membutuhkan memori sebesar $100 \times 4$ byte $= 400$ byte, 
\end{itemize}
\end{frame}

\begin{frame}
\frametitle{Rentang Array}
\begin{itemize}
	\item Pada \textbf{balik.pas}, dideklarasikan array sebesar 100 elemen (dari 1 sampai dengan 100), padahal bisa jadi hanya digunakan sebagian saja.
	\item Cara ini memang "boros" memori, tetapi ingat bahwa kita harus mendeklarasikan array tersebut di awal, yang mana pada saat itu tidak diketahui berapa nilai $N$.
	\item Dengan demikian, cara yang paling mudah adalah mendeklarasikannya sebesar nilai $N$ maksimal yang mungkin.
\end{itemize}
\end{frame}

\begin{frame}
\frametitle{Contoh Soal: Ujian Harian}
Deskripsi:
\begin{itemize}
	\item Pak Dengklek menyelenggarakan ujian harian setelah selesai mengajarkan $N$ ekor bebeknya mengenai konsep array.
	\item Setiap bebek ke-i mendapatkan nilai sebesar \textbf{$h_i$}, yang merupakan bilangan bulat.
	\item Untuk menentukan lulus atau tidaknya seekor bebek, nilai bebek tersebut harus tidak kurang dari nilai rata-rata dari seluruh bebek.
	\item Tentukan banyaknya bebek yang lulus ujian!
\end{itemize}
Batasan:
\begin{itemize}
	\item $1 \le N \le 100$
	\item $1 \le h_i \le 100$, untuk $1 \le i \le N$
\end{itemize}
\end{frame}

\begin{frame}
\frametitle{Contoh Soal: Ujian Harian (lanj.)}
Format masukan:
\begin{itemize}
	\item Baris pertama berisi sebuah bilangan bulat $N$.
	\item $N$ baris berikutnya berisi nilai ujian bebek. Baris ke-$i$ ini merupakan $h_i$.
\end{itemize}
Format keluaran:
\begin{itemize}
	\item Sebuah baris yang menyatakan banyaknya bebek yang lulus ujian.
\end{itemize}
\end{frame}

\begin{frame}[fragile]
\frametitle{Contoh Soal: Ujian Harian (lanj.)}
Contoh masukan:
\begin{lstlisting}
3
5
6
7
\end{lstlisting}
Contoh keluaran:
\begin{lstlisting}
2
\end{lstlisting}
\begin{block}{Penjelasan}
Nilai rata-rata dari seluruh bebek adalah 6, dan terdapat 2 ekor bebek yang nilainya tidak kurang dari 6.
\end{block}
\end{frame}

\begin{frame}
\frametitle{Petunjuk}
\begin{itemize}
	\item Salah satu solusinya adalah melalui dua tahap:
	\begin{enumerate}
		\item Hitung rata-ratanya.
		\item Hitung banyaknya bebek yang nilainya tidak kurang dari rata-rata.
	\end{enumerate}
	\item Sebisa mungkin, hindari penggunaan \textbf{\alert{floating-point}}!
	\begin{itemize}
		\item Ingat bahwa tipe data \textbf{floating-point} kurang bisa menyatakan bilangan secara akurat; nilai 1/3*3 bisa jadi 0.999999999999999 atau 1.0000000000001.
		\item Pengoperasian tipe data \textbf{integer} oleh komputer jauh lebih cepat daripada pengoperasian tipe data \textbf{floating-point}!
	\end{itemize}
\end{itemize}
\end{frame}

\begin{frame}[fragile]
\frametitle{Contoh Solusi: lulus.pas}
\begin{lstlisting}
var
   N, i, total, lulus: longint;
   h: array[1..100] of longint;
begin
   readln(N);
   for i := 1 to N do begin
      readln(h[i]);
   end;

   total := 0;
   for i := 1 to N do begin
      total := total + h[i];
   end;
\end{lstlisting}
\end{frame}

\begin{frame}[fragile]
\frametitle{Contoh Solusi: lulus.pas (lanj.)}
\begin{lstlisting}
   lulus := 0;
   for i := 1 to N do begin
      (* trik menghindari pembagian *)
      if (h[i]*N >= total) then begin 
         lulus := lulus + 1;
      end;
   end;

   writeln(lulus);
end.
\end{lstlisting}
\end{frame}

\section{Penggunaan Array Lanjutan}
\frame{\sectionpage}

\begin{frame}[fragile]
\frametitle{Array Dua Dimensi}
\begin{itemize}
	\item Struktur array bisa juga membentuk sebuah tabel dua dimensi.
	\item Perhatikan contoh deklarasi berikut:
	\begin{lstlisting}
	matriks: array[1..3, 1..5] of longint;
	\end{lstlisting}
	\item Kini kita mendapatkan variabel bernama $matriks[a][b]$, yang terdefinisi untuk $1 \le a \le 3$ dan $1 \le b \le 5$.
	\item Akses suatu elemen dapat dilakukan dengan dua cara, yaitu matriks[a][b] atau matriks[a,b].
	\item Tabel berikut menunjukkan struktur dari array matriks:
	\begin{table}[h]
		\begin{tabular}{c|c|c|c|c|c|}
			  & 1 & 2 & 3 & 4 & 5\\ 
			\hline 1 & & & & & \\ 
			\hline 2 & & & & & \\ 
			\hline 3 & & & & & \\ 			
			\hline
		\end{tabular}
	\end{table}  
	\item Aturan perhitungan memori tetap sama; banyaknya elemen dikali memori per elemennya.\newline Pada kasus ini: $3 \times 5 \times 4$ byte $= 60$ byte.
\end{itemize}
\end{frame}

\begin{frame}
\frametitle{Contoh Soal:\newline Cokelat Bebek}
Deskripsi:
\begin{itemize}
	\item Pak Ganesh datang bertamu ke peternakan bebek Pak Dengklek.
	\item Pada peternakan bebek Pak Dengklek, terdapat kandang bebek yang tersusun atas petak-petak $N$ baris dan $N$ kolom.
	\item Pak Dengklek memberi $d_{i,j}$ gram cokelat* ke kandang di baris ke-$i$ dan kolom ke-$j$.
	\item Pak Ganesh memberi $g_{i,j}$ gram cokelat* ke kandang di baris ke-$i$ dan kolom ke-$j$.
	\item Tentukan berapa gram cokelat yang diperoleh setiap bebek di kandangnya!
\end{itemize}
Batasan:
\begin{itemize}
	\item $1 \le N \le 100$
	\item $0 \le d_{i,j}, h_{i,j} \le 10$, untuk $1 \le i,j \le N$
\end{itemize}

\tiny *Catatan: bebek-bebek suka cokelat! 
\end{frame}

\begin{frame}
\frametitle{Contoh Soal:\newline Cokelat Bebek (lanj.)}
\begin{itemize}
	\item Sebagai contoh, misalkan $N = 3$.
	\item Kemudian berikut adalah cokelat yang diberikan Pak Dengklek ($D$) dan Pak Ganesh ($G$):
	\vfill
	\(D = 
	\left[\begin{matrix}
	1 & 3 & 0 \\
	6 & 2 & 4 \\
	2 & 1 & 5 
	\end{matrix}\right]
	\)
	\hfil
	\(G =
	\left[\begin{matrix}
	2 & 1 & 7 \\
	0 & 0 & 1 \\
	1 & 1 & 2 
	\end{matrix}\right]
	\) \centering
	\item Maka total cokelat yang didapatkan setiap kandang adalah:
	\vfill
	\(
	\left[\begin{matrix}
	3 & 4 & 7 \\
	6 & 2 & 5 \\
	3 & 2 & 7 
	\end{matrix}\right]
	\) \centering
\end{itemize}
\end{frame}

\begin{frame}
\frametitle{Contoh Soal:\newline Cokelat Bebek (lanj.)}
Format masukan:
\begin{itemize}
	\item Baris pertama berisi sebuah bilangan bulat $N$.
	\item $N$ baris berikutnya berisi $N$ bilangan. Bilangan di baris ke-$i$ dan kolom ke-$j$ ini adalah $d_{i,j}$.
	\item $N$ baris sisanya berisi $N$ bilangan. Bilangan di baris ke-$i$ dan kolom ke-$j$ ini adalah $g_{i,j}$.
\end{itemize}
Format keluaran:
\begin{itemize}
	\item $N$ baris yang berisi $N$ bilangan. Bilangan di baris ke-$i$ dan kolom ke-$j$ ini adalah total makanan yang ada di kandang baris ke-$i$ dan kolom ke-$j$.
\end{itemize}
\end{frame}

\begin{frame}[fragile]
\frametitle{Contoh Soal:\newline Cokelat Bebek (lanj.)}
Contoh masukan:
\begin{lstlisting}
3
1 3 0
6 2 4
2 1 5
2 1 7
0 0 1
1 1 2
\end{lstlisting}
Contoh keluaran:
\begin{lstlisting}
3 4 7
6 2 5
3 2 7
\end{lstlisting}
\end{frame}

\begin{frame}
\frametitle{Petunjuk}
\begin{itemize}
	\item Salah satu cara yang mudah adalah membuat tiga array dua dimensi, masing-masing untuk menampung makanan yang diberikan Pak Dengklek ($D$), Pak Ganesh ($G$), dan hasil akhirnya ($hasil$).
	\item Tentu saja hubungannya adalah $hasil[i][j] = D[i][j] + G[i][j]$, untuk $1 \le i,j \le N$.
\end{itemize}
\end{frame}

\begin{frame}[fragile]
\frametitle{Solusi: cokelat.pas}
Pertama, mari kita deklarasikan variabel yang akan digunakan:
\begin{lstlisting}
var
   N: longint;
   D, G, hasil: array[1..100, 1..100] of longint;
   i, j: longint;
   
\end{lstlisting}
\end{frame}

\begin{frame}[fragile]
\frametitle{Solusi: cokelat.pas (lanj.)}
Kemudian baca masukan sesuai dengan format yang diberikan:
\begin{lstlisting}
begin
   readln(N);
   
   for i := 1 to N do begin
      for j := 1 to N do begin
         read(D[i][j]);
      end;
      readln;
   end;

   for i := 1 to N do begin
      for j := 1 to N do begin
         read(G[i][j]);
      end;
      readln;
   end;
\end{lstlisting}
\end{frame}

\begin{frame}[fragile]
\frametitle{Solusi: cokelat.pas (lanj.)}
Lakukan penjumlahan, lalu cetak hasilnya:
\begin{lstlisting}
   for i := 1 to N do begin
      for j := 1 to N do begin
         hasil[i][j] := D[i][j] + G[i][j];
      end;
   end;   

   for i := 1 to N do begin
      for j := 1 to N do begin
         write(hasil[i][j]);
         if (j < N) then 
            write(' ');
      end;
      writeln;
   end;
end.
\end{lstlisting}
\end{frame}

\begin{frame}[fragile]
\frametitle{Solusi: cokelat\_2.pas}
Nilai array $D$ dan $G$ sebenarnya tidak perlu disimpan, kita bisa menghemat memori dengan langsung menjumlahkannya. 
\begin{lstlisting}
var
   N: longint;
   temp: longint;
   hasil: array[1..100, 1..100] of longint;
   i, j: longint;

begin
   readln(N);
   
   for i := 1 to N do begin
      for j := 1 to N do begin
         read(temp);
         hasil[i][j] := temp;
      end;
      readln;
   end;
\end{lstlisting}
\end{frame}

\begin{frame}[fragile]
\frametitle{Solusi: cokelat\_2.pas (lanj.)}
\begin{lstlisting}
   for i := 1 to N do begin
      for j := 1 to N do begin
         read(temp);
         hasil[i][j] := hasil[i][j] + temp;
      end;
      readln;
   end;

   for i := 1 to N do begin
      for j := 1 to N do begin
         write(hasil[i][j]);
         if (j < N) then 
            write(' ');
      end;
      writeln;
   end;
end.
\end{lstlisting}
\end{frame}

\begin{frame}[fragile]
\frametitle{Array Multidimensi}
\begin{itemize}
	\item Tidak hanya sampai dua dimensi, dimensi tiga, empat, atau lebih pun bisa.
	\item Sebagai contoh:
	\begin{lstlisting}
	data: array[1..2, 1..50, 1..50] of longint;
	\end{lstlisting}
	\item Kita akan mendapatkan variabel $data[i][j][k]$ yang terdefinisi untuk $1 \le i \le 2$, dan $1 \le j, k \le 50$.
	\item Akses elemen juga bisa dilakukan dengan $data[i,j,k]$.
\end{itemize}
\end{frame}

\begin{frame}
\frametitle{Selanjutnya...}
\begin{itemize}
	\item Setelah mempelajari konsep array, kita akan mengupas lebih dalam tentang string.
	\item Selamat berlatih untuk memperkuat pemahaman tentang array!
\end{itemize}
\end{frame}

\end{document}